\documentclass[10pt]{amsbook}%
\usepackage{amsmath}%
\usepackage{amsfonts}%
\usepackage{mathabx}%
\usepackage{mathtools}%
\usepackage{amssymb}%
\usepackage{graphicx}
\usepackage{enumerate}
\newcommand{\ii}{\item}
\newenvironment{subproof}[1][Proof]{
\begin{proof}[#1] \renewcommand{\qedsymbol}{∎}}
{\end{proof}}
\newcommand\defeq{\stackrel{\mathclap{\normalfont\mbox{\small{def}}}}{=}}

\theoremstyle{plain}
\theoremstyle{definition}
\newtheorem{definition}[theorem]{Definition}
\newtheorem{definition*}{Definition}
\newtheorem{acknowledgement}{Acknowledgement}
\newtheorem{algorithm}{Algorithm}
\newtheorem{axiom}{Axiom}
\newtheorem{case}{Case}
\newtheorem{claim}{Claim}
\newtheorem{conclusion}{Conclusion}
\newtheorem{condition}{Condition}
\newtheorem{conjecture}{Conjecture}
\newtheorem{corollary}{Corollary}
\newtheorem{criterion}{Criterion}
\newtheorem{example}[theorem]{Example}
\newtheorem*{example*}{Example}
\newtheorem{exercise}{Exercise}
\newtheorem{lemma}{Lemma}
\newtheorem{notation}{Notation}
\newtheorem{problem}{Problem}
\newtheorem{proposition}[theorem]{Proposition}
\newtheorem{proposition*}{Proposition}
\newtheorem{remark}{Remark}
\newtheorem{solution}{Solution}
\newtheorem{summary}{Summary}
\newtheorem{theorem}{Theorem}
\numberwithin{equation}{section}
\makeatletter
\renewcommand\tocsection[3]{%
  \indentlabel
  {\vspace{0.25em} \@ifnotempty{#2}{\ignorespaces #1 \S#2.\quad }}#3%
}
\makeatother
\usepackage[usenames,dvipsnames]{xcolor}
\usepackage[colorlinks=true,linkcolor=Red,citecolor=Green]{hyperref}
\usepackage{mdframed}
\makeatletter
\let\theorem\undefined
\let\c@theorem\undefined
\let\lemma\undefined
\let\c@lemma\undefined
\let\proposition\undefined
\let\c@proposition\undefined
\makeatother
\newmdtheoremenv[outerlinewidth=2,leftmargin=20,
  rightmargin=20,backgroundcolor=white,
  outerlinecolor=blue,innertopmargin=0.1\topskip,
  innerbottommargin=10,
  splittopskip=\topskip,skipbelow=\baselineskip,
skipabove=\baselineskip]
{theorem}
{Theorem}

\newmdtheoremenv[outerlinewidth=2,leftmargin=20,
  rightmargin=20,backgroundcolor=white,
  outerlinecolor=blue,innertopmargin=0.1\topskip,
  innerbottommargin=10,
  splittopskip=\topskip,skipbelow=\baselineskip,
skipabove=\baselineskip]
{lemma}
{Lemma}

\newmdtheoremenv[outerlinewidth=2,leftmargin=20,
  rightmargin=20,backgroundcolor=white,
  outerlinecolor=blue,innertopmargin=0.1\topskip,
  innerbottommargin=10,
  splittopskip=\topskip,skipbelow=\baselineskip,
skipabove=\baselineskip]
{proposition}
{Proposition}
%-----------------------------------------------------------
% Macros
\newcommand{\CC}{\mathbb C}
\newcommand{\FF}{\mathbb F}
\newcommand{\NN}{\mathbb N}
\newcommand{\QQ}{\mathbb Q}
\newcommand{\RR}{\mathbb R}
\newcommand{\HH}{\mathbb H}
\newcommand{\ZZ}{\mathbb Z}
\newcommand{\HC}{\mathcal H}
\newcommand{\OC}{\mathcal O}
\newcommand{\CK}{\mathcal C}
\newcommand{\AK}{\mathcal A}
\newcommand{\BK}{\mathcal B}
\newcommand{\DK}{\mathcal D}
\newcommand{\IF}{\mathfrak I}
\newcommand{\CL}{\mathcal L}
\newcommand{\ML}{\mathcal L}
\newcommand{\BC}{\mathcal B}
\newcommand{\SC}{\mathcal S}
\newcommand{\CR}{\mathcal R}
\newcommand{\af}{\mathfrak a}
\newcommand{\sff}{\mathfrak s}
\newcommand{\bff}{\mathfrak b}
\newcommand{\rf}{\mathfrak r}
\newcommand{\mf}{\mathfrak m}
\newcommand{\MF}{\mathfrak M}
\newcommand{\pf}{\mathfrak p}
\newcommand{\ifr}{\mathfrak i}
\newcommand{\qf}{\mathfrak q}
\newcommand{\if}{\mathfrak i}
\newcommand{\PF}{\mathfrak P}
\newcommand{\QF}{\mathfrak Q}
\newcommand{\UF}{\mathfrak U}
\newcommand{\card}{\text{card}}
\newcommand{\Gal}{\text{Gal}}
\newcommand{\Fix}{\text{Fix}}
\DeclareMathOperator{\Null}{null}
\DeclareMathOperator{\rk}{rk}
\newcommand{\charin}{\text{ char }}
\renewcommand{\proof}{ \textbf{Proof: }}
\renewcommand{\counterexample}{ \textbf{Counterexample: }}
\renewcommand{\ie}{i.e., }
\renewcommand{\eg}{e.g., }
\newcommand{\lrangle}[1]{\langle \text{#1} \rangle}
\DeclareMathOperator{\sign}{sign}
\DeclareMathOperator{\Aut}{Aut}
\newcommand{\Stab}[1]{\text{Stab}(#1)}
\DeclareMathOperator{\Inn}{Inn}
\DeclareMathOperator{\Ann}{Ann}
\DeclareMathOperator{\Irr}{Irr}
\DeclareMathOperator{\hgt}{hgt}
\DeclareMathOperator{\Syl}{Syl}
\DeclareMathOperator{\Core}{Core}
\DeclareMathOperator{\PSL}{PSL}
\DeclareMathOperator{\Hom}{Hom}
\DeclareMathOperator{\Gal}{Gal}
\DeclareMathOperator{\Rad}{Rad}
\DeclareMathOperator{\Ker}{Ker}
\DeclareMathOperator{\Syl}{Syl}
\DeclareMathOperator{\lcm}{lcm}
\DeclareMathOperator{\gcd}{gcd}
\DeclareMathOperator{\im}{Im}
\DeclareMathOperator{\GL}{GL} % General linear group
\DeclareMathOperator{\SL}{SL} % Special linear group
\newcommand{\leftnormal}{\trianglelefteq}
\newcommand{\nleftnormal}{\ntrianglelefteq}
\newcommand{\rightnormal}{\trianglerighteq}
\newcommand{\nrightnormal}{\ntrianglerighteq}

%-----------------------------------------------------------
\begin{document}
\frontmatter
\title[notes]{Lecture notes on Algebraic Geometry}
\author[Joel Antonio-V\'asquez]{Joel Antonio-V\'asquez}
\address{Ica, Peru}
\email{hello@joelantonio.me}
\urladdr{http://joelantonio.me/}
\thanks{}%The Author thanks J. Smith
\subjclass{}%Primary 05C38, 15A15; Secondary 05A15, 15A18

\begin{abstract}
  %
\end{abstract}
\maketitle
\tableofcontents


\chapter*{Preface}

\markboth{PREFACE}{PREFACE} 
%

\mainmatter

%\part{The First Part}

\chapter{Group Theory}

\section{Structure of a Group}
  
\chapter{Field Theory}

%\part{The second part}
%
%\chapter{Algebras}
%
%\backmatter \appendix
%
%\chapter{The First Appendix}
%
%The \verb"\appendix" command should be used only once. Subsequent
%appendices can be created using the Chapter command.
%
%\chapter{The Second Appendix}
%%
%Some text for the second Appendix.

%%%%%%%%%%%%%%%%%%%%%%%%%%%%%%%%%%%%%%%%%%%%%%%%%%%%%%%%%%%%%%%%%%%%%%%%%%%%%%%%
%                                 BIBLIOGRAPHY                                 %
%%%%%%%%%%%%%%%%%%%%%%%%%%%%%%%%%%%%%%%%%%%%%%%%%%%%%%%%%%%%%%%%%%%%%%%%%%%%%%%%

% arXiv bibliography macro
\def\arXiv#1{\href{http://arxiv.org/abs/#1}{arXiv:#1}}

\begin{thebibliography}{0}
 
 \bibitem[Va]{vakil} Ravi Vakil,
   \emph{THE RISING SEA, Foundations of Algebraic Geometry. \/}
   2010-2015. \\

 \bibitem[Ha]{hartshorne} Robin Hartshorne,
   \emph{Algebraic Geometry (Graduate Texts in Mathematics). \/}
   Springer Science \& Business Media, 1977. \\

 \bibitem[ShA]{shafarevich} Igor R. Shafarevich,
   \emph{Basic Algebraic Geometry 1 (Varities in Projective Space), Third Edition. \/}
   Springer-Verlag, 1994. \\

 \bibitem[ShB]{shafarevich} Igor R. Shafarevich,
   \emph{Basic Algebraic Geometry 1 (Schemes and Complex Manifolds), Third Edition. \/}
   Springer Science \& Business Media, 1994. \\

 \bibitem[Li]{liu} Qing Liu,
   \emph{Algebraic Geometry and Arithmetic Curves (Oxford Graduate Texts in Mathematics). \/}
   Springer Science \& Business Media, 1994. \\

 \bibitem[Gr]{grothendieck} Alexander Grothendieck,
   \emph{\'El\'ements de g\'eom\'etrie alg\'ebrique: IV. \'Etude locale des sch\'emas et des morphismes de sch\'emas, Troisi\`eme partie. \/ }
   Publications Math\'ematiques De L'I.H.\'E.S., 1966. \\

 \bibitem[Ch]{chen} Evan Chen,
   \emph{Math 137 Lecture Notes. \/}
   Harvard, Spring 2015 \url{http://www.mit.edu/~evanchen/notes/Harvard-137.pdf} .\\



\end{thebibliography}

\end{document}
