\documentclass[11pt]{amsbook}%
\usepackage{amsmath}%
\usepackage{amsfonts}%
\usepackage{mathabx}%
\usepackage{mathtools}%
\usepackage{amssymb}%
\usepackage{graphicx}
\usepackage{enumerate}
\newcommand{\ii}{\item}
\newenvironment{subproof}[1][Proof]{
\begin{proof}[#1] \renewcommand{\qedsymbol}{∎}}
{\end{proof}}
\newcommand\defeq{\stackrel{\mathclap{\normalfont\mbox{\small{def}}}}{=}}

\theoremstyle{plain}
\theoremstyle{definition}
\newtheorem{definition}[theorem]{Definition}
\newtheorem{definition*}{Definition}
\newtheorem{acknowledgement}{Acknowledgement}
\newtheorem{algorithm}{Algorithm}
\newtheorem{axiom}{Axiom}
\newtheorem{case}{Case}
\newtheorem{claim}{Claim}
\newtheorem{conclusion}{Conclusion}
\newtheorem{condition}{Condition}
\newtheorem{conjecture}{Conjecture}
\newtheorem{corollary}{Corollary}
\newtheorem{criterion}{Criterion}
\newtheorem{example}[theorem]{Example}
\newtheorem*{example*}{Example}
\newtheorem{exercise}{Exercise}
\newtheorem{lemma}{Lemma}
\newtheorem{notation}{Notation}
\newtheorem{problem}{Problem}
\newtheorem{proposition}[theorem]{Proposition}
\newtheorem{proposition*}{Proposition}
\newtheorem{remark}{Remark}
\newtheorem{solution}{Solution}
\newtheorem{summary}{Summary}
\newtheorem{theorem}{Theorem}
\numberwithin{equation}{section}
\makeatletter
\renewcommand\tocsection[3]{%
  \indentlabel
  {\vspace{0.25em} \@ifnotempty{#2}{\ignorespaces #1 \S#2.\quad }}#3%
}
\makeatother
\usepackage[usenames,dvipsnames]{xcolor}
\usepackage[colorlinks=true,linkcolor=Red,citecolor=Green]{hyperref}
\usepackage{mdframed}
\makeatletter
\let\theorem\undefined
\let\c@theorem\undefined
\let\lemma\undefined
\let\c@lemma\undefined
\let\proposition\undefined
\let\c@proposition\undefined
\makeatother
\newmdtheoremenv[outerlinewidth=2,leftmargin=20,
  rightmargin=20,backgroundcolor=white,
  outerlinecolor=blue,innertopmargin=0.1\topskip,
  innerbottommargin=10,
  splittopskip=\topskip,skipbelow=\baselineskip,
skipabove=\baselineskip]
{theorem}
{Theorem}

\newmdtheoremenv[outerlinewidth=2,leftmargin=20,
  rightmargin=20,backgroundcolor=white,
  outerlinecolor=blue,innertopmargin=0.1\topskip,
  innerbottommargin=10,
  splittopskip=\topskip,skipbelow=\baselineskip,
skipabove=\baselineskip]
{lemma}
{Lemma}

\newmdtheoremenv[outerlinewidth=2,leftmargin=20,
  rightmargin=20,backgroundcolor=white,
  outerlinecolor=blue,innertopmargin=0.1\topskip,
  innerbottommargin=10,
  splittopskip=\topskip,skipbelow=\baselineskip,
skipabove=\baselineskip]
{proposition}
{Proposition}
%-----------------------------------------------------------
% Macros
\newcommand{\CC}{\mathbb C}
\newcommand{\FF}{\mathbb F}
\newcommand{\NN}{\mathbb N}
\newcommand{\QQ}{\mathbb Q}
\newcommand{\RR}{\mathbb R}
\newcommand{\HH}{\mathbb H}
\newcommand{\ZZ}{\mathbb Z}
\newcommand{\HC}{\mathcal H}
\newcommand{\OC}{\mathcal O}
\newcommand{\CK}{\mathcal C}
\newcommand{\AK}{\mathcal A}
\newcommand{\BK}{\mathcal B}
\newcommand{\DK}{\mathcal D}
\newcommand{\IF}{\mathfrak I}
\newcommand{\CL}{\mathcal L}
\newcommand{\ML}{\mathcal L}
\newcommand{\BC}{\mathcal B}
\newcommand{\SC}{\mathcal S}
\newcommand{\CR}{\mathcal R}
\newcommand{\af}{\mathfrak a}
\newcommand{\sff}{\mathfrak s}
\newcommand{\bff}{\mathfrak b}
\newcommand{\rf}{\mathfrak r}
\newcommand{\mf}{\mathfrak m}
\newcommand{\MF}{\mathfrak M}
\newcommand{\pf}{\mathfrak p}
\newcommand{\ifr}{\mathfrak i}
\newcommand{\qf}{\mathfrak q}
\newcommand{\if}{\mathfrak i}
\newcommand{\PF}{\mathfrak P}
\newcommand{\QF}{\mathfrak Q}
\newcommand{\UF}{\mathfrak U}
\newcommand{\card}{\text{card}}
\newcommand{\Gal}{\text{Gal}}
\newcommand{\Fix}{\text{Fix}}
\DeclareMathOperator{\Null}{null}
\DeclareMathOperator{\rk}{rk}
\newcommand{\charin}{\text{ char }}
\renewcommand{\proof}{ \textbf{Proof: }}
\renewcommand{\counterexample}{ \textbf{Counterexample: }}
\renewcommand{\ie}{i.e., }
\renewcommand{\eg}{e.g., }
\newcommand{\lrangle}[1]{\langle \text{#1} \rangle}
\DeclareMathOperator{\sign}{sign}
\DeclareMathOperator{\Aut}{Aut}
\newcommand{\Stab}[1]{\text{Stab}(#1)}
\DeclareMathOperator{\Inn}{Inn}
\DeclareMathOperator{\Ann}{Ann}
\DeclareMathOperator{\Irr}{Irr}
\DeclareMathOperator{\hgt}{hgt}
\DeclareMathOperator{\Syl}{Syl}
\DeclareMathOperator{\Core}{Core}
\DeclareMathOperator{\PSL}{PSL}
\DeclareMathOperator{\Hom}{Hom}
\DeclareMathOperator{\Gal}{Gal}
\DeclareMathOperator{\Rad}{Rad}
\DeclareMathOperator{\Ker}{Ker}
\DeclareMathOperator{\Syl}{Syl}
\DeclareMathOperator{\lcm}{lcm}
\DeclareMathOperator{\gcd}{gcd}
\DeclareMathOperator{\im}{Im}
\DeclareMathOperator{\GL}{GL} % General linear group
\DeclareMathOperator{\SL}{SL} % Special linear group
\newcommand{\leftnormal}{\trianglelefteq}
\newcommand{\nleftnormal}{\ntrianglelefteq}
\newcommand{\rightnormal}{\trianglerighteq}
\newcommand{\nrightnormal}{\ntrianglerighteq}

%-----------------------------------------------------------
\begin{document}
\frontmatter
\title[notes]{Lecture notes on Abstract Algebra}
\author[Joel Antonio-V\'asquez]{Joel Antonio-V\'asquez}
\address{Ica, Peru}
\email{hello@joelantonio.me}
\urladdr{http://joelantonio.me/}
\thanks{}%The Author thanks J. Smith
\subjclass{}%Primary 05C38, 15A15; Secondary 05A15, 15A18

\begin{abstract}
  The intention to this lectures notes is present a lot
  of examples in basic abstract Algebra.
\end{abstract}
\maketitle
\tableofcontents


\chapter*{Preface}

\markboth{PREFACE}{PREFACE} 
I must say that this notes are not finished at all, until now I only can say that this is
a recompilation of many definitions of the books which are in the references, the first
section has a different taste since I’ve written my own examples and proofs, the rest is incomplete,
you must to consider this. As soon as possible I will update this notes and I'll focus on
examples and different proofs, please don’t take this notes totally seriously yet, but
if you have any suggestion or you see any typo, please send me a mail, thanks.

\mainmatter

%\part{The First Part}

\chapter{Group Theory}
First, we learn one of the most important structure on Abstract Algebra,
namely \textbf{groups}.

\begin{definition*}
  \label{def-group}
  Let $G$ be a set with a function $*: G \times G \longrightarrow G$ such that
  for each $a, b \in G$ then $a * b \in G$. Further, $G$ claims the following
  axioms: 
  \begin{enumerate}[G1. ]
      \ii For all $a, b, c \in G$ then $a*(b*c) = (a*b)*c$, (\textbf{associativity}).
      \ii There exists an element $e \in G$ such that $e*a = a*e = a$, (\textbf{identity}, denoted by $1_{G}$).
      \ii There exists an element $a' \in G$ such that $a*a' = a'*a = e$,
      where $e$ is the identity element, (\textbf{inverse}, denoted by $a^{-1}$).
  \end{enumerate} 
  Then, $G$ is called a group, denoted by $(G, *)$.
\end{definition*} 

\begin{proposition}
  $1_{G} \in G$ is unique in any group.
\end{proposition} \vspace{1.8em}
\textbf{Proof: } Let $1, 1'$ be identities of $G$. Then $1 = 1*1' = 1'$. \qedsymbol

\begin{proposition}
  \label{prop-2}
  The inverse of any element $a \in G$ is unique.
\end{proposition} \vspace{1.8em}
\textbf{Proof: } Suppose that $a', a''$ are inverses of $a$. Then $a'*a = a'*a*a'' = a''$. \qedsymbol \\

\begin{example*}
  Let $\rho : y \longrightarrow \sin(y)$ be the sine curve map. Then its isometries forms a group.
  \begin{center}
    \begin{bmatrix}
      0 && v && -v \\
      -v && 0 && v \\
      v && -v && 0 \\
    \end{bmatrix}
  \end{center}
    The map $\rho$ is only linear for this case, since $\sin(v + -v) = \sin(v) + \sin(-v) = 0$.
    Indeed, the identity element is $0$ and asserts properties G1 and G2. Then, the isometries
    of the sine curve forms a group.
\end{example*}

From here, we can define other kind of groups with additional axioms. Indeed,
if we have a proper subset $G' \subset G$ such that $G'$ asserts the axioms in definition \ref{def-group}
of $G$ into $G'$ (i.e., if $a,b \in G'$ so does $ab \in G$, $1_{G} \in G'$ and for every $a \in G$ so does $a^{-1} \in G$),
then $G'$ is called a \textbf{subgroup}, denoted by $G' < G$.

\begin{example*}
  The trivial subgroups (i.e. not proper subgroups) of a group $G$ are itself and $1_{G}$.
\end{example*}

\begin{example*}
  $H_{i \in I}$ are subgroups of $G$, then so does $\bigcap_{i \in I} H_{i}$. \\
  \textbf{Proof: } If $\bigcap_{i \in I} H_{i} = 1_{G} \text{ or } G$, then is trivially true.
  Let $G' \neq 1_{G}, G$, Then $1_{G} \in G'$ since every $H_{i}$ must
  have the identity element. If $x \in G'$ so does $x^{-1} \in G'$ since
  every $H_{i}$ have the inverses of elements. $G'$ is associative for the above reasons.
  So, $G'$ is a subgroup of $G$. \qedsymbol
\end{example*}

\begin{example*}
  $H_{i \in I}$ are subgroups of $G$, then not necessarily does $\bigcup_{i \in I} H_{i}$. \\
  \textbf{Counterexample: } $\QQ[\sqrt[2]{5}] \cup \QQ[\sqrt[3]{5}] \nless \QQ[\sqrt[6]{5}]$.
\end{exammple*}

\begin{definition*}
  \label{def-abelian}
  Let $G$ be a group in the sense of definition \ref{def-group}. Further, for all $a, b \in G$
  asserts strictly that $a*b = b*a$. Then, $G$ is called an \textbf{abelian group} (\textbf{commutative}).
\end{definition*}

\begin{example*}[Dihedral group]
  The dihedral group, is the group of symmetries of a regular $n$-gon in the plane. It has the
  presentation
  $$
  \langle x, a \ \vert \ a^{n} =  x^{2} = e, xax^{-1} = a^{-1} \rangle,
  $$
  where $e$ is the identity element. \\
  The $D_{n}$s are nonabelian for $n \geq 3$, the elements $a, x$ do not commute.
  But $D_{2}$ is, its elements are $\{1, p, r, pr\}$.
\end{example*}

\begin{definition*}
  \label{def-semigroup}
  Let $S$ be a set endowed with associative binary operation $*: S \times S \longrightarrow S$. Then
  $S$ is called a \textbf{semigroup}.
\end{definition*}
We can say that a semigroup is called a \textbf{monoid} iff asserts the axiom G2. 
It's easy to see that a group is a monoid with the additional axiom G3. \\ \\
Up to this point, we've seen basic and easy definitions related to groups, the main
purpose now is to build several properties and operations around groups in the section
\ref{sect-structure-group}. Before finished this little introduction, let's
list some classic group examples.
\begin{example*} 
  $&&$ \\
  \begin{enumerate}[$\circ$ ]
      \ii A symmetric $\RR$-correlation;
      \ii A skew $\CC$-correlation;
      \ii A skew $\widetilde{\HH}$- or equivalently, $\overline{\HH}$-correlation;
      \ii Let $J = \left(\begin{matrix} 0 && I_{n} \\ -I_{n} && 0 \end{matrix} \right)$;
      \ii Let $V$ be a vector space. Then $\Aut(V) := \{T: V \longrightarrow V \vert T \text{ a linear isomorphism}\}$.
  \end{enumerate}
\end{example*}
Don't worry if you don't know at all the above examples, the idea of this is to start getting used of notation, in
later sections we'll work on them.


\section{Structure of a Group}
\label{sect-structure-group}
Now, let's see what kind of operations we can do around groups.

\subsection{Homomorphisms}

\begin{definition*}
  \label{def-homomorphism}
  A \textbf{homomorphism} of two groups $G, G'$ is a function $f: G \longrightarrow G'$ such that
  $f(ab) = f(a)f(b)$ for every $a, b \in G$.
\end{definition*}
Actually, we can map the set of homomorphisms of two groups $G, G'$ as
$$
\Hom(G, G')  = \{\text{homomorphisms } f: G \longrightarrow G' \}.
$$
\begin{lemma}
  If $\varphi: A \longrightarrow B$ and $\rho: B \longrightarrow C$ are homomorphism groups.
  Then so is $\varphi \circ \rho: A \longrightarrow C$.
\end{lemma} \vspace{1.8em}
\textbf{Proof: }Let $a, b, c \in A, B, C$, respectively. $\varphi \circ \rho = \rho(\varphi(a)) = \rho(b) = c$.
Moreover, $1_{\varphi} \circ 1_{\rho} = 1_{\varphi \circ \rho}$ and
$\varphi^{-1} \circ \rho^{-1} = \varphi(\rho(c)^{-1})^{-1} = \varphi(b)^{-1} = a$.
Indeed, homomorphisms preserve identity elements and inverses. \qedsymbol \\ \\
A group homomorphism $\varphi: (G, \circ) \longrightarrow (G', *)$ sastifies the following properties:
\begin{enumerate}[1. ]
    \ii $\varphi(1_{G}) = 1_{G'}$.
    \ii $\varphi(a^{-1}) = \varphi(a)^{-1}$ for all $a \in G$.
    \ii $\Ker \varphi \subset G$ and $\im \varphi \subset G'$ are subgroups.
\end{enumerate}
\textbf{Proof: }
\begin{enumerate}[(1). ]
    \ii $1_{G'}\varphi(1_{G}) = \varphi(1_{G}) = \varphi(1_{G}1_{G}) = \varphi(1_{G})\varphi(1_{G})$.
    Then $1_{G'} = \varphi(1_{G})$. 
    \ii Since $1_{G'} = \varphi(1_{G})$, then $1_{G'} = \varphi(aa^{-1}) = \varphi(a)\varphi(a^{-1})$.
    Finally, $\varphi(a)^{-1} = \varphi(a^{-1})$. 
    \ii Let $x, y \in \Ker \varphi$, by (1) $\varphi(1_{G}) = 1_{G'}$ and 
    $\varphi(x) = \varphi(y)$. Then,
    $$
    = \varphi(x^{-1} \circ y ) = \varphi(x^{-1}) * \varphi(y) \\

    \hspace{9.7em} = (\varphi(x))^{-1} * \varphi(y) \\

    \hspace{9.7em} = 1_{T}^{-1} * 1_{T} \\ 

    \hspace{9.7em} = 1_{T}
    $$

    So, $x^{-1} \circ y \in \Ker \varphi$. Indeed, $\Ker \varphi$ is a subgroup.
    (Same to show that $\im \varphi$ is a group).
    \qedsymbol
\end{enumerate}

\subsection{Cosets}
\begin{definition*}
  Let $H < G$ be a subgroup of $G$ of the form $gH = \{gh: h \in G\}$. Then 
  the subset $gH$ is called a \textbf{left coset} of a subgroup $H$
  (same apply to \textbf{right coset}, denoted by $Hg$).
\end{definition*} \\
A \textbf{coset} is left or right. Any element of a coset is called a
\textbf{representative} of that coset.

\begin{definition*}
  A subgroup $H < G$ is called \textbf{normal} when the left cosets and
  the right cosets of the subgroup coincide (i.e., $gH = Hg$ for all $g \in G$),
  denoted by $H \leftnormal G$.
\end{definition*}

\begin{proposition}
  All the subgroups of an abelian group are normal.
\end{proposition} \vspace{1.8em}
\textbf{Proof: }Let $H < G$ be a subgroup of any abelian group $G$. Let $h \in H$
be any element of $H$. Then for all $g \in G$ we have that $gh \in H$.
Since $G$ is commutative so is $H$, then $hg \in H$, so $gh = hg$. Indeed,
$G \leftnormal H$. \qedsymbol

\begin{example*}
  Let $1_{S_{2}}$ be the identity of any subgroup of index 2, then to
  $S_{2}$ could be a group there exists another element $v$
  such that $v \cdot  1_{S_{2}} = v$ since the group has index
  2, only left $v \cdot v = 1_{S_{2}}$. Now, $S_{2} \leq G_{n}$
  for a $n \geq 2$, since $S_{2}$ is a subgroup of $G_{n}$ then
  $1_{S_{2}} = 1_{G_{n}}$ and for any $u \in G_{n}$ we have that
  $uS = Su$. Indeed, any subgroup of index 2 is normal. \qedsymbol
\end{example*}

\begin{example*}
  $D_{n}$ has normal subgroups, which consists of its rotations $R$ and
  its reflections $F$ subject to the relations $R^{n} = F^{2} = 1_{D_{n}}$
  and $(RF)^{2} = 1_{D_{n}}$. For $n$ odd the normal subgroups are
  $\{1_{D_{n}}\}$ and $\langle R^{d} \rangle$ (i.e., cyclic group) for all divisors
  $d \vert n$. If $n$ is even, there are two more normal subgroups
  $\langle R^{2}, F \rangle$ and $\langle R^{2}, RF \rangle$.
\end{example*}

\begin{proposition}
  Let $A, B \leq G$ be subgroups of $G$. If $A \leftnormal B$ and
  $B \leftnormal G$ then not necessarilty $A \leftnormal G$.
\end{proposition} \vspace{1.8em}
\textbf{Counterexample: }Let $G = S_{4}$ (symmetry group), $A = \langle (12)(34) \rangle$
and $B = \{(12)(34), (13)(42), (23)(41), 1_{B}\}$. Clearly, $A \leftnormal B$,
$B \leftnormal S_{4}$ but not $A \leftnormal S_{4}$. \qedsymbol

\begin{proposition}
  Let $\varphi: A \longrightarrow B$ be an arbitrary homomorphism of groups, then
  $N \leftnormal A$ does not necessarily imply $\varphi(N) \leftnormal B$.
\end{proposition} \vspace{1.8em}
\textbf{Counterexample: } Let $H$ be any non-normal subgroup of $B$.
If $N = A = H$, then claim follows. \\ \\
We need to notice that having a left or right coset $H$ of a group $G$, then 
represents a \textbf{quotient} of $G$ by $H$, denoted by $G/H$.

\begin{definition*}
  \label{def-center}
  The \textbf{center} of a group $G$ is
  $$
  Z(G) = \{g \in G \ \vert \ gx = xg \text{ for all } x \in G \}.
  $$
\end{definition*}
$Z(G)$ is clearly a normal subgroup of $G$.

\begin{example*}
  Let $H \leq Z(G)$ and let $z \in Z(G)$, then $gzg^{-1} = z$ for all
  $g \in G$. Hence $gHg^{-1} = H$ for all $H$. Indeed, all
  subgroups of $Z(G)$ are normal subgroups of $G$.
\end{example*}

\subsection{Order}
\begin{definition*}
  The \textbf{order} of a group $G$ is the cardinality $|G|$, which could be
  a positive integer or $\infty$.
\end{definition*}
Obviously, the order of a element $g$ in a group $G$ is infinite
if $g^{m} \neq 1$ for all $m \neq 0$. \\

\begin{example*}
  Suppose that for any trivial group that $G = 1_{G}$, then its order is 1.
\end{example*}

\begin{theorem}[Lagrange's Theorem]
  \label{theo-lagrange}
  If $H$ is a subgroup of a finite group $G$, then $|H|$ divides $|G|$.
\end{theorem} \vspace{1.8em}
\textbf{Proof: }Let $H$ be a proper subset which is subgroup of $G$.
We know that for all $g \in G$ then $|gH| = |H|$. So
we define the \textit{left translation map} $\varphi_{g}: H \longrightarrow Hg$
which is an onto map.

\begin{lemma}
  \label{lemm-2}
  $\varphi_{g}$ is bijective.
\end{lemma} \vspace{1.8em}
\textbf{Proof: } Since $\varphi_{g}$ is onto, only left to show that is also
one-to-one map. If $gh = gh'$ where $h, h' \in G$ then implies
that $h = h'$. Indeed, $\varphi_{g}$ is bijective. \qedsymbol \\ \\
By lemma \ref{lemm-2}, we have that exists a $\HC$ left
transversal for $H$ in $G$, then
$$
|G| = \sum_{\HC} |gH| = \sum_{\HC} |H| = |\HC||H|
$$
\qedsymbol \\ \\
However, the converse of Lagrange's Theorem is false, let's see
the following example.
\begin{example*}
  Let $A_{4}$ be the \textbf{alternating group} (i.e., group of even permutations of a finite set)
  of order 12, which has no subgroup of order 6.
\end{example*}

\begin{definition*}
  The \textbf{order of an element} $g$ in a finite group $G$ is given by
  $g^{m} = 1_{G}$. By Theorem \ref{theo-lagrange}, $m$ divides
  $|G|$.
\end{definition*}

\begin{proposition}
  Any group of even order, has an element of order two.
\end{proposition} \vspace{1.8em}
\textbf{Proof: } Let $G$ be a group and suppose that $|G| = 2m$, then
one elemenet of $G$ is obviously $1_{G}$ which left us
$2m - 1$ elements. Apart of the elements paired with their
inverses, there is an element $g \in G$ such that $g = g^{-1}$.
Thus, $g$ is an element of order two. \qedsymbol

\begin{definition*}[Cyclic group]
  Let $C_{n}$ be a group which is generated by a single element $c$ such that
  $C_{n} = \langle c \rangle$ where every element of $C_{n}$ is a power
  of $c$.
\end{definition*}

\begin{proposition}
  If $m$ and $n$ are relative primes, then $C_{mn} \cong C_{m} \times C_{n}$.
\end{proposition} \vspace{1.8em}
\textbf{Proof: }Let $C_{mn} = \langle c \rangle$ be cyclic of order $mn$.
Let $M = \langle c^{m} \rangle \cong C_{m}$ and $N = \langle c^{n} \rangle \cong C_{n}$.
$c^{k} \in M$ iff $k$ divides $m$ and $c^{k} \in N$ iff $k$ divides $n$.
Now, if $c^{k} \in M \cap N$, then $k$ divides $m$ and $n$, in fact 
$k$ divides $mn$, thus $M \cap N = 1_{C_{mn}}$. Hence $MN = C_{mn}$,
since $m$ and $n$ are relative primes. \qedsymbol

\begin{definition*}[Euler's $\phi$ function]
  Euler's function $\phi(n)$ is the number of integers $1 \leq k \leq n$
  that are relatively prime to $n$. Where 
  $$
  \phi(n) = n \prod_{p \ \text{prime}, \ p \vert n} (1 - 1/p).
  $$
\end{definition*}

\begin{proposition}
  A cyclic group of order $n$, has exactly $\phi(n)$ elements order $n$.
\end{proposition} \vspace{1.8em}
\textbf{Proof: }Let $G = \langle c \rangle$ be cyclic of order $n$.
Let $1 \leq k \leq n$ such that $(c^{k})^{n} = 1_{G}$, then $k$ has order
$n$ iff $k$ and $n$ are relatively prime. If $\gcd(k, n) = 1$, then
$c^{k}$ has order $n$. If $(c^{k})^{m} = 1_{G}$, then $n$ divides
$km$ and $n$ divides $m$. \qedsymbol

\begin{definition*}[$p$-groups]
  Let $p$ be a prime, then a $p$-group is a group of order power
  of $p$.
\end{definition*}

\begin{theorem}[Fundamental Theorem of Finite Abelian Groups]
  \label{theo-finite-abelian}
  Let $G$ be a finite additive group (i.e., $+:G \times G \longrightarrow G$),
  let $p > 0$ be a prime number that divides $G$ and let $G(p)$ be the
  unique $p$-subgroup of $G$ of maximal order. Then,
  $$
  G = \bigoplus_{p | |G|} G(p),
  $$
  where $G(p) = \{x \in G \ | \ x^{p^{r}} = 1_{G}, \ \text{for some integer} \ r \} \leftnormal G \ \text{and} \ |G(p)| = p^{n}$. \\ \\
  Futhermore, if $p | \ |G|$, then
  $$
  G(p) \cong \bigtimes_{i=1}^{r} \ZZ/p^{p_{n_{i}}} \ZZ,
  $$
  where $r$ is unique and $1 \leq n_{1} \leq \dots \leq n_{r}$ also
  unique relative to this ordering.
\end{theorem} \vspace{1.8em}
\textbf{Proof: }First, let's prove that $G \cong G(p_{1}) \times \dots \times G(p_{r})$.
Clearly, $|G(p_{1}) \dots G(p_{r})| = |G(p_{1})| \dots |G(p_{r})|$ since $G(p_{1}) \dots G(p_{r})$ is a group,
if $x_{1} \dots x_{r} = 1_{G}$ with $x_{i} \in G(p_{i})$ for $i = 1, \dots, r$, then
$G(p_{1}) \times \dots \times G(p_{r}) \longrightarrow G$ is a group homomorphism since
$x_{j}^{n/p_{i}^{m_{i}}} = 1_{G}$ for all $j \neq i$. Thus, the map is clearly
isomorphism. By the above, each $G(p)$ is isomorphic to product of clyclic
$p$-groups, so now we need to show that
\begin{equation}
  \label{eq-1.1}
  \bigtimes_{i=1}^{r} \ZZ/p^{n_{i}} \ZZ \cong \bigtimes_{j=1}^{s}\ZZ/p^{m_{j}}\ZZ, \tag{*}
\end{equation}
with $n_{1} \geq \dots \geq n_{r}$ and $m_{1} \geq \dots \geq m_{s}$. Mutliplying
(\ref{eq-1.1}) by $p$ we get
$$
  \bigtimes_{i=1}^{r} \ZZ/p^{n_{i}-1} \ZZ \cong \bigtimes_{j=1}^{s}\ZZ/p^{m_{j}-1}\ZZ
$$
as $p(\ZZ/p^{k}\ZZ) \cong \ZZ/p^{k-1}\ZZ$ for all $k$. Then $|G(p)| = \prod_{i=1}^{r} p^{n_{i}} = \prod_{i=1}^{s} p^{m_{i}}$,
so we conclude that $r = s$ and $n_{i} = m_{i}$ for all $i$. \qedsymbol \\ \\
As you can see, Theorem \ref{theo-finite-abelian} generalizes to finitely generated abelian
groups. Futhermore, an abelian group $G$ is called:
\begin{enumerate}[i)]
    \ii \textit{Torsion-free}, if every non-identity element in $G$ has infinite order.
    \ii \textit{Torsion}, if every non-identity element in $G$ has finite order.
\end{enumerate}

\subsection{Isomorphism}
$&&$ \\
\subsubsection{The first isomorphism theorem}

\begin{theorem}[First Isomorphism Theorem]
  \label{theo-first-isomorphism}
  Let $\varphi: G \longrightarrow G'$ be a group homomorphism.
  Then $\Ker \varphi$ is a normal subgroup of $G$, and there
  is an isomorphism $\rho: G/\Ker \varphi \longrightarrow \im \varphi$
  given by $\rho(g\Ker \varphi)  = \varphi(g)$. In fact, $G/\Ker \varphi \cong \im \varphi$.
\end{theorem} \vspace{1.8em}
\textbf{Proof: }We need to prove that $\rho$ is surjective and injective.
Let $\psi: G \longrightarrow G/\Ker \varphi$,
Since $\rho = \varphi \ \circ \ \psi$, then $\rho$ is surjective.
Now, let $x \in \Ker \rho$ such that $x$ has the form $g\Ker \varphi$, then $\rho(x) = \varphi(g)$.
Thus, $g \in \Ker \rho$. So $\rho$ is injective and indeed is an isomorphism.

\begin{proposition}
  \label{prop-9}
  Any two cyclic groups of order $n$ are isomorphic.
\end{proposition} \vspace{1.8em}
\textbf{Proof 1: }Let $G_{1} = \lrangle{a}$ and $G_{2} = \lrangle{b}$ such
that $|a| = |b| = n$. Then there is a map $\varphi: G_{1} \longrightarrow G_{2}$
such that there is a bijection $\varphi(a^{n}) = b^{n}$ since that if $a^{n} = a^{r}$
and $b^{r} = b^{n}$ then $\varphi(a^{n}) = \varphi(a^{r}) = b^{r} = b^{n}$.
Now, let $x = a^{s}, y = a^{t} \in G_{1}$ then
$$
\varphi(xy) = \varphi(a^{s}a^{t}) 

\hspace{15em } = \varphi(a^{s+t}) 

\hspace{15em} = \varphi(b^{s+t}) 

\hspace{15em}  = \varphi(b^{s}b^{t}) 

\hspace{15em} = \varphi(a^{s})\varphi(a^{t}) 

\hspace{15em} = \varphi(x)\varphi(y)
$$ \\ \\

Thus, $\varphi$ is homomorphism. Indeed, $G_{1} \cong G_{2}$. \qedsymbol \\ \\
\textbf{Proof 2: }There is a more abstract proof about proposition \ref{prop-9}.
Let $G$ be a cyclic group with generator $g$, then there is a surjective homomorphism
$(\ZZ, +) \longrightarrow G$ sending 1 to $g$. Let $H \leq \ZZ$ be a subgroup,
by Theorem \ref{theo-first-isomorphism} $G \cong \ZZ/H$, thus we can generate the
subgroups $0\ZZ, 1\ZZ, 2\ZZ, \dots$. The quotient group $\ZZ/n\ZZ$ has order
$n$ if $n \neq 0$ and order $\infty$ if $n = 0$, the order of $G$ determines its
isomorphism class. So, if $|G| = n$ is finite, then $G \cong \ZZ/n\ZZ$ and if
$|G|$ is infinite, then $G \cong \ZZ$. \qedsymbol 

$&&$ \\
\subsubsection{The second isomorphism theorem}
\begin{theorem}[Second Isomorphism Theorem]
  \label{theo-second-isomorphism}
  Let $G$ be a group, $H < G$ be a subgroup and $N \lhd G$ be
  a normal subgroup. Then $HN \leq G$  is a subgroup, 
  $N \leftnormal HN$ a normal subgroup, $H \cap N$ is a
  normal subgroup of $H$. In fact, there is a map isomorphism
  $\varphi: H/(H \cap N) \longrightarrow HN/N$.
\end{theorem} \vspace{1.8em}
\textbf{Proof: }First, we show that $HN \leq G$. Since $N \lhd  G$,
then $NH = HN$ and $1_{G} \in HN$. Let $h \in H$ and $n \in N$,
then $hn \in HN$ and $(hn)^{-1} = n^{-1}h^{-1} \in HN = NH$.
So, $HNHN = HHNN = HN$. \\ \\
Now, as $N$ is normal in $G$,
it's certainly normal in $HN$, so we have a map
$$
\rho: H \longrightarrow HN/N,
$$
such that $\rho$ sends $h$ to $hN$. Suppose $x \in HN/N$, then
$x = hnN = hN$, thus $\rho$ is surjective. Suppose that
$h \in \Ker \rho$, then $hN = N$ the identity coset, so that $h \in N$.
Thus, $h \in H \cap N$. Then, $h \in \Ker \varphi$ and by Theorem \ref{theo-first-isomorphism} $\varphi$
is isomorphic. \qedsymbol

$&&$ \\

\subsubsection{The third isomorphism theorem}
\begin{theorem}[Third Isomorphism Theorem]
  \label{theo-thrid-isomorphism}
  Let $G$ be a group and $N, M \leftnormal G$
  be two normal subgroups such that $N \subset M$. Then
  $$
  G/M \cong (G/N)(M/N).
  $$
\end{theorem} \vspace{1.8em}
\textbf{Proof: }Let $\rho: G \longrightarrow G/M$ be a map such that
$N \subset \Ker \rho$, then there is a homomorphism $\phi: G/N \longrightarrow G/M$. \\ \\
So, $\phi$ is surjective since $\phi$ sends $gN$ to the left coset $gM$.
Now, suppose that $gN \in \Ker \phi$, then the left coset $gM$ is the identity coset,
that is, $gM = M$, so that $g \in M$. Thus, the kernel consists of those left
cosets of the form $gN$, for $g \in M$, that is, $M/N$. Indeed, by Theorem \ref{theo-first-isomorphism} then
$G/M \cong (G/N)(M/N)$. \qedsymbol


\subsection{Direct Products}
One way to construct groups from small groups is thought direct products.

\begin{definition*}
  \label{def-direct-product}
  Let $G_{1}, G_{2}$ be two groups, then its direct product is denoted by
  $G_{1} \ \oplus \ G_{2}$, which is in the end the cartesian product
  (i.e., $(x_{1}, x_{2}, \dots x_{n})(y_{1}, y_{2}, \dots, y_{n}) = (x_{1}y_{1}, x_{2}y_{2}, \dots x_{n}y_{n})$)
\end{definition*}

\begin{proposition}
  Let $G_{1}, G_{2}, \dots G_{m}$ be groups of order $n$. Then
  $$
  G = \bigoplus\limits_{i} G_{i},
  $$
  is in fact a group.
\end{proposition} \vspace{1.8em}
\textbf{Proof: }We have that $1_{G} = (1_{G_{1}} \dots 1_{G_{m}})$ is the identity
element of $G$. Since we have a cartesian product of groups, then
$G$ is associative. Now, let $g$ be any element of $G$, since
$g = (g_{1} \dots g_{m})$ then its inverse is $g^{-1} = (g_{1}^{-1} \dots g_{m}^{-1})$.
Indeed, $G$ is a group. \qedsymbol

\begin{proposition}
  \label{prop-8}
  Let $G_{1}, G_{2}$ be groups, we say that a group $G$ is isormorphic
  to the direct product $G_{1} \oplus G_{2}$ iff there exists
  normal subgroups $A, B \lhd G$ such that
  $A \cong G_{1}$, $B \cong G_{2}$, $A \cap B = 1_{G}$
  and $AB = G$.
\end{proposition} \vspace{1.8em}
\textbf{Proof: }Let $\varphi: G_{1} \oplus G_{2} \longrightarrow G_{1}$ and
$\rho: G_{1} \oplus G_{2} \longrightarrow G_{2}$ be two projective maps
then $\Ker \varphi$ and $\Ker \rho$ are two normal subgroups
from $G_{1} \oplus G_{2}$. We can see that $\Ker \varphi \cap \Ker \rho = 1_{G}$
and $\Ker \varphi \oplus \Ker \rho = G_{1} \oplus G_{2}$
since $(x_{1}, x_{2}) = (x_{1}, 1)(1, x_{2})$ for every $(x_{1}, x_{2}) \in G_{1} \oplus G_{2}$
and every $(x_{1}, 1 ) \in \Ker \rho$ and $(1, x_{2}) \in \Ker \varphi$. \\ \\
Suppose that the map $\tetha: G_{1} \oplus G_{2} \longrightarrow G$
be an isomorphism map, then $A = \tetha(\Ker \rho)$ and $B = \tetha(\Ker \varphi)$
are normal subgroups of $G$. Indeed, $A \cong \Ker \rho \cong G_{1}$,
$B \cong \Ker \varphi \cong G_{2}$, $A \cap B = 1_{G}$ and $AB = G$. \qedsymbol \\ \\
By proposition \ref{prop-8} we can define the \textbf{internal direct sum} as follows.
\begin{definition*}
  \label{def-internal-sum}
  Let $G$ be a group, which is the internal direct sum of the groups $G_{1} \oplus \dots \oplus G_{n}$
  when $G_{i} \leftnormal G$ for all $i = 1, \dots, n$, $\bigcap_{i} G_{i} = 1_{G}$ and $G_{1}  \dots G_{n} = G$.
\end{definition*}
\begin{example*}
  \label{exam-direct-sum}
  Let $p_{1}, \dots, p_{n}$ be primes numbers. An abelian group of order $p_{1}^{k_{1}}, \dots, p_{n}^{k_{n}}$
  is a direct sum of subgroups of order $p_{1}^{k_{1}}, \dots, p_{n}^{k_{n}}$.
\end{example*}
About internal direct sums, we cay say that a group $G$ is \textbf{indecomposable} when
$G \neq 1$ and $G = M \oplus N$, that implies that $M = 1$ or $N = 1$. In this way,
every finite group is a direct sum of indecomposable subgroups.

\begin{proposition}
  Every group of finite length is a direct sum of finitely indecomposable subgroups.
\end{proposition} \vspace{1.8em}
\textbf{Proof: }Let $G = A \oplus B$ and $M = C \oplus D$, then $G = A \oplus C \oplus D$
for some subgroup $A$, so $C, D \leftnormal G$. Then $C, D \subsetneq M$, so $M$
is not indecomposable, but $C$ and $D$ are direct sums of indecomposable subgroups,
the so is $M$, which is the required contradiction. \qedsymbol \\ \\
We need to notice, that a simple group and indecomposable group are not equivalence.
A simple group is indecomposable but an indecomposable group need not be simple.
\begin{example*}
  $\ZZ/p^{2}\ZZ$ is indecomposable but not simple.
\end{example*}

\begin{theorem}[Krull-Schmidt Theorem]
  \label{theo-krull-schmidt}
  Let $G$ be a group of finite length such that is a direct sum
  $$
  G = G_{1} \oplus \dots \oplus G_{n} = H_{1} \oplus \dots \oplus H_{m}
  $$
  of indecomposable subgroups. Then $m = n$ and $G_{i} \cong H_{i}$. Hence,
  $$
  G = G_{1} \oplus \dots \oplus G_{k} \oplus H_{k+1} \oplus \dots \oplus H_{n}.
  $$
\end{theorem} \vspace{1.8em}
\textbf{Proof: } REMAINDER. \\

It's interesting see how in definition \ref{def-internal-sum} is needed
to keep both subgroups as normal subgroups. \textit{\textbf{What would happen if only one of them would be normal?}} \\ \\
Well, supposing that $G_{1}$ is normal and $G_{2}$ is not, we can get the following interesting facts: \\

\begin{enumerate}[1. ]
    \ii Since $G_{1}$ is normal, then for every $g_{2} \in G_{2}$ we have $g_{2}G_{1}g_{2}^{-1} = G_{1}$,
    that means, each $g_{2}$ induces an automorphism of $G_{1}$. So, we can define a
    homomorphism map $\varphi: G_{1} \longrightarrow \Aut G_{2}$, by letting $g_{2}$
    map to the homomorphism $g_{1} \mapsto g_{2}g_{1}g_{2}^{-1}$. \\
    \ii If $G_{1}$ and $G_{2}$ are abelian groups, that not mean that $G_{1} \oplus G_{2}$
    is necessarily abelian. By example, the symmetry group $S_{3}$ is a nonabelian group
    of order 6 viewed as the permutations of $\{1, 2, 3\}$, which is the cyclic group $C_{3}$ of order 3.
    If we letting $G_{1} = C_{3}$ and $G_{2} = C_{2}$ (i.e., cyclic group of order 2), then $G = \{I, (1, 2, 3), (1, 3, 2)\}$, which
    is not an abelian group. \\
    \ii If $G_{1}$ and $G_{2}$ have the same order $k$, that not mean that $G$ should be of the same order (no as example \ref{exam-direct-sum}).
    By example, letting $G_{1} = C_{2} \times C_{2} = \{1, x\} \times \{1, x\}$ (which is of exponent 2), if we take
    $G_{2} = \{1, n\} = C_{2}$, also of exponent 2, and let the nontrivial element
    of $G_{2}$ act on $G_{1}$ by the rule $n^{-1}(a, b)n = (b, a)$. Then, $(x, 1)n$ has order 4: \\
    \begin{align*}
      \bigl((x,1)n\bigr)^2 = (x,1)n(x,1)n &= (x,1)(n^{-1}(x,1)n) = (x,1)(1,x) = (x,x)\\
      \bigl((x,1)n\bigr)^3 = (x,x)(x,1)n &= (1,x)n\\
      \bigl((x,1)n\bigr)^4 = (1,x)n(x,1)n &= (1,x)(n^{-1}(x,1)n) = (1,x)(1,x) = (1,1).
    \end{align*}
\end{enumerate}
So far, we've seen interesting facts if only one of the groups is normal. So, let's make of this
a defintion.
\begin{definition*}[Semidirect products]
  \label{def-semi-product}
  Let $G$ be a group, which is isomorphic to the semidirect product of the groups $G_{1}, G_{2}$ iff
  there exist subgroups $M$ and $N$ of $G$ such that $M \cong G_{1}$, $N \cong G_{2}$ where
  $M \lhd G$, $M \cap N = \{1_{G}\}$ and $MN = G$.
\end{definition*}
Perhaps, the notation for direct products and direct sums could be a little messy,
however, we will specify which direct operation we are working on to be more precise.
In the end, we will see what are the differents between these operations.

\subsection{Conjugacy}
\begin{definition*}
  Conjugacy, is an equivalence relation on a group $G$ such that $g, x \in G$,
  then $&&^{g}x = gxg^{-1}$, these equivalence classes are the \textbf{conjugacy classes} of $G$.
  So, any group $G$ can be partitioned into conjugacy classes,
  $$
  &&^{G}x = \{&&^{g}x: g \in G\}
  $$
  for the conjugacy class of $x$ in $G$.
\end{definition*}

\begin{definition*}
  The centralizer in $G$ of an element $x$ of a group $G$ is
  $$
  C_{G}(x) = \{g \in G \ \vert \ gx = xg\}.
  $$
\end{definition*}
We notice that the number of conjugates of an element of a group $G$ is the index
of its centralizer in $G$.

\begin{definition*}
  Let $G$ be a group and $H < G$ be a subgroup. The \textbf{normalizer}
  of $H$ in $G$ is defined by
  $$
  N_{G}(H) := \{g \in G \ \vert \ gHg^{-1} = H\}.
  $$
\end{definition*}

\begin{proposition}[The Class Equation]
  In a finite group $G$,
  $$
  |G| = \sum |C| = |Z(G)| + \sum_{|C| > 1} |C|.
  $$
\end{proposition} \vspace{1.8em}
\textbf{Proof: }As the conjugacy classes constitute a partition of $G$, then $|G| = \sum |C|$.
Iff $x \in Z(G)$, then the conjugacy class of $x$ is trivial, this is $|C| = 1$.
Hence,
$$
|G| = \sum |C| = \sum_{|C| = 1} + \sum_{|C| > 1} |C| = |Z(G)| + \sum_{|C| > 1} |C|.
$$ \qedsymbol

\section{Groups Actions}
Now that we know some basic structures about groups, let's see how a group $G$ \textbf{acts} on a set $X$
and which is the effect of this.
\begin{definition*}
  \label{def-19}
  Let $G$ be a group, let $X$ be a set and let a homormorphism $*: G \longrightarrow S_{X}$ where
  $S_{X}$ is the permutation set of $X$. We say that the pair $(X, *)$ is called a $G$-action.
\end{definition*} 
Thus, in Definition \ref{def-19} for each $g \in G$ gives a permutation $*_{g}$ of $X$, which
sends any element $x \in X$ to an element $*(g)x$. We need to remark that if
$*: G \times X \longrightarrow X$, then $(X, *)$ is called a left $G$-action and
if $*: X \longrightarrow  G \times X$, then is called a right $G$-action and
$X$ is a $G$-set. \\ \\
Now, we can set up an equivalence relation. If $X$ is a $G$-set, then we define
the \textbf{stabilizer} of $x \in X$ by
$$
\Stab{x} = G_{x} = \{g \in G \ | \ g * x = x\}.
$$
\begin{definition*}
  \label{def-orbit}
  The \textbf{orbit} of an element $x \in X$ is the subset given by
  $$
  G \cdot x = \{g \cdot x : g \in G \} \subseteq X.
  $$
\end{definition*}
\begin{itemize}
  \ii The \textbf{kernel} of a group action $\varphi: G \longrightarrow S_{X}$ is the normal subgroup of $G$
  consisting of the elements acting trivially on $X$. Then,
  $$
  \ker \varphi = \bigcap_{x \in X} G_{x} = \{g \in G : X^{g} = X\}.
  $$
  \ii A $G$-action on $X$ is \textbf{faithful} or \textbf{effective} if $\Ker \varphi$ is trivial.
  Equivalently, the action is faithful if no trivial element of $G$
  acts trivially on $X$. In other words, a group action is faithful iff
  the homomorphism to the group of transformation is an injection.
\end{itemize}
Now, it's necessary to prove that actually a $G$-action generates a bijection.
\begin{proposition}
  Let $X$ be a $G$-set, $x \in X$. Define
  $$
  f_{x}: G/G_{x} \longrightarrow G*x  \text{ by } gG_{x} \mapsto g*x.
  $$
  Then, $f_{x}$ is a well-defined bijection. In particular, if $|G:G_{x}|$ is finite,
  then
  $$
  |G*x| = |G:G_{x}| \text{ and } |G*x| \text{ divides } |G|.
  $$
\end{proposition} \vspace{1.8em}
\textbf{Proof: }Let $g, g' \in G$, then $g*x = g'*x$
\begin{enumerate}[i. ]
    \ii iff $g'^{-1}*(g*x) = x$,
    \ii iff $(g'^{-1}g)*x = x$,
    \ii iff $g'^{-1}g \in G_{x}$,
    \ii iff $gG_{x} = g'G_{x}$.
\end{enumerate}
Then, $f_{x}$ is well-defined, one-to-one and surjective. So, claim follows. \qedsymbol
\begin{example*}
  Let $X$ be a set, and define $\Aut_{\text{Set}}(X)$ to be the set of automorphisms on $X$,
  which arises a group action.
\end{example*}

\begin{example*}
  Let $G = X$ be a $G$-set by itself. The left action is given by
  $$
  *: G \times X \longrightarrow X \text{ by } g*x = gxg^{-1}
  $$
  called \textit{conjugation} by $G$. The orbit of an element $a \in X = G$ is 
  $$
  C(a) := G*a = \{xax^{-1} \ | \ x \in G\},
  $$
  called the \textit{conjugacy class} of $a$, and the \textit{stabilizer} of $a$ is
  $$
  \Stab{a} := G_{a} = \{x \in G \ | \ xax^{-1} = a\} = \{x \in G \ | \ xa = ax\}.
  $$
  This subgroup of $G$, is called the \textit{centralizer} of $a$. So, $Z_{G}(a)$
  is the set of elements of $G$ commuting with $a$. In particular, $\lrangle{a} \subset Z_{G}(a)$.
  The set of fixed points of this action is
  $$
  F_{G}(X) = \{a \in X \ | \ xax^{-1} = a \text{ for all } x \in G\} 
  $$
  $$
  \hspace{2.5em} = \{a \in G \ | \ xa = ax \text{ for all } x \in G \},
  $$

  the \textit{center} $Z(G)$ of $G$.


\begin{definition*}
  Let $n$ be a positive integer and let
  $$
  S_{n} \defeq \Aut_{\text{Set}}(\{1, 2, \dots, n\}).
  $$
  We call this the \textbf{symmetry group} of $n$ letters.
\end{definition*}

\begin{theorem}[Cayley's Theorem]
  Every group $G$ is isomorphic to a subgroup of the symmetric group $S_{G}$.
\end{theorem} \vspace{1.8em}
\proof Let $G$ be a group, and let $S_{G}$ be the permutation group of $G$. For each $g \in G$,
define $\rho_{g}: G \longrightarrow G$ by $\rho_{g}(g) = gh$. Then $\rho_{g}$ is invertible
with inverse $\rho_{g^{-1}}$, and so is the permutation of the set $G$. \\
Define $\phi: G \longrightarrow S_{G}$ by $\phi(g) = \rho_{g}$. Then $\phi$ is a
homomorphism, since
$$
(\phi(gh))(x) = \rho_{gh}(x) = ghx = \rho_{g}(hx) = (\rho_{g} \circ \rho_{g})(x) =  ((\phi(g))(\phi(h)))(x),
$$
and $\phi$ is injective, since if $\phi(g) = \phi(h)$ then $\rho_{g} = \rho_{h}$, so 
$gx = hx$ for all $x \in X$, and so $g = h$ as required. So $\phi$ is 
an embedding of $G$ into its own permutation group. If $G$ is finite of order $n$, then
simply numbering the element of $G$ gives an embedding from $G$ to $S_{n}$. \qedsymbol \\ \\
Now, let's see a generalization of Cayley's Theorem given by Prof. Terence Tao (\href{https://terrytao.wordpress.com/2014/10/06/a-trivial-generalisation-of-cayleys-theorem/}{see}), this generalization is about to have an index $n$ subgroup that is isomorphic to a fixed group $H$.

\begin{theorem}[Cayley's Theorem for $H$-sets]
  Let $H, G$ be groups such that $G$ has index $n$ and is isormorphic to $H$.
  Then, $G$ is isomorphic to a subgroup ${\tilde G}$ of the semidirect product $S_{n} \ltimes H^{n}$,
  defined explicity as the set of tuples 
  $$
  (\phi, (h_{i})_{i=1}^{n})(\rho, (k_{i})_{i=1}^{n}) := (\phi \circ \rho, (h_{\rho(i)}k_{i})_{i=1}^{n})
  $$
  and inverse
  $$
  (\phi, (h_{i})_{i=1}^{n})^{-1} := (\phi^{-1}, (h_{\phi(i)}^{-1})_{i=1}^{n}).
  $$

\end{theorem} \vspace{1.8em}
\proof Let $X, Y$ be $H$-sets and let a morphism $f: X \longrightarrow Y$ which
respects the right action of $H$, thus $f(x)h = f(xh)$ for all $x \in X$ and $h \in H$.
Suppose that $H \subset G$, then we can see $G$ as an $H$-set, in this way,
we notice that $G$ is isomorphic to the $H$-set $\{1. \dots, n\} \times H$ with the
right action of $H:(i, h)k := (i, hk)$ and identified with $S_{n} \ltimes H$, acting
on the former $H$-set by the rule
$$
(\phi, (h_{i})_{i=1}^{n})(i, h) := (\phi(i), h_{i}h).
$$
\qedsymbol

\subsection{Orbits}
Now, let's extend the idea of equivalence relation introduced in Definition \ref{def-orbit}.
\begin{lemma}
  \label{lemm-equiv-relation}
  Let $X$ be a $G$-set under $*$, then $\sim_{G}$ is an equivalence relation on $X$.
\end{lemma} \vspace{1.8em}
\proof We need to prove three points:
\begin{enumerate}[1. ] 
    \ii \textbf{Reflexitivity: }For all $x \in X$, we see that $x = 1_{G}*x$, so $x \sim_{G} x$.
    \ii \textbf{Symmetry: }If $x_{1} \sim_{G} x_{2}$, then there is a $g \in G$ such that $x_{1} = g*x_{2}$,
    indeed
    $$
    g^{-1}*x_{1} = g^{-1}*(g*x_{2}) = 1_{g}*x_{2} = x_{2}.
    $$
    Of course, $x_{1} = g*x_{2}$ iff $x_{1}*g^{-1} = x_{2}$.
    \ii \textbf{Transitivity: }Suppose that $x_{1} \sim_{G} x_{2}$ and $x_{2} \sim_{G} x_{3}$.
    Let $g, g' \in G$ such that $x_{1} = g*x_{2}$ and $x_{2} = g'*x_{3}$.
    Then, $x_{1} = g*x_{2} = g*(g'*x_{3}) = (g.g')*x_{3}$, so $x_{1} \sim_{G} x_{3}$.
\end{enumerate}
\qedsymbol \\ \\
Thanks to Lemma \ref{lemm-equiv-relation}, we can see that the equivalence class of $x$ via $\sim_{G}$
is actually the set
$$
\overline{x} = \{g*x \ | \ g \in G\}.
$$
As we've seen in Definition \ref{def-orbit}, the set $\overline{x}$ is called the orbit of $x$ under $*$,
which gives a system of representatives $\OC$ for the equivalence classes under $\sim_{G}$. 

\begin{definition*}[Mantra of $G$-actions]
  \label{def-mantra-g-actions}
  Let $X$ be a $G$-set and $\OC$ a system of representatives. Then
  $$
  X = \bigvee_{\OC} G*x \text{ and if } |X| \text{ is finite, then } X = \sum_{\OC} |G*x|.
  $$
\end{definition*}

\begin{theorem}[Orbit Descomposition Theorem]
  \label{theo-orbit}
  Let $X$ be a $G$-set. Then
  $$
  X = F_{G}(X) \vee \bigvee_{\OC^{*}} G*x.
  $$
  In particular, if $X$ is a finite set, then
  $$
  |X| = \sum\limits_{\OC} |G*x| = |F_{G}(X)| + \sum\limits_{\OC^{*}} [G:G_{x}].
  $$
\end{theorem} \vspace{1.8em}
\proof By Definition \ref{def-mantra-g-actions}, $|G*x| = [G:G_{x}]$. So, claim follows. \qedsymbol

\begin{example*}[A precise example]
  Let $G = D_{3} = \{e, r, r^{2}, f, fr, rf\}$, with $|G| = 6$ and satisfying $r^{3} = e = f^{2}$ and
  $frf^{-1} = r^{-1} = r^{2}$. We have $fr = r^{2}f$ and $rf = fr^{2}$, so
  $$
  C(e) = \{e\} \hspace{3.6em} \text{ and } \hspace{0.5em} 1|6
  $$
  $$
  C(r) = \{r, r^{2}\} \hspace{2.2em} \text{ and } \hspace{0.5em} 2|6
  $$
  $$
  C(f) = \{f, fr, rf\} \hspace{0.5em} \text{ and } \hspace{0.5em} 3|6
  $$
  and
  $$
  Z_{G}(e) = G \hspace{3.6em} \text{ and } \hspace{0.5em} |C(e)| = [G:Z_{G}(e)] = 1
  $$
  $$
  Z_{G}(r) = \{e, r, r^{2}\} \hspace{0.8em} \text{ and } \hspace{0.5em} |C(r)| = [G:Z_{G}(r)] = 2
  $$
  $$
  Z_{G}(f) = \{e, f\} \hspace{2em} \text{ and } \hspace{0.5em} |C(f)| = [G:Z_{G}(f)] = 3
  $$
  with fixed points
  $$
  Z(G) = \{e\},
  $$
  so
  $$
  |G| \hspace{0.5em} = \hspace{0.5em} |Z(G)| \hspace{0.5em} + \hspace{0.5em} |C(r)| \hspace{0.5em} + \hspace{0.5em} |C(f)|
  $$
  $$
  6 \hspace{1.5em} = \hspace{1.5em} 1 \hspace{1.5em} + \hspace{1.5em} 2 \hspace{1.5em} + \hspace{1.5em} 3.
  $$
\end{example*}

\begin{theorem}[Cauchy's Theorem]
  \label{theo-cauchy}
  Let $p$ be a prime number which divides the order of the finite group $G$. Then there exists
  an element of $G$ of order $p$.
\end{theorem} \vspace{1.8em}
\proof Let $G$ be a finite group and $p$ a prime number which divides the order of $G$. Set
$$
X = \{(g_{1}, \dots, g_{p}) \in G^{p} \ | \ g_{1} \dots g_{p} = e \text{ in } G\} \subset G^{p},
$$
where $G^{p} = G \times \dots \times G$ ($p$ times). Let $(\ZZ/p\ZZ, +)$ act on $X$.
Applying Orbit Descomposition Theorem \ref{theo-orbit}, we get
$$
|G^{p}| = |X| = |F_{\ZZ/p\ZZ}(X)| + \sum_{\OC^{*}} \ [\ZZ/p\ZZ:(\ZZ/p\ZZ)_{x}].
$$
If $x \in \OC^{*}$, then $(\ZZ/p\ZZ)_{x} = 1$ as $\ZZ/p\ZZ$ has only the trivial subgroups.
So,
$$
0 \equiv |X| \equiv |F_{\ZZ/p\ZZ}(X)| \mod p.
$$
Clearly, $e \neq g_{1} = \dots = g_{p}$. Then, $g^{p} = e$ as needed. \qedsymbol

\begin{proposition}
  The order of the orbit of an element is equation to the index of its
  stabilizer.
\end{proposition} \vspace{1.8em}
\proof Let $G$ act on $X$. Let $x \in X$, there is a surjection map
$*:G \longrightarrow G \times X$ onto the orbit of $x$ induces a injection
correspondence between the elements of the orbit $x$ and the classes
of the equivalence relation induced on $G$ by $*$. The latter
are the left cosets of $\Stab{x}$, since $g*x = h*x$ is equivalent
to $x = g^{-1}h*xx$ and to $g^{-1}h \in \Stab{x}$. Hence the order of the
orbit of $x$ equals the number of left cosets of $\Stab{x}$. \qedsymbol

\section{Simple Groups}
Now that we know some facts about Definition \ref{def-center}, we can establish a relation
between the center and simple groups. First, let's see what a simple group is.
\begin{definition*}
  Let $G$ be a group. We say that $G$ is simple if $G$ has no normal subgroups other than trivial ones (i.e., $1_{G}$ and $G$).
\end{definition*}

\begin{proposition}
  \label{prop-simple}
  Every homomorphism from a simple group to another group is either injective or trivial.
\end{proposition} \vspace{1.8em}
\proof Let $f: G \longrightarrow G'$ a nontrivial homomorphism from a simple group $G$ into another group $G'$,
since $\Ker f$ is a normal subgroup of $G$, then $f$ is automatically injective and claim follows. \qedsymbol \\ \\
By Proposition \ref{prop-simple}, if $G$ is nonabelian simple, then the center $Z(G) = \{1_{G}\}$. By Lagrange's Theorem
\ref{theo-lagrange}, any group of order prime is simple.

\begin{example*}
  $A_{5}$ is the unique simple non-abelian group of smallest order.
\end{example*}

\begin{example*}
  The $\PSL_{n}(K)$ for $n \geq 3$ and $k$ has at least
  four elements, is simple.
\end{example*} 

\begin{definition*}
  Let $G$ be a group and $H$ a subgroup of $G$. We define the \textbf{core}
  of $H$ in G by $\Core_{G}(H) := \bigcap_{G} xHx^{-1}$.
\end{definition*}

\begin{lemma}
  \label{lemm-core}
  Let $H$ be a group, and suppose that $H \substeq G$ is a subgroup with
  $|G:H| = n$. Then $H$ contains a normal subgroup $N$ of $G$ such that
  $|G:N|$ divides $n!$.
\end{lemma} \vspace{1.8em}
\proof Letting $N = \Core_{G}(H)$. Then $G/N$ is isomorphic to a subgroup of the symmetric group $S_{n}$,
and by Lagrange's Thereom \ref{theo-lagrange}, $|G/N|$ divides $|S_{n}| = n!$. \qedsymbol

\begin{lemma}
  Let $G$ be a simple group such $G$ contains a subgroup of index $n > 1$.
  Then $|G|$ divides $n!$.
\end{lemma} \vspace{1.8em}
\proof Taking $N$ from Lemma \ref{lemm-core}, then it's proper in $G$ because $n > 1$.
Since $G$ is simple, $N = 1$ and thus $|G| = |G/N|$ divides $n!$. \qedsymbol

\section{Sylow's theorem}
\label{section-sylow-theorem}
Let $p > 1$ and $m > 1$ be relatively primes. Let $G$ be a group of order
$p^{r}m$ where $r > 0$. We define the Sylow $p$-subgroup $H$ of $G$ by
$$
\Syl_{p}(G) := \{H \ | \ H \text{ a Sylow $p$-subgroup of } G\}.
$$

\subsection{First Sylow Theorem}

\begin{theorem}[First Sylow Theorem]
  \label{theo-first-sylow}
  Let $p$ be a prime number. Let $k > 0$ such that $p^{k}$ divides
  the order a group $G$, then $G$ has a subgroup of order $p^{k}$.
\end{theorem} \vspace{1.8em}
\proof By induction and assuming that $k \geq 2$, let $T$ be a finite group
such that $|T| < |G|$ and $p^{k}||T|$, then $T$ contains a subgroup of 
order $p^{k}$. Now, we need to show that $G$ actually contains a subgroup
of order $p^{k}$. \\ \\
Let $H$ be a proper subset of $G$ such that $H = \lrangle{a}$ where
$a \in G$. If $a \in Z(G)$, we see that $xa^{i}x^{-1} = a^{i}$ for all $x \in G$ and $i \in \ZZ$.
Then $p \ | \ [G:Z_{G}(a)]$. In particular,  $H \leftnormal G$ and $G/H$
is a group of order $[H:H] = |G|/|H| = |G|/p < |G|$. Since $p^{k-1}| |G/H|$,
then there exists a subgroup $T \subset G/H$ of order $p^{k-1}$.
Let $*: G \longrightarrow G/H$ be the canonical epimorphism. Then $H$ is the 
kernel and there exists a subgroup $\tilde{T}$ of $G$ containing $G$ and
satisfying $T = \tilde{T}/H$. Hence $|\tilde{T}| = |T||H| = p^{m}$, and claim follows. \qedsymbol \\ \\
First Sylow Theorem can be viewed as a generalization of Cauchy's Theorem \ref{theo-cauchy}.

\begin{lemma}
  Let $P$ be a Sylow $p$-subgroup of a group $G$. Then every $p$-subgroup of the normalizer $N_{G}(P)$
  is contained in $P$.
\end{lemma} \vspace{1.8em}
\proof Let $H \leq N_{G}(P)$ be a $p$-subgroup. Since $H$ normalizes $P$, the product $HP$ is
a subgroup of $N_{G}(P)$ and $P \leftnormal HP$. Since
$$
HP/P \cong H/H \cap P
$$
is a quotient of the $p$-group $H$, and $P$ is a $p$-group, it follows that $HP$ is a $p$-group.
And $P \leq HP$. But $P$ is maximal $p$-subgroup of $G$, by definition of Sylow
$p$-subgroup. Hence $HP = P$, meaning that $H \leq P$. \qedsymbol

\subsection{Second and Third Sylow Theorems}
\begin{theorem}[Second Sylow Theorem]
  \label{theo-second-sylow}
  Let $p$ be a prime number. The number of Sylow $p$-subgroups of a finite group $G$
  divides the order of $G$ and is congruent to 1 module $p$.
\end{theorem} 

\begin{theorem}[Third Sylow Theorem]
  \label{theo-third-sylow}
  Let $p$ be a primer number. All Sylow $p$-subgroups of a finite group are conjugate.
\end{theorem} \vspace{1.8em}
\proof As Sylow did it, let's prove Theorem \ref{theo-second-sylow} and \ref{theo-third-sylow} together.
Let $S$ be a Sylow $p$-subgroup. A conjugate of a Sylow $p$-subgroup is a Sylow $p$-subgroup, therefore
$S$ acts on the set $\mathfrak{S}$ of all Sylow $p$-subgroups by inner automorphisms. Under this action,
$\{S\}$ is an orbit, since $aSa^{-1} = S$ for all $a \in S$. Conservely, if $\{T\}$ is a trivial orbit, then
$aTa^{-1} = T$ for all $a \in S$ and $s \subseteq N_{G}(T)$, then $T \leftnormal N_{G}(T)$ and yields $S = T$.
Thus $\{S\}$ is the only trivial orbit. The orders of the other orbits are indexes in $S$ of stabilizers and
are multiples of $p$. Hence $|\mathfrak{S}| \equiv 1 \mod p$. \\ \\
Suppose that $\mathfrak{S}$ contains two distinct conjugacy $\mathcal{C}$ and $\mathcal{C}'$ of subgroups.
Any $S \in \mathcal{C}$ acts on $\mathcal{C}$ and $\mathcal{C}' \subseteq \mathfrak{S}$ by inner automorphisms.
Then the trivial orbit $\{S\}$ is in $\CK$ , by the above, $|\CK| \equiv 1$ and $|\CK'| \equiv 0 \mod p$.
But any $T \in \CK'$ also acts on $\CK \cup \CK'$ by inner automorphisms, then the trivial orbit $\{T\}$
is in $\CK'$, so that $|\CK'| \equiv 1$ and $|\CK| \equiv 0 \mod p$. This shows that $\mathfrak{S}$
cannot contain two distinct conjugacy classes of subgroups. Therefore $\mathfrak{S}$ is a conjugacy class
and $|\mathfrak{G}|$ divides $|G|$. \qedsymbol

\begin{proposition}
  \label{prop-p-contained}
  In a finite group, every $p$-subgroup is contained in a Sylow $p$-subgroup.
\end{proposition} \vspace{1.8em}
\proof As we seen, a $p$-subgroup $H$ of a finite group $G$ acts by inner automorphisms on the set
$\mathfrak{G}$ of all Sylow $p$-subgroups. Since $|\mathfrak{G}| \equiv 1 \mod p$ there is a least
one trivial orbit $\{S\}$. Then $hSh^{-1} = S$ for all $h \in H$ and $H \subseteq N_{G}(S)$. Now,
$S$ is a Sylow $p$-subgroup of $N_{G}(S)$, and $H \subseteq S$, so $S \leftnormal N_{G}(S)$. \qedsymbol

\section{Series}
\begin{definition*}
  Let $G$ be a non-trivial group. A sequence of groups
  $$
  N_{0} \subset N_{1} \subset \dots \subset N_{n} = G,
  $$
  is called \textit{subnormal} if $N_{i} \leftnormal N_{i+1}$ and
  called \textit{normal} if $N_{i} \leftnormal G$.
  The quotients $N_{n}/N_{n-1}, \dots, N_{1}/N_{0}$ are called
  the \textit{factors} of the series.
  Letting $N_{0} = 1_{G}$ and $N_{n} = G$, then it's called
  \begin{enumerate}[I. ]
      \ii \textbf{Cyclic series}, if $N_{i+1}/N_{i}$ is cyclic for all $i$.
      \ii \textbf{Abelian series}, if $N_{i}/N_{i}$ is abelian for all $i$.
      \ii \textbf{Composition series}, if $N_{i+1}/N_{i}$ is simple for all $i$.
  \end{enumerate}
\end{definition*} \\
Two normal series are equal if both of them have the same length, the same
factors and up to isomorphism and indexing.

\begin{example*}
  Let $C = \lrangle{c}$. Then, the series
  $1 \leftnormal \{1, c^{3}\} \leftnormal C$ and $1 \leftnormal \{1, c^{2}, c^{4} \} \leftnormal C$
  are equal.
\end{example*}

\subsection{Normal series}
\begin{definition*}
  A \textit{refinement} of a normal series $\mathcal{A}: \  1_{G} = A_{0}, \leftnormal \dots \leftnormal A_{m} = G$
  is a normal series $\mathcal{B}: \ 1_{G} = B_{0} \leftnormal \dots \leftnormal B_{n} = G$
  such that every $A_{i}$ is one of the $B_{j}$'s.

\end{definition*}

\begin{theorem}[Schreier Refinement Theorem]
  \label{theo-refinement}
  Any two series of a group have equivalent refinements.
\end{theorem} \vspace{1.8em}
\proof Let $\mathcal{A}: \  1_{G} = A_{0}, \leftnormal \dots \leftnormal A_{m} = G$
  is a normal series $\mathcal{B}: \ 1_{G} = B_{0} \leftnormal \dots \leftnormal B_{n} = G$
  two normal series of a group $G$. Let $C_{k}, D_{k}$ where $0 \leq k \leq mn$,
  then we can write $k = ni + j$ for $0 \leq i < m$ and $0 \leq j < n$ as
  we can write $k = mj' + i'$ for $0 \leq i' < m$ and $0 \leq j' < n$.
  Then, there is a permutation $\phi: ni+j \longrightarrow mj+i$ of $\{0, 1, \dots, mn-1\}$. \\ \\
  By the above, we can see that $A_{i} = C_{ni} \subseteq \dots \subseteq C_{ni+n} = A_{i+1}$
  and $B_{j} = D_{mj} \subseteq \dots \subseteq D_{mj+m} = B_{j+1}$, for all $0 \leq i < m$ and
  $0 \leq j < n$. Letting $A = A_{i}, A' = A_{i+1}$ and $B = B_{j}, B' = B_{j+1}$, then
  $C_{ni+j} = A(A' \cap B)$, $C_{ni+j+1} = A(A' \cap B')$, $D_{mj+i} = B(B' \cap A)$, $D_{mj+i+1} = B(B' \cap A')$
  are subgroups of $G$, $C_{ni+j} \leftnormal C_{ni+j+1}$ and $D_{mj+i} \leftnormal D_{mj+i+1}$.
  So, they are normal series and refinements of $\mathcal{A}$ and $\mathcal{B}$. \qedsymbol

\subsection{Composition series}
\begin{proposition}
  \label{prop-finite-group}
  Every finite group has a composition series.
\end{proposition} \vspace{1.8em}
\proof Let $\mathcal{A}: \ 1_{G} = A_{0} \nleftnormal \dots \nleftnormal A_{m} = G$ has length
$m \leq n$. Hence $G$ has a strictly ascending normal series of maximal length. \qedsymbol

\begin{theorem}[Jordan-H{\"o}lder Theorem]
  Any two composition series of a group are equivalent.
\end{theorem} \vspace{1.8em}
\proof Let $\AK$ and $\BK$ be two compositions series from $G$. By Theorem \ref{theo-refinement},
$\AK$ and $\BK$ have equivalent refinements $\CK$ and $\DK$, respectively. Let $\phi$ be
a permutation such that $C_{k}/C_{k-1} \cong D_{\phi(k)}/D_{\phi(k-1)}$ for all $k > 0$ sends
the nontrivial factors of $\CK$ onto the nontrivial factors of $\DK$, and sends the
factors of $\AK$ onto the factors of $\BK$, therefore $\AK$ and $\BK$ are equivalent. \qedsymbol

\begin{example*}
  Let $m = a_{1} \dots a_{n} \in \ZZ$ where $a_{i} > 1$. Then we have a subnormal series
  $$
  0 = a_{1} \dots a_{n} \ZZ/m\ZZ < \dots < a_{1}\ZZ/m\ZZ < \ZZ/m\ZZ.
  $$
  As $\ZZ/m\ZZ$ is finite group, by Proposition \ref{prop-finite-group} it has a composition series.
\end{example*}

\section{Solvable Groups}
\begin{definition*}
  A solvable group is a group with a normal series whose factors are abelian.
\end{definition*}

\begin{example*}
  Every abelian group is solvable.
\end{example*}
\proof Since any abelian group has a normal series $\AK$, then for any two $A_{i+1}/A_{i}$
where $A_{i+1}, A_{i} \in \AK$ is actually a normal subgroup, then claim follows. \qedsymbol

\begin{example*}
  The dihedral group $D_{n}$ is solvable for every $n \geq 2$ because it has a chain of subgroups $1 < C_{n} < G$
  and $G/C_{n} = C_{2}$.
\end{example*}

\begin{definition*}
  The \textit{commutator} of two elements $x, y$ is $xyx^{-1}y^{-1}$, the commutator subgroup of a group $G$
  is the subgroup $G'$ of $G$ generated by all commutators.
\end{definition*}

\begin{proposition}
  \label{prop-smallest-normal}
  $G'$ is the smallest normal subgroup $N$ of $G$ such that $G/N$ is abelian.
\end{proposition} \vspace{1.8em}
\proof Since the inverse of a commutator is in fact a commutator, let $x \in G'$
b a product of commutators $x = c_{1} \dots c_{n}$, then
$$
axa^{-1} = ac_{1}a^{-1} \dots ac_{n}a^{-1} \in G' \text{ for all } a \in G'.
$$
Thus, $G' \leftnormal G$. If $N \leftnormal G$ and $G/N$ is abelian, then $Nxy = Nyx$ and
$xyx^{-1}y^{-1} \in N$ for all $x, y \in G$, and $G' \subseteq N$. \qedsymbol

\begin{theorem}
  Let $G$ be a group. Then $G$ is solvable iff there exist an integer $n$ such that $G^{(n)} = 1_{G}$.
\end{theorem} \vspace{1.8em}
\proof It's known that $G^{(i)}/G^{(i+1)}$ is abelian, hence if $G^{(n)} = 1_{G}$ for some
$n$, then $G$ is solvable. Conversely, suppose that
$$
1_{G} = N_{n} \subset N_{n-1} \subset \dots \subset N_{1} \subset G
$$
is an abelian series. By Proposition \ref{prop-smallest-normal}, $G' \subset N_{1}$. By induction,
we suppose that $G^{(i-1)} \subset N_{n-1}$. Finally, $G^{(i)} = [G^{(i-1)}, G^{(i-1)}] \subset [N_{i-1}, N_{i-1}] \subset N_{i}$.
\qedsymbol

\begin{proposition}
  \label{prop-p-solvable}
  Every finite $p$-group is solvable.
\end{proposition} \vspace{1.8em}
\proof As $G$ has a subgroup $N$ of order $p-1$, which is normal, then $N$ and $G/N$ are solvable.
By induction, then $G$ is in fact solvable. \qedsymbol \\ \\
We can generalize Proposition \ref{prop-p-solvable} with the following proposition.
\begin{proposition}
  \label{prop-pq-solvable}
  Every group of order $p^{m}q^{n}$, where $p$ and $q$ are prime numbers, is solvable.
\end{proposition} \vspace{1.8em}
\proof Since there are subgroups of order $p^{m}$ and $q^{n}$, then by Proposition \ref{prop-p-solvable}
are solvable. Then claim follows. \qedsymbol

\begin{definition*}
  A normal series $1_{G} = C_{0} \leftnormal \dots \leftnormal C_{m} = G$ is \textbf{central normal series} when
  $C_{i} \leftnormal G$ and $C_{i+1}/C_{i} \subseteq Z(G/C_{i})$ for all $0 \leq i < m$.
\end{definition*}

\begin{definition*}
  A group is \textbf{nilpotent} when it has a central normal series.
\end{definition*}
We can say that nilpotent groups are a kind of solvable groups with futher properties.

\begin{proposition}
  Every finite $p$-group is nilpotent.
\end{proposition} \vspace{1.8em}
\proof The trivial case is when $p \leq 2$, then claim follows. Suppose that $n > 2$, then
$G$ has a nontrivial center, then $G/Z(G)$ is nilpotent and so does $G$. \qedsymbol



\section{The Hall Theorems}
The Hall Theorems can be viewed as stronger versions of Sylow's Theorem seen in Section \ref{section-sylow-theorem}.
Futhermore, these theroems hold in Solvable groups.

\begin{theorem}
  \label{theo-first-hall}
  Let $m$ and $n$ be relative prime numbers. Every solvable of order $mn$ contains
  a subgroup of order $m$.
\end{theorem} \vspace{1.8em}
\proof The trivial case is when $m$ is a power of any prime number, then claims follows by First Sylow Theorem \ref{theo-first-sylow}.
Otherwise, let $G$ be a group and suppose that it has a normal subgroup $N$ of order $p^{k} > 1$ for some prime $p$ and $k \in \ZZ$.
Then, $p^{k}$ divides $m$ or $n$. \\ \\
If $p^{k}$ divides $m$, then $|G/N| = (m/p^{k})n$, where $m/p^{k}$ and $n$ are relatively prime and $|G/N| < |G|$. By induction,
$G/N$ has a subgroup $H/N$ of order $m/p^{k}$, where $N \subset H \leq G$, then $|H| = m$. \\ \\
If $P^{k}$ divides $n$, then $|G/N| = (n/p^{k})m$, where $n/p^{k}$ and $m$ are relatively prime and $|G/N| < |G|$. By induction,
$G/N$ has a subgroup $H/N$ of order $m$ when $N \subseteq H \leq G$. Then $|H| = mp^{k}$. Now, $N \leftnormal H$,
$N$ is abelian, and $N$ has order $p^{k}$, which is relatively prime to $m$. Then $H$ has a subgroup of order $m$,
and then so does $G$. \qedsymbol

\begin{theorem}
  \label{theo-second-hall}
  In a solvable group of order $mn$, where $m$ and $n$ are relatively prime, all subgroups of order $m$ are conjugate.
\end{theorem} \vspace{1.8em}
\proof Let $G$ be group of order $mn$, if $m$ is a power of any prime number, then claim follows by Third Sylow Theorem \ref{theo-third-sylow}.
Otherwise, let $G$ be a group and suppose that it has a normal subgroup $N$ of order $p^{k} > 1$ for some prime $p$ and $k \in \ZZ$.
Let $A, B \leq G$ have order $m$. \\ \\
Suppose that $p^{k}$ divides $m$. By induction, $A/N$ and $B/N$ are conjugate in $G/N$, $G/N = (Nx)(A/N)(Nx)^{-1}$ for some $x \in G$. Then
$$
B = \bigcup_{b \in B} Nb = \bigcup_{a \in A}(Nx)(Na)(Nx)^{-1}
$$
$$
\hspace{3.3em} = \bigcup_{a \in A}Nxax^{-1} = N(xAx^{-1}) = xAx^{-1},
$$
since $N = xNx^{-1} \subseteq xA^{-1}$. Thus $A$ and $B$ are conjugate in $G$. \\ \\
Now, suppose that $p^{k}$ divides $n$. Then $A \cap N = B \cap N = 1$, hence $|NA| = |NB| = p^{k}m$,
and the subgroups $NA/N \cong A/(A \cap N)$ and $NB/N \cong B/(b \cap N)$ of $G/N$ have order $m$.
By induction, $NA/N$ and $NB/N$ are conjugate in $G/N$. Therefore, $A$ and $B$ are conjugate in $G$. \qedsymbol

\begin{theorem}
  \label{theo-third-hall}
  In a solvable group of order $mn$, where $m$ and $n$ are relatively prime, every subgroup whose order
  divides $m$ is contained in a subgroup of order $m$.
\end{theorem} \vspace{1.8em}
\proof Let $G$ be group of order $mn$, if $m$ is a power of any prime number, then claim follows by Proposition \ref{prop-p-contained}.
Otherwise, let $G$ be a group and suppose that it has a normal subgroup $N$ of order $p^{k} > 1$ for some prime $p$ and $k \in \ZZ$.
Let $H$ be a subgroup of $G$ whose order $l$ divides $m$. \\ \\
Suppose that $p^{k}$ divides $m$. Then $|NH/N| = |H|/|H \cap N|$ divides $m$, is relatively prime to $n$, and divides
$|G/N| = (m/p^{k})n$. By induction, $H/N$ is contained in a subgroup $K/N$ of $G/N$ of order $m$, where $N \subseteq K \leq G$,
then $|K| = p^{k}m$ and $H \subseteq HN \subseteq K$. If $p^{k} < n$, then $|K| < |G|$ and $H$ is contained
in a subgroup of $K$ of order $m$, by the induction. \\ \\
Now, suppose that $p^{k} = n$. Let $A$ be a subgroup of $G$ of order $m$. Then $A \cap N = 1, |NA| = |N||A| = |G|$ and
$NA = G$. Hence $|A \cap NH| = |A||NH|/|ANH| = mp^{k}l/mn = l$. Thus, $H$ and $K = A \cap NH$ are subgroups
of $NH$ of order $l$. By Theorem \ref{theo-second-hall}, $H$ and $K$ are conjugate in $NH:H = xKx^{-1}$ for
some $x \in NH$. Then $H$ is contained in the subgroup $xAx^{-1}$ of $G$, which has order $m$. \qedsymbol

\chapter{Ring Theory}
Another structure and one of the most used and important in Abstract Algebra, are rings.
\begin{definition*}
  \label{def-ring}
  Let $R$ be a a set. A ring is an ordere triple $(R, +, \cdot)$ where $+$ and $\cdot$ are binary
  operations, such that:
  \begin{enumerate}[R1. ]
      \ii $(R, +)$ is an abelian group.
      \ii $(R, \cdot)$ is a semigroup. Futhermore, the multiplication is associative.
      \ii The multiplication is distributive: $x(y + z) = xy + yz$ and $(y + z)x = yx + zx$
      for all $x, y, z \in R$.
  \end{enumerate}
\end{definition*}
From the above definition, we can remark that if $(R, \cdot)$ has an identity element, then we say that $(R, +, \cdot)$
is a ring with identity (i.e., $1_{R}$). Meanwhile, the identity from $(R, +)$ is denoted by $0_{R}$.

\begin{definition*}
  \label{def-commutative-ring}
  A ring $R$ is called \textbt{commutative ring} if for all $a, b \in R$, then
  $$
  ab = ba.
  $$
  Futhermore, a commutative ring $R$ is called integral domain if $0_{R} \neq 1_{R}$ and if $ab = 0$, then
  either $a = 0$ or $b = 0$.
\end{definition*}

\begin{definition*}
  A \textbf{unit} in a ring is an invertible element.
\end{definition*}

\begin{definition*}
  \label{def-ring-field}
  A commutative division ring is called a \textbf{field}.
\end{definition*}

\begin{example*}
  $&&$
  \begin{itemize} 
      \ii $\ZZ$ is a domain.
      \ii $\QQ, \RR, \CC$ are fields.
      \ii \ZZ/n\ZZ, is a commutative ring for $n \in \ZZ^{+}$ and it's a domain iff $n$ is a prime number.
  \end{itemize}
\end{example*}

\end{example*}

\begin{definition*}
  \label{def-ring-homomorphism}
  Let $R, S$ be rings. A map $\varphi: R \longrightarrow S$ is a ring homomorphism if
  for all $a, b \in R$, sastifies that:
  \begin{enumerate}[1. ]
      \ii $\varphi(0_{R}) = 0_{S}$.
      \ii $\varphi(a + b) = \varphi(a) + \varphi(b)$.
      \ii $\varphi(1_{R}) = 1_{S}$.
      \ii $\varphi(a \cdot b) = \varphi(a) \cdot \varphi(b)$.
  \end{enumerate}
\end{definition*}
By Definition \ref{def-ring-homomorphism}, we can say that the map $\varphi: R \longrightarrow S$ is called
\begin{enumerate}[I. ]
    \ii \textit{Ring monomorphism} if $\varphi$ is injective.
    \ii \textit{Ring epiomorphism} if $\varphi$ is surjective.
    \ii \textit{Ring isomorphism} if $\varphi$ is bijective with inverse ring homomorphism.
    \ii \textit{Ring automorphism} if $R = S$ and $\varphi$ is a ring isomorphism.
\end{enumerate}

\begin{example*}[Polynomial rings]
  For any ring $R$, we have the ring
  $$
  R[x] = \{a_{0} + a_{x}x + \dots + a_{n}x^{n} \ | \ a_{1}, \dots, a_{n} \in R \},
  $$
  called the \textit{ring of polynomials with coefficients in } $R$, where $x$ is the indeterminate.
\end{example*}

\begin{definition*}
  Let $R$ be a ring. A subring is a subset $S \subset R$ which sastifies the conditions of Definition \ref{def-ring}.
\end{definition*}

\begin{example*}
  $\ZZ \subset \QQ$ and $\ZZ, \RR, \QQ \subset \CC$ are subrings.
\end{example*}

\begin{definition*}[Binomial Theorem]
  \label{def-binomial-theorem}
  In a commutative ring $R$,
  $$
  (x + y)^{n} = \sum_{0 \leq i \leq n} \binom{n}{i} x^{i}y^{n-i}, \text{ where }  \binom{x}{y} \frac{n!}{i!(n-i)!}.
  $$
\end{definition*}
The Definition \ref{def-binomial-theorem} works perfectly for any ring, as long as $xy = yx$.

\section{Ideals}
Ideals can be viewed as the analogue in ring theory of a normal subgroup in group theory.
\begin{definition*}
  Let $R$ be a ring. An ideal is a proper subset $I \subset (R, +)$ such that $x \in I$, then $xy \in I$
  and $yx \in I$ for all $y \in R$.
\end{definition*}

\begin{example*}
  Let $R, S$ be rings. A map $\varphi: R \longrightarrow S$ is a ring homomorphism, then the
  $\Ker \varphi$ is an ideal of $R$.
\end{example*}
\proof Clearly, $\Ker \varphi \subset R$. Let $x \in \Ker \varphi$ then $\varphi(x) = 0_{S}$.
Now, for all $y \in R$
$$
\varphi(y \cdot_{R} x) = \varphi(y) \cdot_{S}  \varphi(x)
$$ 
$$
\hspace{3.2em} = \varphi(x) \cdot_{S} 0_{S}
$$
$$
= 0_{S}
$$
Same for $\varphi(x \cdot_{R} y)$, then claim follows. \qedsymbol

\begin{proposition}
  \label{prop-ideal-generated}
  Let $R$ be a ring. Let $(I)$ be an ideal of $R$ generated by the subset $I$.
  If $R$ is commutative, then $(I)$ is of all linar combinations of elements of $I$
  with coefficients in $R$.
\end{proposition} \vspace{1.8em}
\proof If $I \in (I)$, then $xiy$ must in $(I)$ for any $i \in I$ and for all $x, y \in R$.
If $R$ is commutative, then $xiy = (xy)i$ and $(I)$ is the set of all finite sums $x_{1}i_{1} + \dots + x_{n}i_{n}$
with $n \geq 0, x_{1}, \dots, x_{n} \in R$ and $i_{1}, \dots, i_{n} \in I$. \qedsymbol

\begin{definition*}
  \label{def-principal-ideal}
  Let $a \in R$. Define
  $$
  (a) = RaR := \{\sum_{i=1}^{n} x_{i}ay_{i} \ | \ n \in \ZZ^{+}, x_{i}, y_{i} \in R, 1 \leq i \leq n\},
  $$
  an ideal of $R$ called \textbf{principal ideal} generated by $a$. It's the smallest ideal
  in $R$ containing $a$.
\end{definition*}

\begin{proposition*}
  Let $R$ be a ring. The principal ideal generated by $a \in R$ is the set $aR$ of all multiplies of $a$.
\end{proposition*} \vspace{1.8em}
\proof By Proposition \ref{prop-ideal-generated}, a linear combination $x_{1}a_{1} + \dots + x_{n}a_{n}$
of copies of $a$ is a multiple $(x_{1} + \dots + x_{n})a$ of $a$. Then, claim follows. \qedsymbol

\begin{lemma}
  The intersection of two ideals is an ideal.
\end{lemma} \vspace{1.8em}
\counterexample Let $(2)$ and $(3)$ be two ideals which are multiples of 2 and 3, respectively.
So, $(2) \cap (3) = (6)$ but 6 is not a prime number as well as $(6)$ is not a prime ideal. \qedsymbol

\begin{definition*}
  \label{def-maximal-ideal}
  A \textbf{maximal ideal} of a ring $R$ is an ideal $M \neq R$ of $R$ such that there is no
  ideal $M \subsetneq I \subsetneq R$.
\end{definition*}

\begin{proposition}
  Let $R$ be a ring with identity. Every proper ideal is contained in a maximal ideal.
\end{proposition} \vspace{1.8em}
\proof Since an ideal is proper iff not contains the identity element, then having an ideal
$I \neq R$, we can apply Zorn's lemma to a chain of proper ideals of $R$. Then, by Zorn's lemma
there is a maximal element $M$ which is proper and claim follows. \qedsymbol

\begin{definition*}
  \label{def-prime-ideal}
  Let $R$ be a commutative ring and let $I$ be an ideal. We say that $I$
  is a prime ideal in $R$ if
  $$
  ab \in I \text{ implies that } a \in I \text{ or } b \in I.
  $$
\end{definition*}

\begin{lemma}
  \label{lemm-maximal-ideal}
  Every maximal ideal is a prime ideal.
\end{lemma} \vspace{1.8em}
\proof Let $M$ be a maximal ideal, by Definition \ref{def-maximal-ideal} if $xy \in M$ then
$x \in M$ or $y \in M$ since $M$ is proper, clearly follows the Definition \ref{def-prime-ideal}.

\begin{example*}
  Let $p$ be a prime number, then $p\ZZ = (p)$ which is a maximal ideal and by Lemma \ref{lemm-maximal-ideal} is also
  a prime ideal.
\end{example*}

\section{Homomorphism}
\begin{definition*}
  \label{def-homomorphism-ring}
  Let $R, S$ be rings. A map $\rho: R \longrightarrow S$ is a homomorphism of rings. Then image or range of $\rho$ is
  $$
  \im \rho = \{\rho(x) \ | \ x \in R\}.
  $$
  The kernel of $\rho$ is
  $$
  \Ker \rho = \{x \in R \ | \ \rho(x) = 0\}.
  $$
\end{definition*}

\begin{definition*}
  Let $I$ be an ideal of a ring $R$. The ring of all cosets of $I$ is the quotient ring $R/I$ of $R$ by $I$.
  The homomorphism $x \longmapsto x+1$ is the canonical projection of $R$ onto $R/I$.
\end{definition*}

\begin{definition*}
  Let $n$ be an positive integer number. The ring $\ZZ_{n}$ of the integers modulo $n$ is the quotient
  ring $\ZZ/\ZZ n$.
\end{definition*}

\begin{theorem}[Factorization Theorem]
  \label{theo-factorization}
  Let $I$ be an ideal of a ring $R$. Every homomorphism of ring $\rho: R \longrightarrow S$ whose
  kernel contains $I$ factors uniquely through the canonical projection $\phi: R \longrightarrow R/I$, then
  there is homomorphism $\varphi: R/I \longrightarrow S$ unique such that $\rho = \phi \circ \varphi$.
\end{theorem} \vspace{1.8em}
\proof $\phi((x + I)(y + U)) = \phi(xy + I) = \rho(xy) = \rho(x)\rho(y) = \varphi(x + I)\varphi(y + I)$
for all $x + I, y + I \in R/I$, and $\phi(1_{R}) = \phi(1_{R} + I ) = \rho(1_{R}) = 1_{R}$. So, claim follows. \qedsymbol

\begin{theorem}[Homomorphism Theorem]
  If $\rho: R \longrightarrow S$ is a homomorphism of rings, then
  $$
  R/\Ker \rho \cong \im \rho.
  $$
  Indeed, there is an isomorphism $\theta: R/\Ker f \longrightarrow \im f$ unique such that
  $\rho = \iota \circ \theta \circ \pi$ where $\iota: \im f \longrightarrow S$ is the 
  inclusion homomorphism and $\pi: R \longrightarrow R/\Ker f$ is the canonical projection.
\end{theorem} \vspace{1.8em}
\proof Clearly, there is an isomorphism of abelian groups $\theta: (R/\Ker f, +) \longrightarrow (\im f, +)$ 
unique such that $\rho = \iota \circ \theta \circ \pi$, equivalently , $\theta(x + \Ker \rho) = \rho(x)$
for all $x \in R$. Then $\rho$ is homomorphism and an isomorphism. \qedsymbol

\section{Noetherian Rings}

\begin{definition*}
  \label{def-ascending-chain}
  Let $R$ be a commutative ring, the ascending chain condition has three equivalent forms:
  \begin{enumerate}[a. ]
      \ii Every infinite ascending sequence $\af_{1} \subseteq \af_{2} \subseteq \dots \subseteq \af_{n} \subseteq \af_{n+1} \subseteq \dots$
      of ideals of $R$ terminates, then there exists $N > 0$ such that $\af_{n} = \af_{N}$ for all $n \geq N$, \label{(a)}
      \ii there is no infinite strictly ascending sequence $\af_{1} \subsetneq \af_{2} \subsetneq \dots \subsetneq \af_{n} \subsetneq \af_{n+1} \subsetneq \dots$
      of ideals of $R$, \label{(b)}
      \ii every nonempty set $\SC$ of ideals of $R$ has a maximal element (an element $\sff$ of $\SC$,
      not necessarily a maximal ideal of $R$, such that there is no $\sff \subsetneq \af \in \SC$). \label{(c)}
  \end{enumerate}
\end{definition*}

\begin{definition*}
  \label{def-noetherian-ring}
  A commutative ring is Noetherian when its ideals satisfy the ascending chain condition in the sense
  of Definition \ref{def-ascending-chain}.
\end{definition*}

\begin{proposition}
  A commutative ring $R$ is Noetherian iff every ideal of $R$ is finitely generated (as an ideal).
\end{proposition} \vspace{1.8em}
\proof Let $\af$ be an ideal of $R$. Let $\SC$ be the set of all finitely generated ideals of $R$
contained in $\af$. Then $\SC$ contains principal ideals and is not empty. If $R$ is Noetherian, then $\SC$
has a maximal element $\sff$ by (c). Then $\sff \subseteq \sff + (a) \in \SC$ for every $a \in \af$,
since $\sff + (a)$ is finitely generated, by $a$ and the generators of $\sff$. Since $\sff$ is maximal
in $\SC$ it follows that $\sff = \sff + (a)$ and $a \in \sff$. Hence $\af = \sff$ and $\af$ is finitely generated. \\ \\
Conversely, assume that every ideal of $R$ is finitely generated. Let $\af_{1} \subseteq \af_{2} \subseteq \dots \subseteq \af_{n} \subseteq \af_{n+1} \subseteq \dots$ be ideals of $R$. Then $\af = \bigcup_{n>0} \af_{n}$ is an ideal of $R$ and is finitely generated, by, say $a_{1}, \dots, a_{k}$. Then $a_{i} \in \af_{n_{i}}$
for some $n_{i} > 0$. If $N \geq n_{1}, \dots, n_{k}$, then $a_{N}$ contains $a_{1}, \dots, a_{k}$. Hence $\af \subseteq \af_{N}$, and
$\af \subseteq \af_{N} \subseteq \af_{n} \subseteq \af$ shows that $\af_{n} = \af_{N}$ for all $n \geq N$. \qedsymbol

\begin{theorem}[Hilbert Basis Theorem]
  Let $R$ be a commutative ring with identity. If $R$ is Noetherian, then $R[X]$ is Noetherian.
\end{theorem} \vspace{1.8em}
\proof Let $\UF$ be an udeal of $R[X]$. We construct a finite set of generators of $\UF$.
For every $n \geq 0$ let
$$
\af_{n} = \{r \in R \ | \ rX^{n} + a_{n-1}X^{n-1} + \dots + a_{0} \in \UF \text{ for some } a_{n-1}, \dots, a_{0} \in R\}.
$$
Then $\af_{n}$ is an ideal of $R$, since $\UF$ is an ideal of $R[X]$, and $\af_{n} \subseteq \af_{n+1}$, since
$f(x) \in \UF$ implies $Xf(X) \in \UF$. Since $R$ is Noetherian, the ascending sequence
$\af_{0} \subseteq \af_{1} \subseteq \dots \subseteq \af_{n} \subseteq \af_{n+1} \subseteq \dots$
terminates at some $\af_{m}(\af_{n} = \af_{m} \text{ for all } n \geq m)$. Also, each
ideal $\af_{k}$ has a finite generating set $\SC_{k}$. \\ \\
For each $s \in \SC_{K}$ there exists $g_{\SC} = sX^{k} + a_{k-1}X^{k-1} + \dots + a_{0} \in \UF$.
We show that $\UF$ coincides with the ideal $\BF$ generated by all $g_{\SC}$ with $s \in S_{0} \cup S_{1} \cup \dots \cup S_{m}$.
Hence $\UF$ is finitely generated, and $R[X]$ is Noetherian. Now let $f = a_{n}X^{n} + \dots + a_{0} \in \UF$
have degree $n \geq 0$. Then $a_{n} \in \af_{n}$. \\ \\
If $n \leq m$, then $a_{n} = r_{1}s_{1} + \dots + r_{k}s_{k}$ for some $r_{1}, \dots, r_{n} \in R$ and
$s_{1}, \dots, s_{k} \in \SC_{n}$, then $g = r_{1}g_{s_{1}} + \dots + r_{k}g_{s_{k}} \in \BF$ has degree at most $n$,
and the coefficient of $X^{n}$ in $g$ is $r_{1}s_{1} + \dots + r_{k}s_{k} = a_{n}$. Hence
$f-g \in \UF$ has degree less than $n$. Then $f-g \in \BF$, by induction $f \in \BF$. \\ \\
If $n > m$ then $a_{n} \in \af_{n} = \af_{m}$ and $a_{n} = r_{1}s_{1} + \dots + r_{k}s_{k}$ for some
$r_{1}, \dots, r_{n} \in R$ and $s_{1}, \dots, s_{k} \in S_{m}$, then $g = r_{1}g_{s_{1}} + \dots + r_{k}g_{s_{k}} \in \BF$
has degree at most $m$ and the coefficient of $X^{m}$ in $g$ is $r_{1}s_{1} + \dots + r_{k}s_{k} = a_{n}$.
Hence $X^{n-m}g \in \BF$ has degree at most $n$, and the coefficient of $X^{n}$ in $g$ is $a_{n}$.
As above, $f - X^{n-m}g \in \BF$ by induction and $f \in \BF$. \qedsymbol


\section{Principal Ideal Domain}
\begin{definition*}
  A principal ideal domain (or PID) is a domain (i.e., commutative ring with identity and no zero divisors)
  in which every ideal is principal.
\end{definition*}

\begin{definition*}
  An element $p$ of a domain $R$ is prime when $p$ is not zero 
  or a unit, and $p|a$ or $p|b$. An element $q$ of a domain $R$
  is irreducible when $q$ is not zero or a unit, and $q = ab$
  implies that $a$ is a unit or $b$ is a unit.
\end{definition*}

\begin{theorem}
  \label{theo-irreducible-elements}
  In a PID $R$, every element, other than $0$ and units, is a nonempty
  product of irreducible elements. If futhermore two nonempty products
  $p_{1}p_{2} \dots p_{m} = q_{1}q_{2} \dots q_{n}$ of irreducible elements are equal,
  then  $m=n$ and the terms can be indexed so that $Rp_{i} = Rq_{i}$ for all $i$.
\end{theorem} \vspace{1.8em}
\proof Suppose that $R$ has elements besides 0 and units that are not
product of irreducible elements, the principal ideal generated by these
elements, say $\IF$, consitute a set of ideals of $R$. Let's prove that
$\IF$ has a maximal element $Ri$. Otherwise, let $Ri_{1} \in IF$. Since
$Ri_{1}$ is not maximal, there exists $Ri_{1} \subsetneq Ri_{2} \in \IF$.
By induction, $Ri_{1} \subsetneq Ri_{2} \subsetneq \dots \subsetneq Ri_{n} \subsetneq Ri_{n+1} \subsetneq \dots$
Then $\mathfrak{i} = \bigcup_{n > 0} Ri_{n}$ is an ideal of $R$. Since $R$
is a PID, \mathfrak{i} is generated by some $i \in R$. Then $i \in Ri_{n}$
for some $n$, and $(i) \subseteq Ri_{n} \subsetneq Ri_{n+1} \subsetneq \mathfrak{i} = (i)$.
This contradiction shows that $\IF$ has a maximal element $Rm$, where $m$
is not 0, not unit and not irreducible. \\ \\
Let $m = ab$ for some $a, b \in R$, neither of which is 0 or a unit.
Then $Rm \subsetneq Ra$ and $Rm \subsetneq Rb$. Hence $a$ and $b$ cannot be neither
0 nor unit and are product of irreducible elements. But then so is
$m = ab$. This contradiction shows that every element of $R$, other than
0 and unit, is a product of irreducible elements. \\ \\
Now, suppose that $p_{1}p_{2} \dots p_{m} = q_{1}q_{2} \dots q_{n}$, where
$m, n > 0$ and all $p_{i}, q_{j}$ are irreducible. Since $q_{k}$ is
irreducible, $q_{k} = up_{m}$ for some unit $u \in R$ and $Rq_{k} = Rp_{m}$.
By induction, $m-1 = n-1$ and the remaining terms can be reindexed so that
$Rp_{1} = Ruq_{q} = Rq_{1}$ and $Rp_{i} = Rq_{i}$ for all $1 < i < m$. \qedsymbol

\begin{theorem}
  \label{theo-representative-irreducible}
  Every nonzero element of $R$ can be written as the product $p_{1}^{k_{1}} p_{2}^{k_{2}} \dots p_{n}^{k_{n}}$
  of a unit and of positive powers of distinct representative irreducible
  elements, which are unique up to the order of the terms.
\end{theorem} \vspace{1.8em}
\proof Theorem \ref{theo-irreducible-elements} implies Theorem \ref{theo-representative-irreducible}.
So, claim follows. \qedsymbol

\begin{definition*}
  \label{def-lcm}
  In a domain, an element $m$ is a least common multiple (l.c.m.) of two elements $a$ and $b$
  when $m$ is a multiple of $a$ and $b$.
\end{definition*}

\begin{definition*}
  \label{def-gcd}
  In a domain, an element $d$ is a greatest common divisor (g.c.d.) of two elements $a$ and $b$
  when $d$ divides $a$ and $b$.
\end{definition*}

\begin{proposition}
  In a PID $R$, every $a, b \in R$ have a least l.c.m and g.c.d. Moreover, $m = \lcm(a, b)$
  iff $Rm = Ra \cap Rb$, and $d = gcd(a, b)$ iff $Rd = Ra + Rb$. In particular,
  $d = gcd(a, b)$ implies $d = xa + yb$ for some $x, y \in R$.
\end{proposition} \vspace{1.8em}
\proof By Definition \ref{def-lcm}, $m = \lcm(a, b)$ iff $m \in Ra \cap Rb$, and $c \in Ra \cap Rb$ 
implies $c \in Rm$ iff $Rm = Ra \cap Rb$. As l.c.m exists then $Ra \cap Rb$ must be principal. \\ \\
By Definition \ref{def-gcd}, $d = \gcd(a, b)$ iff $a, b \in Rd$ and $a, b \in Rc$ implies
$c \in Rd$, iff $Rd$ is the smallest principal ideal of $R$ contains both $Ra$ and $Rb$.
The latter is $Ra + Rb$, since every ideal of $R$ is principal. Hence $d = \gcd(a, b)$
iff $Rd = Ra + Rb$, and then $d = xa + yb$ for some $x, y \in R$. As g.c.d. exists
then $Ra + Rb$ must be principal. \qedsymbol

\subsection{Modules over a PID}
\label{subsect-modules-over-pid}

\begin{theorem}
  \label{theo:1}
  Let $M$ over a PID $\CR$. There is a unique decresing sequence of proper ideals
  $$
  d_{1} \supseteq \dots \supseteq d_{n}
  $$
  such that $M$ is isomorphic to the sum of cyclic modules
  $$
  M \cong \bigoplus_{i} \CR/(d_{i}).
  $$
  The $d_{i}$s are called invariant factors of $M$.
\end{theorem} \vspace{1.8em}
\proof Let $\varphi$ be a $\CR$-linear map such that can be determined by $\varphi(e_{1}) = f_{1}, \dots, \varphi(e_{n}) = f_{n}$
where $e_{1}, \dots, e_{n}$ is the basis of $n$-dimensional $\CR$. Then $\varphi(e_{j}) = \sum_{i=1}^{n} c_{ij}e_{i}$, such that $(c_{ij})$ is the
matrix presentation of $\varphi$ with respect to the basis. Then
$$
\varphi(\CR) = \CR\varphi(e_{1}) \oplus \dots \oplus \CR\varphi(e_{n}) = \CR f_{1} \oplus \dots \oplus \CR f_{n},
$$
by aligned bases of $\CR$ and its module $\varphi(\CR)$, we can say that
$$
\CR = \CR v_{1} \oplus \dots \oplus \CR v_{n}, \hspace{3em} \varphi(R) = \CR a_{1}v_{1} \oplus \dots \oplus \CR a_{n}v_{n},
$$
where $a_{i}$s are nonzero integers. Then
$$
\CR/\varphi(R) \cong \bigoplus_{i} \CR/a_{i}\CR.
$$
Obvioulsy, $\CR/\varphi(R)$ is our $M$ and claim follows. \qedsymbol \\ \\
As an useful comment, we can calculate the invariant factors with the Smith Normal
Form (SNF). \\ \\
We need to remember that no every module has a basis, that's because
we use free module here.
\begin{definition*}
  \label{def-free-module}
  A free module is a module with a basis.
\end{definition*}
\begin{lemma}
  \label{lem:1}
  Let $\CR$ be a $n$-dimensional module over a PID, then every
  $\CR$-submodule of $\CR$ is an ideal.
\end{lemma} \vspace{1.8em}
\proof Let $x \neq 0 \in \CR$, then for $\CR x$ since all ideals in
$\CR$ are principal, it's clearly that $\CR x \cong \CR$ as $\CR$-modules.
\qedsymbol
\begin{lemma}
  \label{lem:2}
  Let $\CR$ be a commutative ring and $M$ be an $R$-module. Let $f$ be
  an $\CR$-linear and onto map such that $f: M \longrightarrow \CR$, then
  there is an $\CR$-module isomorphism $h: M \cong \CR^{n} \oplus \Ker f$
  where $h(m) = (f(m), *)$, making $f$ the first component of $h$.
\end{lemma} \vspace{1.8em}
\proof Let $\CR^{n} = \CR e_{1} \oplus \dots \oplus \CR e_{n}$ where
$e_{1}, \dots, e_{n}$ is the basis of $\CR$, let $m_{i} \in M$ such that
$f(m_{i}) = e_{i}$ then there is a map $g: \CR^{n} \longrightarrow M$ such that
$$
g(c_{1}e_{1} + \dots + c_{n}e_{n}) = c_{1}m_{1} + \dots + c_{n}m_{n},
$$
Now, we define the function $h: M \longrightarrow \CR^{n} \oplus \Ker f$ such that
$h(m) = (f(m), m - g(f(m)))$. \qedsymbol
\begin{theorem}
  \label{theo:2}
  Let $M \subset \CR$ be a free $\CR$-module of rank $n$ where $\CR$ is a PID,
  then for any $S$ submodule of $M$ is free of rank $\leq n$.
\end{theorem} \vspace{1.8em}
\proof The free $\CR$-module is $\CR^{n}$ by lemma \ref{lem:1}. By induction on $n$, let $S \subset \CR^{n+1}$ be a submodule.
We gonna show that $S$ is free of rak $\leq n+1$. The a projection of direct
sum $\phi: \CR \oplus \CR^{n} \longrightarrow \CR^{n}$ (i.e. $\CR^{n+1} = \CR \oplus \CR^{n}$), then $N = \phi(S) \subset \CR^{n}$ is free of rank $\leq n$.
Now, by lemma \ref{lem:2}
$$
S \cong N \oplus \Ker \phi \vert_{S},
$$
so $N \oplus \Ker \phi \vert_{S}$ is free of rank $\leq n+1$, so $S$ does. \qedsymbol

\begin{definition*}
  Let $\CR$ be a module and let $x \in \CR$, which is called a torsion
  element if there exists a nonzero $r \in \CR$ such that $rx = 0$. If $rx \neq 0$ for all $r \neq 0 \in R$, then the element $x$ is called a torsion-free.
\end{definition*}

\begin{definition*}
  Let $T$ be a module, we say that $T$ is called a torsion-free module, if every
  element of $T$ is a torsion-free module.
\end{definition*}

\begin{definition*}
  Let $T$ be a finitely torsion module over the PID $\CR$. By theorem \ref{theo:1},
  we write $T \cong R/(d_{1}) \oplus \dots \oplus R/(d_{m})$, then the $\CR$-cardinality of $T$ to be the ideal
  $$
  \card_{\CR}(T) = (d_{1}d_{2} \dots d_{m}).
  $$
\end{definition*}

\begin{theorem}
  \label{theo:3}
  Let $T_{1}$ and $T_{2}$ be two finitely generated torsion $\CR$-modules, then
  $$
  \card_{\CR}(T_{1} \oplus T_{2}) = \card_{\CR}(T_{1}) \card_{\CR}(T_{2}).
  $$
\end{theorem} \vspace{1.8em}
\proof We combine cyclic decompositions of $T_{1}$ and $T_{2}$
and then get $T_{1} \oplus T_{2}$. \qedsymbol \\ \\
If we pick $x_{1}, \dots, x_{n}$ the generating set for a torsion-free module $T$ as an $\CR$-module,
then we have a linear map $f: \CR^{n} \longrightarrow T$ where $f(e_{i}) = x_{i}$
for the basis $e_{1}, \dots, e_{n}$ of $\CR^{n}$ such there exists a linearly
indepedent sequence $y_{1}, \dots, y_{n}$ of $T$ such that $y_{j} = \sum_{i=1}^{n} a_{ij}x_{i}$ with $a_{ij} \in \CR$. By zorn's lemma, there is a linearly
independent subset of $T$ with maximal size $t_{1}, \dots, t_{d}$ such that
$\sum_{j=1}^{d} At_{j} \cong T^{d}$. Then we can get an isomorphism map
$$
T \rightarrow aT \hookrightarrow \sum_{j=1}^{d} Tt_{j} \rightarrow A^{d},
$$
for a linearly dependent set $x, t_{1}, \dots, t_{d}$ and a nontrivial linear
realtion $ax + \sum_{i=1}^{d} a_{i}t_{i} = 0$ with $a \neq 0$. Now, we can
say the following
\begin{lemma}
  \label{lem:3}
  Let $T$ be a finitely generated torsion-free module over a PID $\CR $ such that
  $T \neq 0$, then there is an embedding $T \hookrightarrow \CR^{d}$ for some
  $d \geq 1$ such that the image of $T$ intersects each standard coordinate
  axis of $\CR^{d}$.
\end{lemma} \vspace{1.8em}
Now, we use the above lemma to formulate the next theorem
\begin{theorem}
  \label{theo:4}
  Let $\CR$ be a PID, then every finitely generated torsion-free $\CR$-module
  is a free $\CR$-module.
\end{theorem} \vspace{1.8em}
\proof By lemma \ref{lem:3}, there is a module that embeds a finite
free $\CR$-module, then it's finite free too by theorem \ref{theo:2}. \qedsymbol \\ \\
As last, we have the following theorem
\begin{theorem}
  \label{lem:5}
  Let $\CR$ be a PID, every finitely $\CR$-module has the form $F \oplus T$ where
  $F$ is a finite free $\CR$-module and $T$ is a finitely generated torsion
  $\CR$-module. Moreover, $T \cong \bigoplus\limits_{j} \CR/(a_{j})$ with a nonzero $a_{j}$.
\end{theorem} \vspace{1.8em}
\proof Let $T$ be a finitely generated $\CR$-module, with generators
$x_{1}, \dots, x_{n}$. We define $f:\CR^{n} \longrightarrow T$ by $f(e_{i}) = x_{i}$. We know that 
$$
\CR^{n}/N \cong \left(\bigoplus\limits_{j}^{m} \CR/(a_{j})\right) \oplus \CR^{n-m},
$$
for some $m \leq n$, a quotient $\CR^{n}/N$ and nonzero $a_{j}$s. The direct
sum of the $A/(a_{j})$'s is a torsion module and $\CR^{n-m}$ is a finite
free $\CR$-module. \qedsymbol

\begin{example*}
  Describe, as a direct sum of cyclic groups, the cokernel of the map
  $\phi: \mathbb{Z}^{3} \longrightarrow \mathbb{Z}^{3}$ given by left multiplication by the matrix
  $$
\left(\begin{matrix} 15 & 6 & 9 \\ 6 & 6 & 6 \\ -3 & -12 & -12 \end{matrix}\right)
$$
\end{example*}
\proof  Let $\phi$ be a $\mathbb{Z}$-linear map such that can be determined by $\phi(e_{1}) = f_{1}, \dots,  \phi(e_{n}) = f_{n}$, where $e_{1}, \dots, e_{n}$ be the basis of $\mathbb{Z}^{n}$. Then $\phi(e_{j}) = \sum_{i=1}^{n} = c_{ij}e_{i}$ for $j = 1, \dots, n$, so $(c_{ij})$ is the matrix representation of $\phi$ with respect to the basis. \\ \\
Then,
  $$
  \phi(\mathbb{Z}^{n}) = \mathbb{Z}\phi(e_{1}) \oplus \dots \oplus \mathbb{Z}\phi(e_{n}) = \mathbb{Z}f_{1} \oplus \dots \oplus \mathbb{Z}f_{n},
  $$
  by aligned bases for $\mathbb{Z}^{n}$ and its modulo $\phi(\mathbb{Z}^{n})$, we can say that
  $$
  \mathbb{Z}^{n} = \mathbb{Z}v_{1} \oplus \dots \oplus \mathbb{Z}v_{n}, \hspace{1 em} \phi(\mathbb{Z}^{n}) = \mathbb{Z}a_{1}v_{1} \oplus \dots \oplus \mathbb{Z}a_{n}v_{n}
  $$
  where $a_{i}$'s are nonzero integers. Then
  $$
  \mathbb{Z}^{n}/\phi(\mathbb{Z}^{n}) \cong \bigoplus_{i=1}^{b} \mathbb{Z}/a_{i}\mathbb{Z}.
  $$
  Now, for our solution we need to get the Smith Normal Form, since each $a_{i}$ is the $M_{i,i}$ element of the matrix, the Smith Normal Form of the cokernel is:
  $$
  \left(\begin{matrix}
      3 & 0 & 0 \\
      0 & 3 & 0 \\
      0 & 0 & 18
  \end{matrix}\right)
  $$
  So, we can describe the cokernel as the sum of cyclic groups:
  $$
  \mathbb{Z}^{3}/\phi(\mathbb{Z}^{3}) \cong  \mathbb{Z}/3\mathbb{Z} \oplus \mathbb{Z}/3\mathbb{Z} \oplus \mathbb{Z}/18\mathbb{Z}
  $$ \qedsymbol


\section{Commutative Rings}
In this Section, let's see many properties related this kind of rings that we haven't seen yet.
\begin{definition*}
  \label{def-transporter}
  Let $\af$ and $\bff$ be ideals of a commutative ring $R$ and let $S$ be a subset of $R$. The set
  $$
  \af:S = \{r \in R \ | \ rs \in \af \text{ for all } s \in S\}
  $$
  is an ideal and it's called the \textbf{transporter} of $S$ into $\af$.
\end{definition*}

\begin{definition*}
  \label{def-radical}
  Let $R$ be a commutative ring. The radical $\Rad \af$ of an ideal $\af$ is the intersection
  of all primes ideals of $R$ that contain $\af$.
\end{definition*}

\begin{proposition}
  $\Rad \af = \{x \in R \ | \ x^{n} \in \af \text{ for some } n > 0 \}$.
\end{proposition} \vspace{1.8em}
\proof Let $x \in R$ and let $\rf$ be the intersection of all prime ideals that contain
$\af$. If $x \in R/\rf$, then $x \notin \pf$ for some prime ideal $\pf \subset \af$,
$x^{n} \notin \pf$ for all $n > 0$ since $\pf$ is prime, and $x^{n} \notin \af$ for all $n > 0$. \\ \\
Suppose $x^{n} \notin \af$ for all $n > 0$. By Zorn's lemma there is an ideal $\pf$
that contains $\af$, and contains no $x^{n}$, and is maximal with thes properties.
Let $a, b \in R/\pf$. By choice of $\pf, \pdf + (a)$ contains some $x^{m}$ and
$\pf + (b)$ contains some $x^{n}$. Then $x^{m} = p + ra, x^{n} = q + sb$ for some
$p, q \in \pf$ and $r, s \in R$, $x^{m+n} = pq + psb + qra + rsab \in \pf + (ab) \subsetneq \pf$, and
$ab \notin \pf$. Thus $\pf$ is a prime ideal, since $x \notin \pf$, it follows that $x \notin \rf$. \qedsymbol

\begin{definition*}
  \label{def-semiprime}
  An ideal $\af$ of a commutative ring $R$ is semiprime when it's an intersection of prime ideals.
\end{definition*}

\begin{definition*}
  \label{def-ideal-primary}
  Let $R$ be a commutative ring and let $\rf$ be an ideal of $R$. We say that $\rf$ is \textbf{primary}
  when $\rf \neq R$ and, for all $x, y \in \rf$ implies $x \in \rf$ or $y^{n} \in \rf$
  for some $n > 0$. An ideal $\rf$ of $R$ is $\pf$-primary when $\rf$ is primary and
  $\Rad \rf = \pf$.
\end{definition*}

\begin{lemma}
  \label{lemm-ideal-noetherian}
  Every ideal of a Noetherian ring $R$ is the intersection of finitely many irreducible ideals of $R$.
\end{lemma} \vspace{1.8em}
\proof Let $\bff$ be the ideal of a ring $R$. Suppose $\bff$ is not the intersection
of finitely many irreducible ideals of $R$. Is this is false, then the set of such ideals of $R$
is empty, since $R$ is Noetherian, there is a maximal ideal $\pf$ which is not irreducible.
Therefore, $\pf = \af \cap \bff$ for some ideals $\af, \bff \subsetneq \pf$. By the maximality
of $\pf, \af$ and $\bff$ are intersections of finetely many irreducible ideals, but then so is $\pf$,
a contradiction. \qedsymbol

\begin{theorem}
  In a Noetherian ring, every ideal is the intersection of finitely manu primary ideals.
\end{theorem} \vspace{1.8em}
\proof By Lemma \ref{lemm-ideal-noetherian}, we only need to show that every irreducible ideal $\rf$
of a Noetherian ring $R$ is primary. Suppose that $ab \in \rf$ and $b \notin \Rad \rf$.
Let $\af_{n} = \rf:b^{n}$. Then $a \in \af_{1}$, $\rf \subset \af_{n}$, $\af_{n}$ is an ideal,
and $\af_{n} \subseteq \af_{n+1}$, since $xb^{n} \in \rf$ implies $xb^{n+1} \in \rf$. Since $R$ is Noetherian,
the ascending sequence $(\af_{n})_{n > 0}$ terminates, hence $\af_{2n} = \af_{n}$ if $n$ is large enough.
Let $\bff = \rf + Rb^{n}$. If $x \in \af_{n} \cap \bff$, then $xb^{n} \in \rf$ and $x = t + yb^{n}$ for some
$t \in \rf$ and $y \in R$, whence $tb^{n} + yb^{2n} \in \rf$, $yb^{2n} \in \rf$, $y \in \af_{2n} = \af_{n}$,
$yb^{n} \in \rf$, and $x = t + yb^{n} \in \rf$. Hence $\af_{n} \cap \bff = \rf$.
Now, $\bff \subsetneq \rf$, since $b^{n} \notin \rf$. Therefore $\af_{n} = \rf$, hence
$\af_{1} = \rf$ and $a \in \rf$. \qedsymbol

\begin{definition*}
  \label{eef-ring-extension}
  A ring extension of a commutative ring $R$ is a commutative ring $E$ of which $R$
  is a subring.
\end{definition*}

\begin{definition*}
  \label{def-r-module}
  Let $R$ be a ring. A $R$-module is an abelian group $M$ together
  with an action $(r, x) \longmapsto rx$ of $R$ on $M$ such that
  $r(x + y) = rx + ry, (r + s)x = rx + sx, r(sx) = (rs)x$ and $1_{R}x = x$,
  for all $r, s \in R$ and $x, y \in M$. A submodule $R$-module $M$ is additive
  msubgroup $N$ of $M$ such that $x \in N$ implies $rx \in N$ for every $r \in R$.
\end{definition*}

\begin{definition*}
  An element $\alpha$ of a ring extension $E$ of $R$ is integral over $R$ when it sastifies
  the following conditions:
  \begin{itemize}
    \ii $f(\alpha) = 0$ for some monic polynomial $f \in R[X]$,
    \ii $R[\alpha]$ is a finetely generated submodule of $E$,
    \ii $\alpha$ belongs to a subring of $E$ that is a finitely generated $R$-module.
  \end{itemize}
\end{definition*}

\subsection{Integral Extension}

\begin{definition*}
  A ring extension $R \subset E$ is integral, and $E$ is integral over $R$, when
  every element of $E$ is integral over $R$.
\end{definition*}

\begin{definition*}
  A ring extension $E$ of $R$, an ideal $\UF$ of $E$ lies over an ideal $\af$
  of $R$ when $\UF \cap R = \af$.
\end{definition*}

\begin{proposition}
  If $E$ is an ring extension of $R$ and $\UF \subset E$ lies over $\af \subset R$,
  then $\af/R$ may be identified with a subring of $E/\UF$, if $E$
  is integral over $R$, then $E/\UF$ is integral over $R/\af$.
\end{proposition}  \vspace{1.8em}
The inclusion homomorphism $R \longrightarrow E$ induces a homomorphism $R \longrightarrow E/\UF$
whose kernel is $\UF \cap R = \af$, and in injective homomorphism $R/\af \longrightarrow E/\UF$,
$r + \af \longmapsto r + \UF$. Hence $R/\af$ may be identified with a subring of $E/\UF$.
If $\alpha \in E$ is integral over $R$, then $\alpha + \UF$, then $\alpha + \UF \in E/\UF$ is integral
over $R/\af$. \qedsymbol

\begin{definition*}
  The integral closure of a ring $R$ in a ring extension $E$ of $R$ is the subgring $\overline{R}$
  of $E$ of all elements of $E$ that are integral over $R$. The elements of $\overline{R} \subseteq E$
  are the algebraic integerss of $E$ (over $R$).
\end{definition*}

\begin{definition*}
  A domain $R$ is integrally closed when its integral closure in its quotient field $Q(R)$ is
  $R$ itself (when no $\alpha \in Q(R)\backslash R$ is integral over $R$).
\end{definition*}

\begin{proposition}
  Let $R$ be a domain and let $E$ be an algebraic extension of its quotient field. The integral
  closure $\overline{R}$ of $R$ in $E$ is an integrally closed domain whose quotient field is $E$.
\end{proposition} \vspace{1.8em}
\proof Every $\alpha \in E$ is algebraic over $Q(R)$, $r\alpha$ is integral over $R$ for some $r \in R$,
hence $E = Q(\overline{R})$. If $\alpha \in E$ is integral over $\overline{R}$, then $\alpha$
is integral over $R$ and $\alpha \in \overline{R}$, so $\overline{R}$ is integrally closed. \qedsymbol

\subsection{Localization}

\begin{definition*}
  \label{def-localization}
  A multiplicative subset of a commutative ring $R$ is a subset $S$ of $R$ that contain the identity element of $R$ and
is closed under multiplication. A multiplicative subset $S$ is proper when $0 \notin S$.
\end{definition*}

\begin{definition*}
  If $S$ is a proper multiplicative subset of a commmutative ring $R$, then $S^{-1}R$ is the ring of faction of
  $R$ with denominators in $S$.
\end{definition*} \\
If $R$ is a domain, then $S = R\backslash \{0\}$ is a proper multiplicative subset
and $S^{-1}R$ is the field of fractions or quotient fild $Q(R)$ of $R$.

\begin{definition*}
  Let $S$ be a proper multiplicative subset of $R$. The contraction of an ideal
  $\UF$ of $S^{-1}R$ is $\UF^{C} = \{a \in R \ | \ a/1 \in \UF\}$. The expansion
  of an ideal $\af$ of $R$ is $\af^{E} = \{a/s \in S^{-1}R \ | \ a \in \af, s \in S\}$.
\end{definition*}

\begin{definition*}
  The localization of a commutative ring $R$ at a prime ideal $p$ is the ring
  of fractions $R_{p} = (R\p)^{-1}R$.
\end{definition*}

\begin{definition*}
  A commutative ring is local when it has only one maximal ideal.
\end{definition*}
We need to remark that Localization tranfers properties from local rings to more general rings.

\begin{theorem}
  Every homomorphism of a ring $R$ into an algbraically closed field $L$ can be extended to every
  integral extension $E$ of $R$.
\end{theorem} \vspace{1.8em}
\proof If $R$ is a field, then $E$ is a field and $E$ is an algebraic extension of $R$. \\
Now, let $R$ be local and let $\varphi: R \longrightarrow L$ be a homomorphism whose kernel
is the maximal ideal $\mf$ of $R$. Then $\varphi$ factors through the projection $R \longrightarrow R/\mf$
and induces a homomorphism $\phi: R/\mf \longrightarrow L$. There is a maximal ideal $\MF$ of $E$
that lies over $\mf$. The field $R/\mf$ may be identified with a subfield of $E/\MF$.
Then $E/\MF$ is algebraic over $R/\mf$ and $\pi: R/\mf \longrightarrow L$ can be extended to $E/\MF$.
Hence $\varphi$ can be extended to $E$. \\ \\
Finally, let $\varphi: R \longrightarrow L$ be any homomorphism. Then $\pf = \Ker \varphi$ is a
prime ideal of $R$ and $S = R/\pf$ is a proper multiplicative subset of $R$ and of $E$.
Therefore $\phi$ extends to $S^{-1}E$, which is integral over $S^{-1}R$, hence $\varphi$
extends to $E$. \qedsymbol

\subsection{Dedekind Domains}
\begin{definition*}
  \label{def-fractional-ideal}
  A fractional ideal of a domain $R$ is a subset of its quotient field $Q$ of the form $\af/c = \{a/c \in Q \ | \ a \in \af\}$,
  where $\af$ is an ideal of $R$ and $c \in R$, $c \neq 0$.
\end{definition*}

\begin{proposition}
  Let $R$ be a domain and let $Q$ be its quotient field. Every finitely generated submodule
  of $Q$ is a fractional ideal of $R$. If $R$ is Noetherian, then every fractional ideal
  of $R$ is finetely generated as a submodule.
\end{proposition} \vspace{1.8em}
\proof If $n > 0$ and $q_{1} = a_{1}/c_{1}, \dots, q_{n} = a_{n}/c_{n} \in Q$, then
$$
Rq_{1} + \dots + Rq_{n} = Rb_{1}/c + \dots + Rb_{n}/c = (Rb_{1} + \dots + Rb_{n})/c,
$$
where $c = c_{1} \dots c_{n}$, hence $Rq_{1} + \dots + Rq_{}$ is a fractional ideal of $R$. Conversely,
if every ideal $\af$ is finetely generated, $\af = Rb_{1} + \dots + Rb_{n}$ for some
$b_{1}, \dots, b_{n} \in R$ then every fractional ideal $\af/c = Rb_{1}/c + \dots + Rb_{n}/c$ is a
finetely generated  submodule of $Q$.

\begin{definition*}
  A Dedekind domain is a domain that satisfies the equivalent condition:
  \begin{enumerate}[1. ]
      \ii Every nonzero ideal of $R$ is invertible (as a fractional ideal),
      \ii every nonzero fractional ideal of $R$ is invertible,
      \ii every nonzero ideal of $R$ is a product of prime ideals of $R$,
      \ii every nonzero ideal of $R$ can be written uniquely as a product of positive
      powers of distinct prime ideals of $R$.
  \end{enumerate}
  Futhermore, $R$ is Noetherian and every prime ideal of $R$ is maximal.
\end{definition*}

\begin{theorem}[Krull Intersection Theorem]
  Let $\af \neq R$ be an ideal of a Noetherian ring $R$ and let $\rf = \bigcap_{n > 0} \af^{n}$.
  Then $\af\rf = \rf$ and $(1-a)\rf = 0$ for some $a \in \af$. If $R$ is a domain,
  or if $R$ is local, then $\rf = 0$.
\end{theorem} \vspace{1.8em}
\proof Let $\qf$ be a primary ideal that contains $\af\ifr$ and let $\pf$ be its radical.
Then $\pf^{n} \subset \qf$ for some $n$ and $\ifr \subset \qf$, otherwise $\af \subset \qf$, since
$\qf$ is primary and $\ifr \subset \af^{n} \subset \qf$ anyway. Since $\ai\ifr$ is an
intersection of primary ideals, this implies $\ifr \subset \af\ifr$ and $\ifr = \af\ifr$.
Suppose that $(1-a)\ifr = 0$ for some $a \in \af$. Then $1-a \neq 0$. If $R$ is a domain,
then $(1-a)\ifr = 0$ for some $a \in \af$. Then $1-a \neq 0$. If $R$ is a domain, then
$(1-a)\ifr = 0$ implies $\ifr = 0$. If $R$ is local, then $1-a$ is a unit and
again $(1-a)\ifr = 0$ implies $\ifr = 0$. \qedsymbol

\subsection{Krull Dimension}
\begin{definition*}
  In a commutative ring $R$, the height $\hgt \pf$ of a prime ideal $\pf$ is
  the least upper bound of the lengths of strictly descending sequences
  $\pf = \pf_{0} \subsetneq \pf_{1} \subsetneq \dots \subsetneq \pf_{m}$
  of prime ideals of $R$.
\end{definition*}

\begin{definition*}
  \label{def-spectrum-ring}
  The \textbf{spectrum} of a commutative ring is the set of its prime ideals,
  partially ordered by inclusion. The Krull dimension or dimension $\dim R$ of $R$
  is the least upper bound of the heights of the prime ideals of $R$.
\end{definition*}

\begin{lemma}
  Let $R$ be a domain and let $\PF$ be a prime ideal of $R[X]$. If
  $\PF \cap R = 0$, then $\PF$ has height at most 1.
\end{lemma} \vspace{1.8em}
\proof Let $Q$ be the quotient field of $R$ and let $S = R \backslash \{0\}$,
so that $S^{-1}R = Q$. Since every $r \in R \backslash 0$ is a unit in
$Q$ and in $Q[X]$. There is a homomorphism $\theta: S^{-1}(R[X]) \longrightarrow Q[X]$, which
sends $(a_{0} + \dots + a_{n}X^{n})/r \in S^{-1}(R[X])$ to $(a_{0}/r) + \dots + (a_{n}/r)X^{n} \in Q[X]$.
If $g(X) = q_{0} + q_{1}X + \dots + q_{n}X^{n} \in Q[X]$, then rewriting $q_{0}, q_{1}, \dots, q_{n}$
with a common denominator puts $g$ in the form $g = f/r$ for some $f \in R[X]$ and 
$r \in R$, hence $\theta$ is an isomorphism. Thus $S^{-1}(R[X]) \cong Q[X]$ is a PID. \\ \\
Now, let $\PF \cap R = 0$ and let $0 \neq \QF \subset \PF$ be a prime ideal of $R[X]$.
Then $\QF \cap R = 0$, and $\QF^{E} \subseteq \PF^{E}$ are nonzero prime ideals of
the PID $S^{-1}(R[X])$. Hence $\QF^{E}$ is a maximal ideal. $\QF^{E} = \PF^{E}$, and
$\QF = \PF$. \qedsymbol

\begin{theorem}
  If $R$ is Noetherian domain, then $\dim R[X] = 1 + \dim R$.
\end{theorem} \vspace{1.8em}
\proof Let $(X)$ be a prime ideal of $R[X]$ and $R[X]/(X) \cong R$. In particular,
$\dim R[X]$ is infinite when $\dim R$ is infinite, and we may assume that $n = \dim R$ is finite.
We prove by induction on $n$ that $\dim R[X] \leq n+1$. If $n = 0$, then 0 is a maximal ideal of
$R$, $R$ is a field, $R[X]$ is a PID, and $\dim R[X] = 1$. Now, let $n > 0$ and
$$
\PF_{0} \subsetneq \PF_{1} \subsetneq \dots \subsetneq \PF_{m}
$$
be prime ideals of $R[X]$. We want to show that $m \leq n+1$. Since $n \leq 1$ we suppose that $m \leq 2$
and $\PF_{m-1} \cap R \neq 0$. Indeed, suppose that $\PF \cap R = 0$. Then $\PF_{2} \cap R \neq 0$
and there exists $0 \neq a \in \PF_{m-2} \cap R$. Now, $\PF_{m-2}$ has height at least 2
and is not minimal over $(a)$.  \\ \\
Now, $\pf = \PF_{m-1} \cap R$ is a nonzero prime ideal of $R$. Then $\dim R/\pf \leq \dim R-1 = n-1$.
By induction, $\dim (R/\pf)[X] \leq n$. The projection $R \longrightarrow R/\pf$ induces
a surjective homomorphism $R[X] \longrightarrow (R/\pf)[X]$ whose kernel is a nonzero prime ideal
$\PF$ of $R[X]$, which consists of all $f \in R[X]$ with coefficients in $\pf$. Then $\PF \subset \PF_{m-1}$,
since $\pf \subset \PF_{m-1}$, $\dim R[X]/\PF = \dim(R/\pf)[X] \leq n$, and the sequence
$$
\PF_{0}/\PF \subsetneq \PF_{1}/\PF \subsetneq \dots \subsetneq \PF_{m-1}/\PF
$$
has length $m-1 \leq n$, so that $m \leq n+1$. \qedsymbol

\section{Rational Fraction}

\begin{definition*}
  Let $K$ be a field. A \textbf{partial fraction} is a rational fration $f/q^{r} \in K(X)$ in which
  $q$ is monic and irreducible, $r \geq 1$, and $\deg f < \deg q$. Then $f, q$, and
  $r$ are unique.
\end{definition*}

\begin{lemma}
  \label{lemm-rational-fraction}
  Every rational fraction can be written uniquely in reduced form.
\end{lemma} \vspace{1.8em}
\proof Given $f/g$, divide $f$ and $g$ by the leading coefficient of $g$ and then by
a monic g.c.d. of $f$ and $g$, the result is in reduced form. \\ \\
Let $f/g = p/q, fq = gp$, with $g, q$ monic and $\gcd(f, g) = \gcd(p, q) = 1$. Then $q$
divides $gp$, since $\gcd(p, q) = 1$, $q$ divides $g$. Similarly, $g$ divides $q$.
Since $q$ and $g$ are monic,  $q = g$. Then $p = f$. \qedsymbol

\begin{lemma}
  Every rational fraction can be written uniquely as the sum of a polynomial and
  a polynomial-free fraction in reduced form.
\end{lemma} \vspace{1.8em}
\proof By Lemma \ref{lemm-rational-fraction}, we start with a rational fraction $f/g$ in reduced form.
Polynomial division yields $f = gq + r$ with $q, r \in K[X]$ and $\deg r < \deg g$.
Then $f/g = q + r/g$, $r/g$ is polynomial-free and is in reduced form, since $g$ is monic and
$\gcd(r, g) = \gcd(f, g) = 1$. Conversely let $f/g = p + s/h$, with $p \in K[X]$, $\deg s < \deg h$,
$h$ monic, and $\gcd(s, h) = 1$. Then $f/g = (ph + s)/h$. Both fractions are in reduced form,
hence $g=h$ and $f = ph + s = pg + s$, by Lemma \ref{lemm-rational-fraction}. Uniqueness
in polynomial division then yields $p = q$ and $s = r$. \qedsymbol

\section{Unique Factorization Domains}
\begin{definition*}
  \label{def-ufd}
  A unique factorization domain (or UFD) is a domain $R$ (i.e., a commutative ring with identity
  and no zero divisors) in which
  \begin{enumerate}[1. ]
      \ii Every element, other than 0 and units, is a nonempty product of irreducible
      elements of $R$, and
      \ii if two nonempty products $p_{1}p_{2} \dots p_{m} = q_{1}q_{2} \dots q_{n}$ of irreducible elements
      of $R$ are equal, then $m = n$ and the terms can be indexed so that $Rp_{i} = Rq_{i}$
      for all $i$.
  \end{enumerate}
\end{definition*}

\begin{definition*}
  A polynomial $p$ over a UFD $R$ is primitive when no irreducible element of $R$ divides
  all the coefficients of $p$.
\end{definition*}

\begin{lemma}
  Every nonzero polynomial $f(X) = Q(X)$ can be written in the form $f(X) = tf^{*}(X)$, where
  $t \in Q, t \neq 0$, and $f^{*}(X) \in R[X]$ is primitive. Moreover, $t$ and $f^{*}$ are
  unique up to multiplication by units of $R$.
\end{lemma} \vspace{1.8em}
\proof We have $f(X) = (a_{0}/b_{0}) + (a_{1}/b_{1})X + \dots + (a_{n}/b_{n})X^{n}$, where
$a_{i}, b_{i} \in R$ and $b_{i} \neq 0$. Let $b$ be a common denominator (for instance, $b = b_{0}b_{1} \dots b_{n}$).
Then $f(X) = (1/b)(c + c_{1}X + \dots + c_{n}X^{n})$ for some $c_{i} \in R$. Factoring
out $a = \gcd(c_{0}, c_{1}, \dots, c_{n})$ yields $f(X) = (a/b)f^{*}(X)$, where $f^{*}$ is primitive. \\ \\
Suppose that $(a/b) \ g(X) = (c/d) \ h(X)$, where $g, h$ are primitive. Since $g$ and $h$ are primitive,
$ad$ is a g.c.d. of the coefficients of $ad \ g(X)$, and $bc$ is a g.c.d, of the coefficients
of $bc \ h(X)$. Hence, $bc = adu$ for some unit $u$ of $R$, so that $g(X) = u  h(X)$ and
$(a/b)u = c/d$ in $Q$. \qedsymbol

\begin{proposition}[Eisenstein's Criterion]
  Let $R$ be a UFD and let $f(X) = a_{0} + a_{1}X + \dots + a_{n}X^{n} \in R[X]$. If $f$ is primitive
  and there exists an irreducible element $p$ of $R$ such that $p$ divides $a_{i}$ for all $i < n$,
  $p$ does not divide $a_{n}$, and $p^{2}$ does not divide $a_{0}$, then $f$ is irreducible.
\end{proposition} \vspace{1.8em}
\proof Suppose that $f = gh$, let $g(X) = b_{0} + b_{1}X + \dots + b_{r}X^{r}$ and $h(X) = c_{0} + c_{1}X + \dots + c_{s}X^{s} \in R[X]$,
where $r = \deg g$ and $s = \deg h$. Then $a_{k} = \sum_{i+j=k} b_{i}c_{j}$ for all $k$. In particular,
$a_{0} = b_{0}c_{0}$. Since $p^{2}$ does not divide $a_{0}$, $p$ does not divide both $b_{0}$ and $c_{0}$.
But $p$ divides $a_{0}$, so $p$ divides, say, $b_{0}$, but not $c_{0}$. Also, $p$ does not divide $b_{r}$,
since $p$ does not divide $a_{n} = b_{r}c_{s}$. Hence there is a least $k \leq r$ such that $p$ does not divide
$b_{k}$, and then $p$ divides $b_{i}$ for all $i < k$. Now $p$ divides
every term of $\sum_{i+j=k} b_{i}c_{j}$ except for $b_{k}c_{0}$. Hence $p$ does not divide $a_{k}$.
Therefore $k = n$, since $k \leq r \leq r+s = n$ this implies $r = n$, and $h$ is contant. \qedsymbol

\begin{example*}
  Let $f = 3X^{3} + 4X - 6 \in \ZZ[X]$ is irreducible in $\ZZ[X]$. Indeed, $f$ is primitive, 2 divides
  all the coefficients of $f$ except the leading coefficient, and 4 does not divide the constant
  coefficient. Futhermore, $f$ is also irreducible in $\QQ[X]$, and so is $\frac{5}{6}f = \frac{5}{2}X^{3} - \frac{5}{3}X - 5$.
\end{example*}


\chapter{Module Theory}
Already we have seen some introduction to modules on the Sub-Section \ref{subsect-modules-over-pid}. In this chapter, let's
expand the theory about modules. We can see modules as a generalization of abelian groups, which sastifies all the axioms
of a vector space (except for scalars, which can come from another ring or field). From now, let $R$ be a fixed ring.

\begin{definition*}
  \label{def-module}
  A (left) $R$-module is an additive group $M$ together with a map $\cdot: R \times M \longrightarrow M$ (the $R$-action) called
  \textit{scalar multiplication}, which sastifies for all $r, s \in R$ and $x, y \in M$ the following
  properties:
  \begin{enumerate}[1. ]
      \ii $r \cdot (x + y) = r \cdot x + r \cdot y$,
      \ii $(r + s) \cdot x = r \cdot x + s \cdot x$,
      \ii $r \cdot (s \cdot x) = (r \cdot s) \cdot x$,
      \ii $1 \cdot x = x$.
  \end{enumerate}
\end{definition*}

\begin{example*}
  Let $K$ be a field. A $K$-module is a vector space over $K$ with scalars on the left.
\end{example*}

\begin{example*}
  Every abelian group $A$ is a unital $\ZZ$-module, in which $nx$ is the usual integer multiple (i.e., $nx = x + x + \dots + x$, when $n > 0$).
\end{example*}

\begin{example*}
  Let $\UF$ be a left ideal in $R$. Then $\UF + \UF \subset \UF$ and $R \UF \subset \UF$, so $\UF$
  is a submodule of $R$. In this way, an $R$-module generalizes the notion of left ideal.
\end{example*}
Now, let $M$ be a $R$-module and $N \subset M$, the \textbf{factor group} $M/N$ becomes an $R$-modules by
$$
\cdot : R \times M/N \longrightarrow M/N \text{ defined by } r(m + N) = rm + N,
$$
for all $r \in R, m \in M$. Then, it's called the \textbf{quotient} or \textbf{factor module}
of $M$ by $N$.

\begin{definition*}
  A submodule of a left $R$-module $M$ is an additive subgroup $A$ of $M$ such that
  $x \in A$ implies $ax \in A$ for all $r \in R$.
\end{definition*}

\begin{definition*}
  \label{def-homomorhism-module}
  A map $f: M \longrightarrow N$ of $R$-modules is called an $R$-homomorphism if $f$ is $R$-linear
  (i.e., $f(rx + y) = rf(x) + f(y)$ for all $r \in R$ and $x, y \in M$).
\end{definition*}

\begin{example*}
  Let $M$ be a $R$-module where $R$ is a commutative ring in the sense of Definition \ref{def-commutative-ring} and $r \in R$.
  Then
  $$
  \lambda_{r}: M \longrightarrow M \text{ by } x \mapsto rx
  $$
  is an $R$-homomorphism.
\end{example*}

\begin{example*}
  Let $R$ be a commutative ring and $M, N$ are $R$-modules, then
  $$
  \Hom_{R}(M, N) = \{f: M \longrightarrow N \ | \ f \text{ an } R\text{-homomorphism} \}
  $$
  is an $R$-module with the usual $+$ for functions and the $R$-action $\cdot$ given by $r \cdot f: x \mapsto rf(x)$.
\end{example*}

\begin{definition*}
  \label{def-annihilator}
  The \textbf{Annihilator} of a left $R$-module $M$ is the ideal $\Ann(M) = \{r \in R \ | \ rx = 0 \text{ for all } x \in M \}$
  of $R$.
\end{definition*}
We need to remark that a left $R$-module is \textit{faithful} when its annihilator is 0.

\begin{lemma}
  \label{lemm-properties-ann}
  Let $M$ be an $R$-module and $m, m'$ elements in $M$. Then
  \begin{enumerate}[(i) ]
      \ii \label{l-i} $\Ann_{R}(m) \subset R$ is a left ideal.
      \ii \label{l-ii}$ \rho_{m}: R \longrightarrow M$ given by $r \mapsto rm$ is an $R$-homomorphism and satifies $\Ker \rho_{m} = \Ann_{R}(m)$.
      \ii \label{l-iii} If $\rho_{m}$ is the homomorphism in (\ref{l-ii}), then $\rho_{m}$ induces an $R$-isomorphism $\overline{\rho_{m}}: R/\Ann_{R}(m) \longrightarrow Rm$.
      \ii \label{l-iv} If $R$ is a commutative ring and $Rm \subset Rm'$, then $\Ann_{R}(m) \subset \Ann_{R}(m')$.
  \end{enumerate}
\end{lemma} \vspace{1.8em}
\proof (\ref{l-i}) and (\ref{l-ii}) are trivial. For (\ref{l-iii}), we can use First Isomorphism Theorem \ref{theo-first-isomorphism}
for modules and claim follows. For (\ref{l-iv}), suppose that $m = am'$ for some $a \in R$. If $rm' = 0$, then
$ram' = arm' = 0$. \qedsymbol

\begin{proposition}
  Let $M$ be an $R$-module. Then $M$ is a cyclic $R$-module iff there exists a left ideal $\UF$ in $R$
  sastifying $M \cong R/\UF$.
\end{proposition} \vspace{1.8em}
\proof $R/\UF = \langle 1 + \UF \rangle = R(1 + \UF)$ is cylcic and claim follows by Lemma \ref{lemm-properties-ann}.

\begin{definition*}
  \label{def-direct-product-modules}
  The direct product of a family $(A_{i})_{i \in I}$ of left $R$-modules is their cartesian product $\prod_{i \in I} A_{i}$ in the sense of Definition
  \ref{def-direct-product} with componentwise addition and action of $R$:
  $$
  (x_{i})_{i \in I} + (y_{i})_{i \in I} = (x_{i} + y_{i})_{i \in I}, r(x_{i})_{i \in I} = (rx_{i})_{i \in I}.
  $$
\end{definition*}

\begin{definition*}
  \label{def-direct-sum-modules}
  The sum direct product (or external sum) of a family $(A_{i})_{i \in I}$ of left $R$-modules is the submodule of $\prod_{i \in I} A_{i}$:
  $$
  \bigoplus_{i \in I} A_{i} = \{(x)_{i \in I} \in \prod_{i \in I} A_{i} \ | \ x_{i} = 0 \text{ for almost all } i \in I\}.
  $$
\end{definition*}

\begin{definition*}
  \label{def-internal-sum-modules}
  A left $R$-module $M$ is the internal direct sum $M = \oplus_{i \in I} A_{i}$ of submodules $(A_{i})_{i \in I}$ when
  every element of $M$ can be written uniquely as a sum $\sum_{i \in I} a_{i}$, where $a_{i} \in A_{i}$ for
  all $i$ and $a_{i} = 0$ for almost all $i$.
\end{definition*}
We need to remark that a nonzero $R$-module is \textbf{descomposable} if it's isomorphic to the direct
sum of two nonzero submodules. Otherwise is \textbf{indecomposable}.

\begin{proposition}
  \label{prop-equiv-sums}
  Let $(M_{i})_{i \in I}$ be left $R$-modules. For a left $R$-module $M$ the following conditions
  are equivalent:
  \begin{enumerate}[(1) ]
      \ii \label{e-1} $M \cong \bigoplus_{i \in I} M_{i}$,
      \ii \label{e-2} $M$ contains submodules $(A_{i})_{i \in I}$ such that $A_{i} \cong M_{i}$ for all $i$
      and every element of $M$ can be written uniquely as a sum $\sum_{i \in I} a_{i}$, where
      $a_{i} \in A_{i}$ for all $i$ and $a_{i} = 0$ for almost all $i$,
      \ii \label{e-3} $M$ contains submodules $(A_{i})_{i \in I}$ such that $A_{i} \cong M_{i}$ for all $i$,
      $M = \sum_{i \in I} A_{i}$, and $A_{j} \cap (\sum_{i \neq j} A_{i}) = 0$ for all $j$.
  \end{enumerate}
\end{proposition} \vspace{1.8em}
\proof (\ref{e-1}) implies (\ref{e-2}). Since $\bigoplus_{i \in I} M_{i}$ contains submodules
$M'_{i} = \iota(M_{i}) \cong M'_{i}$ such that every element of $\bigoplus_{i \in I} M_{i}$ can
be written uniquely as a sum $\sum_{i \in I} a_{i}$, where $a_{i} \in M'_{i}$ for all $i$
and $a_{i} = 0$ for almost all $i$. If $\theta: \bigoplus_{i \in I} M_{i} \longrightarrow M$
is an isomorphism, then the submodules $A_{i} = \theta(M'_{i})$ of $M$
have similar properties. \\ \\
(\ref{e-2}) implies (\ref{e-3}). By (\ref{e-2}), $M = \sum_{i \in I} A_{i}$, moreover, 
if $x \in A_{j} \cap (\sum_{i \neq k} A_{i})$, then $x$ is a sum $x = \sum_{i \in A} a'_{i}$
in which $a'_{j} = x, a'_{i} = 0$ for all $i \neq j$, and a sum $x = \sum_{i \in I} a''_{i}$
in which $a''_{j} = 0 \in A_{j}, a''_{i} \in A_{i}$ for all $i$, $a''_{i} = 0$ for almost all $i$,
by (\ref{e-2}), $x = a'_{j} = a''_{j} = 0$. \\ \\
(\ref{e-3}) implies (\ref{e-2}). By (\ref{e-3}), $M = \sum_{i \in I} A_{i}$, so that every
element of $M$ is a sum $\sum_{i \in I} a_{i}$, where $a_{i} \in A_{i}$ for all $i$
and $a_{i} = 0$ for almost all $i$. If $\sum_{i \in I} a'_{i} = \sum_{i \in I} a''_{i}$ where
$a'_{i}, a''_{i} \in A_{i}$, then for every $j \in I$, $a''_{j} - a'_{j} = \sum_{i \neq j} (a'_{i} - a''_{i}) \in A_{i} \cap (\sum_{i \neq j} A_{i})$
and $a''_{j} = a'_{j}$. \\ \\
(\ref{e-2}) implies (\ref{e-1}). The inclusion homomorphism $A_{i} \longrightarrow M$ induce a
module homomorphism $\theta: \bigoplus_{i \in I} A_{i} \longrightarrow M$, namely $\theta((a_{i})_{i \in I}) = \sum_{i \in I} a_{i}$.
Then $\theta$ is bijective by (\ref{e-2}). The isomorphisms $M_{i} \cong A_{i}$, then induce an
isomorphism $\bigoplus_{i \in I} M_{i} \cong \bigoplus_{i \in I} A_{i} \cong M$. \qedsymbol \\
We need to remark that by Proposition \ref{prop-equiv-sums}, internal and external sums differ
only by isomorphisms, that is, if $M$ is an external sum of modules $(A_{i})_{i \in I}$, then $M$
is an internal direct sum of submodules $A_{i} \cong M_{i}$, if $M$ is an internal direct sum
of submodules $(A_{i})_{i \in I}$, then $M$ is isomorphic to the external direct sum $\bigoplus_{i \in I} A_{i}$.

\begin{example*}
  Let $R = K[x]$. For $n \geq 1$, $\langle x^{n} \rangle$ is an ideal, and therefore $M = R/\langle x^{n} \rangle$ 
  is a module which is indecomposable.
\end{example*}
\proof  From the internal characterisation of direct product groups, we need only show that any
two submodules intersect nontrivially. We show indeed that any nonzero $N \leq M$ contains
$x^{n-1} + \langle x^{n} \rangle$. We suppose that

$$
N \in \alpha_{j}x^{j} + \dots + \alpha_{n-1}x^{n-1} + \langle x^{n} \rangle, \alpha_{j} \neq 0 := n.
$$
Then, $n\alpha_{j}^{-1}x^{n-1-j} = x^{n-1} + \langle x^{n} \rangle$. \qedsymbol


\section{Free Modules}
We expand the Definition \ref{def-free-module} in the following way.
\begin{definition*}
  \label{def-expand-free-module}
  A nonzero $R$-module $M$ is called a free $R$-module if there exists a basis
  for $M$, that is, a subset $\BC$ of $M$ satisfying:
  \begin{enumerate}[(i) ]
      \ii $M = \langle \BC \rangle$ (\ie $\BC$ generates or spans $M$).
      \ii $\BC$ is linearly independent (\ie $\sum_{\BC} r_{x}x = 0$ for all $x \in \BC, r \in R$).
  \end{enumerate}
\end{definition*}

\begin{example*}
  Let $M$ be a $R$-cyclic. Then $M$ is a free $R$-module if there exists an $x \in M$
  satisfying $M = Rx$ and $\Ann_{R}(x) = 0$. It follows that $M$ is a free $R$-module
  iff $M \cong R$.
\end{example*}

\begin{example*}
  $\QQ$ is not a free $\ZZ$-module.
\end{example*}

\begin{theorem}[Universal Property of Free Modules]
  Let $\BC = \{x_{i}\}_{I}$ be a basis for a free $R$-module $M$. If $N$ is
  an $R$-module and $y_{i}, i \in I$, elements in $N$, then there exists
  a unique $R$-homomorphism $f: M \longrightarrow N$ such that
  $x_{i} \mapsto y_{i}$, for all $i \in I$.
\end{theorem} \vspace{1.8em}
\proof If $z \in M$, there exists a unique $r_{i} \in R$, almost all $r_{i} = 0$
such that $z = \sum_{I} r_{i}x_{i}$. In particular, the uniqueness of the $r_{i}$
implies that $f: M \longrightarrow N$ given by $z \mapsto \sum_{I} r_{i}y_{i}$ is well-defined.
Clearly, $f$ is uniquely determined by $x_{i} \mapsto y_{i}$ and $f$ is an $R$-homomorphism. \qedsymbol

\begin{lemma}
  Let $M$ and $N$ be free $R$-modules on bases $\BC$ and $\CK$, respectively.
  If there exists a bijection $g: \BC \longrightarrow \CK$ (\ie $|\BC| = |\CK|$),
  then $M \cong N$.
\end{lemma} \vspace{1.8em}
\proof The maps $g$ and $g^{-1}$ of sets induce inverse $R$-isomorphisms $M \longrightarrow N$
and $N \longrightarrow M$. \qedsymbol

\section{Noetherian Modules}

\begin{definition*}[The Maximum Principle]
  \label{def-maximum-principle}
  If $S$ is a non-empty set of submodules of $M$, then $S$ contains a maximal element,
  that is a module $M_{0} \in S$ such that if $M_{0} \in N$ with $N \in S$, then
  $N = M_{0}$.
\end{definition*}

\begin{proposition}
  Let $M$ be a $R$-module. Then the following are equivalent:
  \begin{enumerate}[(1) ]
      \ii \label{q-1} Every submodule of $M$ is finetely generated,
      \ii \label{q-2} $M$ satifies the ascending chain condition, that is, if $M_{i} \subset M$
      are submodules and
      $$
      M_{1} \subset M_{2} \subset \dots \subset M_{n} \subset \dots,
      $$
      then there exists a positive integer $N$ such that $M_{N} = M_{N+i}$ for all $i \geq 0$.
      We say every ascending chain of submodules of $M$ stabilizes. Equivalently,
      ther exists no infinite chain 
      $$
      M_{1} < M_{2} < \dots M_{n} < \dots,
      $$
      \ii \label{q-3} $M$ satifies \textit{The Maximum Principle} in the sense of
      Definition \ref{def-maximum-principle}.
  \end{enumerate}
\end{proposition} \vspace{1.8em}
\proof (\ref{q-1}) implies (\ref{q-2}). Let
$$
\CK: M_{1} \subset M_{2} \subset \dots \subset M_{n} \subset \dots  
$$
be a chain of submodules of $M$. It follows that the subset $M' = \bigcup_{i=1}^{\infty} M_{i} \subset M$
is a submodule. By (\ref{q-1}), it's finitely generated, so we can write $M' = \sum_{i=1}^{n} Rx_{i}$
for some $x_{i} \in M'$. By definition, $x_{i} \in M_{j_{i}}$ some $j_{i}$. Let
$s$ be the maximum of the finitely many $j_{i}'$s. Then $M' = M_{S}$. It follows
that $M_{s} = M' = M_{s+i}$ for all $i \geq 0$. \\  \\
(\ref{q-2}) implies (\ref{q-3}). Suppose that $S$ is a non-empty set of submodules of $M$.
Let $M_{1}$ lie in $S$. If $M_{1}$ is not maximal, there exists an $M_{2} \in S$ with
$M_{1} < M_{2}$. Inductively, if $M_{i}$ is not maximal, there exist an $M_{i+1}$ in $S$
with $M_{i} < M_{i+1}$. By the a.c.c., the sequence
$$
M_{1} < M_{2} < \dots M_{i} < \dots
$$
must terminate. \\ \\
(\ref{q-3}) implies (\ref{q-1}). Let $M \subset N$ be a submodule and set
$$
S = \{M_{i} \ | \ M_{i} \subset N \text{ is a finitely generated submodule} \}.
$$
Then $(0) \in S$ so $S \neq \varnothing$. By assumption, there exists a maximal element
$M' \in S$. If $N \neq M'$, then there exists $x \in N\backslash M'$. But $M'$
finitely generated means that $M' + Rx \subset N$ is also finitely generated, so
the submodule $M' + Rx$ of $N$ lies in $S$. This contradicts the maximality
of $M'$. Hence $N = M'$ is finitely generated. \qedsymbol

\begin{definition*}
  \label{def-noeth-ring}
  Let $R$ be a commutative ring. We say that $R$ is a Noethereian ring
  if $R$ is a Noetherian $R$-module.
\end{definition*}
Definition \ref{def-noeth-ring} coincides with the Definition \ref{def-noetherian-ring}.

\begin{proposition}
  \label{prop-sequence-noetherian}
  Let $M$ be an $R$-module and $N$ a submodule of $M$. Then $M$ is $R$-Noetherian iff
  $N$ and $M/N$ are $R$-Noetherian. In particular, if
  $$
  0 \longrightarrow M' \longrightarrow M \longrightarrow M'' \longrightarrow 0
  $$
  is an exact sequence of $R$-modules with two of the modules $M$, $M'$, $M''$ being
  $R$-Noetherian, then they all are $R$-Noetherian.
\end{proposition} \vspace{1.8em}
\proof Since $N_{0} \subset N$ is a submodule, then $N_{0} \subset M$ is a submodule
hence finitely generated $-$ or any ascending chain in $N$ is
an ascending chain in $M$. Thus $N$ is $R$-Noetherian. By the Correspondence
Principle, a (countable) chain of submodules in $M/N$ has the form
$M_{1}/N \subset M_{2}/N \subset \dots$ where $N \subset M_{1} \subset M_{2} \subset \dots$
chain of submodules of $M$. Thus there exists an $r$ such that $M_{r} = M_{r+j}$
for all $j \geq 0$ and hence $M_{r}/N = M_{r+j}/N$ for all $j \geq 0$. \qedsymbol

\begin{theorem}
  Let $R$ be Noetherian ring. If $M$ is a finitely generated $R$-module,
  then $M$ is $R$-Noetherian.
\end{theorem} \vspace{1.8em}
\proof Suppose $M = \sum_{i=1}^{n} Rx_{i}$. Let $f: R^{r} \longrightarrow M$ be the
$R$-epimorphism given by $e_{i} \mapsto x_{i}$, where $\{e_{1}, \dots, e_{n}\}$
is the standard basis for $R^{n}$. Since $R$-Noetherian sinc
$M \amalg N$ is $R$-module because $(M \amalg N)/N \cong M$ where $N, M$ are Noetherian $R$-modules.
Hence, so is $M \cong R^{}/\Ker f$ by Proposition \ref{prop-sequence-noetherian}.

\section{Hilbert's Theorems}

\begin{theorem}[Hilbert Basis Theorem]
  If $R$ is a Noetherian ring so is the ring $R[t_{1}, \dots, t_{n}]$.
\end{theorem} \vspace{1.8em}
\proof By induction on $n$, it suffices to show that $R[t]$ is Noetherian.
Let $\BC \subset R[t]$ be an ideal. We must show that $\BC$ is finitely
generated. Let
$$
\UF = \{r \in R \ | \ r = \text{ lead } f, f \in \BC\}.
$$
First, let's show that $\UF$ is an ideal. Let $a, b \in \UF$ and $r \in R$ with
$ra + b$ nonzero. Choose $f, g \in \BC$ say of degrees $m$ and $n$ respectively,
satisfying lead $f = a$ and lead $g = b$. Set $h = rt^{n}f + t^{m}g$ in $\BC$.
Then $ra + b = $ lead $h$ proving $\UF$ is an ideal. As $R$ is Noetherian,
$\UF = (a_{1}, \dots, a_{n})$ for some $a_{1}, \dots, a_{n} \in \UF$
with $n \in \ZZ^{+}$. Choose $f_{id_{i}}$ in $\BC$ such that $a_{i} = $ lead
$f_{id_{i}}$ and $\deg f_{id_{i}} = d_{i}$. Let $\BC_{0} = (f_{1d_{1}}, \dots, f_{nd_{n}})$,
and ideal in $R[t]$, and $N = \max\{d_{1}, \dots, d_{n}\}$. \\ \\
Let $f \in \BC$ with lead $f = a$ and $\deg f = d$. Suppose that $d > N$.
There exist $r_{i} \in R$ satisfying $a = \sum_{i=1}^{n} r_{i}a_{i}$,
hence $f - \sum_{i=1}^{n} r_{i}t^{d-d_{i}}f_{i, d_{i}}$ lies in $\BC$
and has degree less than $d$. It follows by induction that there exists a $g \in \BC_{0}$
such that $f-g$ lies $\BC$ with $\deg(f-g) \leq N$. As the $R$-module $M = \sum_{i=0}^{n} Rt^{i}$
is finitely generated and $R$ is Noetherian, $M$ is a Noetherian $R$-module.
In particular, the submodule $M_{0} = \{f \in \BC \ | \ \deg f \leq N \}$ is
finitely generated. If $M_{0} = \sum_{i=0}^{m} R[t]g_{i}$, then
$\BC = \BC + \sum_{i=0}^{m} R[t]g_{i}$ is finitely generated. \qedsymbol

\begin{definition*}
  Let $R \subset S$ be commutative rings. We say that $S$ is a finitely generated
  commutative $R$-algebra (or an affine $R$-algebra when $R$ is a field)
  if there exists $x_{1}, \dots, x_{n}$ in $S$ satisfying $S = R[x_{1}, \dots, x_{n}]$
  as rings.
\end{definition*}

\begin{proposition}
  Let $R$ be a commutative ring and $S$ a finitely generated commutative $R$-algebra. If
  $R$ is Noetherian so is $S$.
\end{proposition} \vspace{1.8em}
\proof Let $S = R[x_{1}, \dots, x_{n}]$. Since $R[t_{1}, \dots, t_{n}]$ is Noetherian
and we have a ring epimorphism $R[t_{1}, \dots, t_{n}] \longrightarrow R[x_{1}, \dots, x_{n}]$
via $f(t_{1}, \dots, t_{n}) \mapsto f(x_{1}, \dots, x_{n})$, all ideals of $S$ are
finitely generated by the Correspondence Principle. \qedsymbol

\begin{theorem}[Hilbert Nullstellensatz, Strong Form]
  Suppose that $F$ be an algebraically closed field and $R = F[t_{1}, \dots, t_{n}]$.
  Let $f, f_{1}, \dots, f_{r}$ be elements in $R$ and $\UF = (f_{1}, \dots, f_{r}) \subset R$.
  Suppose that $f(a) = 0$ for all $a \in Z_{F}(\UF)$. Then there exists an integer
  $m$ such that $f^{m} \in \UF$ (\ie f $\in \sqrt{U}$). In particular, if
  $\UF$ is a prime ideal, then $f \in \UF$.
\end{theorem} \vspace{1.8em}
\proof We may assume that $f$ is nonzero. Let $S = R[t]$. Define the ideal $\BC$ in
$S$ by $\BC = (f_{1}, \dots, f_{r}, 1-tf) \subset S$. If $\BC < S$, then there
exists a point $(a_{1}, \dots, a_{n+1}) \in Z_{F}(\BC)$. Thus $f_{i}(a_{1}, \dots, a_{n}) = 0$
for all $i$ and $1-a_{n+1}f(a_{1}, \dots, a_{n+1}) = 0$. In particular,
$(a_{1}, \dots, a_{n})$ lies in $Z_{F}(\UF)$. By hypothesis, this means that $(a_{1}. \dots, a_{n}) = 0$
which in turn implies that $1 = 0$, a contradiction. Thus $\BC = S$. So we can write
$$
1 = \sum_{i=1}^{r} g_{i}f_{i} + g \cdot (1-tf)
$$
for some $g, g_{i} \in S$. Substituting $1/f$ for $t$ and clearing denominators yields the
result. \qedsymbol

  
\chapter{Field Theory}

\begin{definition*}
  \label{def-field}
  A field is a commutative ring (necessarily a domain) whose nonzero elements
  constitute a group under multiplication.
\end{definition*}

\begin{definition*}
  \label{def-subfield}
  A subfield of a field $F$ is a subset $K$ of $F$ such that $K$ is an
  additive subgroup of $F$ on $K \backslash \{0\}$ is a multiplicative subgroup
  of $F \backslash \{0\}$.
\end{definition*}
Equivalently to Definition \ref{def-subfield}, $K$ is a subfield of $F$ iff
\begin{enumerate}[I. ]
    \ii $0, 1 \in K$,
    \ii $x, y \in K$ implies $x-y \in K$,
    \ii $x, y \in K, y \neq 0$ implies $xy^{-1} \in K$.
\end{enumerate}
Then $x, y \in K$ implies $x+y \in K$ and $xy \in K$, so that $K$ inherits an addition
and a multiplication from $F$, and $K$ is a field under these inherited operations,
this field $K$ is also called a subfield of $F$.

\begin{example*}
  $\QQ$ is a subfield of $\RR$, and $\RR$ is a subfield of $\CC$.
\end{example*}

\begin{definition*}
  A homomorphism of a field $K$ into a field $L$ is a mapping $\varphi: K \longrightarrow L$
  such that $\varphi(1) = 1, \varphi(x+y) = \varphi(x) + \varphi(y)$ and
  $\varphi(xy) = \varphi(x)\varphi(y)$ for all $x, y \in K$.
\end{definition*}

\begin{proposition}
  The characteristic of a field is either 0 or a prime number.
\end{proposition} \vspace{1.8em}
\proof REIMAINDER

\begin{proposition}
  Every field $K$  has a smallest subfield, which is isomorphic to $\QQ$
  if $K$ has characteristic 0, to $\ZZ_{p}$ if $K$ has characteristic $p \neq 0$.
\end{proposition} \vspace{1.8em}
\proof REIMAINDER

\begin{definition*}
  An element $r$ of a field $K$ is an $n$th root of unity when $r^{n} = 1$.
\end{definition*}

\section{Extensions}

\begin{definition*}
  \label{def-extension-field}
  A field extension of a field $K$ is a field $E$ of which $K$ is a subfield.
\end{definition*}

\begin{definition*}
  \label{def-k-homomorphism}
  Let $K \subseteq E$ and $K \subseteq F$ be field extensions of $K$. A $K$-homomorphism of $E$ into $F$
  is a field homomorphism $\varphi: E \longrightarrow F$ that is the identity
  on $K(\varphi(x) = x \text{ for all } x \in K)$.
\end{definition*}

\begin{definition*}
  The degree $[E:K]$ of a field extension $K \subseteq E$ is its dimension
  as a vector space over $K$. A field extension $K \subseteq E$ is finite when
  it has finite degree and is finite otherwise.
\end{definition*}

\begin{example*}
  $[\CC:\RR] = 2$.
\end{example*}

\begin{proposition}
  If $K \subseteq E \subseteq F$, then $[F:K] = [F:E][E:K]$.
\end{proposition} \vspace{1.8em}
\proof Let $(\alpha_{i})_{i \in I}$ be a basis of $E$ over $K$ and let
$(\beta_{j})_{j \in J}$ be a basis of $F$ over $E$. Every element
of $F$ is a linear combination of $\beta_{j}$'s with coefficients
in $E$, which are themselves linear combinations of $\alpha_{i}$'s with
coefficints in $K$. \\ \\
Hence every element of $F$ is a linear combination of $\alpha_{i} \beta_{j}$'s with
coefficients  in $K$. Moreover, $(\alpha_{i} \beta_{j})_{(i, j) \in I \times J}$
is a linearly independent family in $F$, viewed as a vector space over $K:$ if
$\sum_{(i, j) \in I \times J} x_{i, j}\alpha_{i}\beta_{j} = 0$  (with $x_{i, j} = 0$
for almost all $(i, j)$), then $\sum_{j \in J} \left( \sum_{i \in I} x_{i, j}\alpha_{i} \right) \beta_{j} = 0$,
$\sum_{i \in I} x_{i, j}\alpha_{i} = 0$ for all $j$, and $x_{i, j} = 0$ for all
$i, j$. Thus $(\alpha_{i}\beta_{j})_{(i, j) \in I \times J}$ is a basis
of $F$ over $K$ and $[F:K] = |I \imes J| = |I||J| = [F:E][E:K]$. \qedsymbol

\begin{definition*}
  Let $K \subset E$ be a field extension. An element of $E$ is \textbf{algebraic}
  over $K$ when $f(x) = 0$ for some nonzero polynomial $f(X) \in K[X]$. Otherwise,
  $\alpha$ is \textbf{transcendental} over $K$.
\end{definition*}


\begin{definition*}
  Let $\alpha$ be algebraic over $K$. The unique monic irreducible polynomial
  $q = \Irr(\alpha:K) \in K[X]$ such that $q(\alpha) = 0$ is the
  irreducible polynomial of $\alpha$ over $K$, the degree of $\alpha$
  over $K$ is the degree of $\Irr(\alpha:K)$.
\end{definition*}
  
\begin{definition*}
  A field extension $K \subseteq E$ is algebraic, and $E$ is algebraic over
  $K$, when every element of $E$ is algebraic over $K$. A field extension
  $K \subseteq E$ is transcendental over $K$, when some element of $E$
  is transcendental over $K$.
\end{definition*}

\begin{example*}
  $\CC$ is algebraic extension of $\RR$ and $\RR$ is a transcendental
  extension of $\QQ$.
\end{example*}

\subsection{The Algebraic Closure}
\begin{definition*}
  A field is algebraically closed when it satisfies the following
  equivalent conditions:
  \begin{enumerate}[1. ]
      \ii The only algebraic extension of $K$ is $K$ itself.
      \ii In $K[X]$, every irreducible polynomial has degree 1.
      \ii Every nonconstant polynomial in $K[X]$ has a root in $K$.
  \end{enumerate}
\end{definition*}

\begin{example*}
  $\CC$ is algebraically closed.
\end{example*}

\begin{theorem}
  \label{theo-homomorphism-field}
  Every homomorphism of a field $K$ into a algebraically closed field can be extended to every
  algebraic extension of $K$.
\end{theorem} \vspace{1.8em}
\proof Let $E$ be an algebraic extension of $K$ and let $\alpha$ be an homomorphism of $K$
into an algebraically closed field $L$. If $E = K(\alpha)$ is simple extension of $K$,
and $q = \Irr(\alpha:K)$, then $&&^{\varphi}q \in L[X]$ has a root in $L$, since
$L$ is algebraically closed and $\varphi$ can be extended to $E$. \qedsymbol \\ \\
We need to remark that $K$ can be embedded into an algebraically closed field $\overline{K}$
that is algebraic over $K$, and then every algebraic extension of $K$ can be embedded in $\overline{K}$,
by Theorem \ref{theo-homomorphism-field}.

\begin{definition*}
  An algebraic closure of a field $K$ is an algebraic extension $\overline{K}$ of $K$ that
  is algebraically closed.
\end{definition*}
  
\begin{definition*}
  An algebraic extension $K \subseteq E$ is separable when the irreducible polynomial of its
  elements are separable (have no multiple roots).
\end{definition*}
  
\begin{example*}
  $f(X) = X^{4} + 2X^{2} + 1 \in \RR[X]$ factors as $f(X) = (X^{2} + 1)^{2} = (X - i)^{2}(X + i)^{2}$ in $\CC[X]$
  and has two multiples roots in $\overline{\RR} = \CC$, it is not separable. But $X^{2} + 1 \in \RR[X]$
  is separable.
\end{example*}

\begin{definition*}
  An element $\alpha$ is separable over $K$ when $\alpha$ is algebraic over $K$ and $\Irr(\alpha:K)$
  is separable. An algebraic extension $E$ of $K$ is separable, and $E$
  is separable over $K$, when every element of $E$ is separable over $K$.
\end{definition*}

\begin{definition*}
  \label{def-purely-inseparable}
  An algebraic extension $K \subseteq E$ is purely inseparable, and $E$ is purely
  inseparable over $K$, when no element of $E \backslash K$ ise separable over $K$.
\end{definition*}

\begin{proposition}
  For every algebraic extension $E$ of $K$, $S = \{\alpha \in E \ | \ $
  $\alpha \text{ is separable over } K \}$
  is a subfield of $E$, $S$ is separable over $K$, and $E$ is purely inseparable over $S$.
\end{proposition} \vspace{1.8em}
\proof First, 0 and 1 $\in K$ are separable over $K$. If $\alpha, \beta \in E$ are separable
over $K$, then $K(\alpha, \beta)$ is separable over $K$ and $\alpha - \beta, \alpha\beta^{-1} \in K(\alpha, \beta)$
are separable over $K$. Thus $S$ is a subfield of $E$. Clearly $S$ is separable over $K$.
If $\alpha \in E$ is separable over $S$, then $S(\alpha)$ is separable over $K$ and $\alpha \in S$. \qedsymbol

\begin{definition*}
  A field extension $K \subseteq E$ is totally transcendental, and $E$ is
  transcendental over $K$, when every element of $E \backslash K$ is transcendental over $K$.
\end{definition*}


\begin{proposition*}
  For every field $K, K((X_{i})_{i \in I})$ is totally transcendental over $K$.
\end{proposition*} \vspace{1.8em}
\proof First, we show that $K(X)$ is totally transcendental over $K$. For clarity's
sake we prove the equivalent result that $K(\gamma) \cong K(X)$ is totally transcendental
over $K$ when $\gamma$ is transcendental over $K$. Let $\alpha \in K(\gamma)$, so that $\alpha = f(\gamma)/g(\gamma)$
for some $f, g \in K[X], g \neq 0$. If $\alpha \notin K$, then $\alpha g(X) \notin K[X], \alpha g(X) \neq f(X)$,
and $\alpha g(X) - f(X) \neq 0$ in $K(\alpha)[X]$. But $\alpha g(\gamam) - f(\gamma) = 0$, so
$\gamma$ is algebraic over $K(\alpha)$. Hence $K(\gamma) = K(\alpha)(\gamma)$ is finite
over $K(\alpha)$. Therefore $[K(\alpha):K]$ is infinite. Otherwise, $[K(\gamma):K]$
would be finite. Hence $\alpha$ is transcendental over $K$. \\ \\
That $K[X_{1}, \dots, X_{n}]$ is totally transcendental over $K$ now follows by
induction on $n$. Let $\alpha \in K(X_{1}, \dots, X_{n})$ be algebraic over $K$.
Then $\alpha \in K(X_{1}, \dots, X_{n-1})(X_{n})$ is algebraic over $K(X_{1}, \dots, X_{n-1})$.
By the case, $n=1, \alpha \in K(X_{1}, \dots, X_{n-1})$, and the
induction hypothesis yields $\alpha \in K$. \\ \\
Finally, let $\alpha = f/g \in K((X_{i})_{i \in I})$ be algebraic over $K$. The
polynomials $f$ and $g$ have only finitely many nonzero terms. Hence
$\alpha \in K ((X_{i})_{i \in J})$ for some finite subset $J$ of $I$. Therefore $\alpha \in K$. \qedsymbol

\section{Separability}

\begin{definition*}
  Two field extension $K \subseteq E \subseteq L$, $K \subseteq F \subseteq L$ are linearly disjoint
  over $K$ when they satisfy the following equivalent conditions:
  \begin{enumerate}[1. ]
      \ii.$(\alpha_{i})_{i \in I} \in E$ linearly independent over $K$ implies
      $(\alpha_{i})_{i \in I}$ linearly independent over $F$.
      \ii $(\beta_{j})_{j \in J} \in F$ linearly independent over $K$ implies
      $(\beta_{i})_{i \in I}$ linearly independent over $E$.
      \ii $(\alpha_{i})_{i \in I} \in E$ and $(\beta_{j})_{j \in J} \in F$ linearly
      independent over $K$ implies $(\alpha_{i}\beta_{j})_{(i, j) \in I \times J} \in L$
      linearly independent over $K$.
  \end{enumerate}
\end{definition*}

\begin{definition*}
  A transcendence base $B$ of a field extension $K \subset E$ is separating
  when $E$ is separable (algebraic) over $K(B)$.
\end{definition*}

\begin{definition*}
  A field extension $E$ of $K$ is separable, and $E$ is separable over $K$,
  when every finitely generated subfield $K \subseteq F$ of $E$ has a separating
  transcendental base.
\end{definition*}

\section{Galois Theory}

\begin{definition*}
  A polynomial $f \in K[X]$ splits in a field extension $E$ of $K$
  when it has a factorization $f(X) = a(X - \alpha_{1})(X - \alpha_{2}) \dots (X- \alpa_{n})$ in $E[X]$.
\end{definition*}

\begin{definition*}
  Let $K$ be a field. A splitting field over $K$ of a polynomial $f \in K[X]$
  is a field extension $E$ of $K$ such that $f$ splits in $E$ and $E$ is 
  generated over $K$ by the roots of $f$. A splitting field over $K$ of a set 
  $\SC \subseteq K[X]$ of polynomials is a field extension $E$ of $K$ such
  that every $f \in \SC$ splits in $E$ and $E$ is generated over $K$ by the
  roots of all $f \in \SC$.
\end{definition*}

\begin{lemma}
  If $E$ and $F$ are splitting fields of $\SC \subseteq K[X]$ over $K$, and $F \subseteq \overline{K}$,
  then $\varphi E = F$ for every $K$ homomorphism $\varphi: E \longrightarrow \overline{K}$.
\end{lemma} \vspace{1.8em}
\proof Every $f \in \SC$ has unique factorizations $f(X) = a(X - \alpha_{1}) \dots (X - \alpha_{n})$
in $F[X] \subseteq \overline{K}[X]$. Since $\varphi$ is the identity on $K$, $f, &^{\varphi}f = a(X - \varphi \alpha_{1}) \dots (X - \varphi \alpha_{n})$
in $\overline{K}[X]$, therefore $\vaprhi\{\alpha_{1, \dots, \alpha_{n}}\} = \{\beta_{1}, \dots, \beta_{n}\}$.
Thus $\varphi$ sends the set $R$ of all roots $f \in \SC$ in $E$ onto the set $\SC$ of all roots
$f \in \SC$ in $F$. Then, $\varphi$ sends $E = K(R)$ onto $K(S) = F$. \qedsymbol

\begin{definition*}
  A normal extension of a field $K$ is an algebraic extension of $K$ that
  satisfies the following equivalent conditions:
  \begin{enumerate}[1. ]
      \ii $E$ is the splitting field over $K$ of a set of polynomials,
      \ii $\varphi E = E$ for every $K$-homomorphism $\varphi: E \longrightarrow \overline{K}$,
      \ii $\varphi E \subseteq E$ for every $K$-homomorphism $\varphi: E \longrightarrow \overline{K}$,
      \ii $\rho E = E$ for every $K$-automorphism $\rho:$ of $\overline{K}$,
      \ii $\rho E \subseteq E$ for every $K$-automorphism $\rho:$ of $\overline{K}$,
      \ii every irreducible polynomial $q \in K[X]$ with a root in $E$ splits in $E$.
  \end{enumerate}
\end{definition*}

\begin{definition*}
  A Galois extension of a field $K$ is a normal and separable extension $E$ of $K$,
  then $E$ is Galois over $K$.
\end{definition*}

\begin{definition*}
  The \textbf{Galois group} $\Gal(E:K)$ of a Galois extension $E$ of a field $K$, also
  called the Galois group of $E$ over $K$, is the group of all $K$-automorphisms
  of $E$.
\end{definition*}

\begin{example*}
  The Galois group $\CC = \overline{\RR}$ over $\RR$ has two extensions, the identity
  on $\CC$ and the complex conjugation.
\end{example*}

\begin{proposition}
  If $E$ is Galois over $K$, then $|\Gal(E:K)| = [E:K]$.
\end{proposition} \vspace{1.8em}
\proof If $E \subseteq \overline{K}$ is normal over $K$, then every $K$-homomorphism of $E$
into $\overline{K}$ sends $E$ and is (as a set of ordered pairs) a $K$-automorphism of $E$.
Hence $|\Gal(E:K)| = [E:K]_{s} = [E:K}$ when $E$ is separable over $K$. \qedsymbol

\begin{definition*}
  Let $E$ be a field and let $G$ be a group of automorphisms of $E$. The fixed field
  of $G$ if $\Fix_{E}(G) = \{\alpha \in E \ | \ \varphi \alpha = \alpha \text{ for all } \varphi \in G \}$.
\end{definition*}

\begin{theorem}[Fundamental Theorem of Galois Theory]
  Let $E$ be a finite Galois extension of a field $K$. \\ \\
  If $F$ is a subfield of E that contains $K$, then $E$ is a finite Galois extension of
  $F$ and $F$ is the fixed field of $\Gal(E:F)$. \\ \\
  If $H$ is a subgroup of $\Gal(E:K)$, then $F = \Fix_{E}(H)$ is a subfield of $E$ that
  contains $K$ , and $\Gal(E:F) = H$. \\ \\
  This defines a one-to-one correspondence between intermediate fields $K \subseteq F \subseteq E$
  and subgroups of $\Gal(E:K)$.
\end{theorem} \vspace{1.8em}

%\part{The second part}
%
%\chapter{Algebras}
%
%\backmatter \appendix
%
%\chapter{The First Appendix}
%
%The \verb"\appendix" command should be used only once. Subsequent
%appendices can be created using the Chapter command.
%
%\chapter{The Second Appendix}
%%
%Some text for the second Appendix.

%%%%%%%%%%%%%%%%%%%%%%%%%%%%%%%%%%%%%%%%%%%%%%%%%%%%%%%%%%%%%%%%%%%%%%%%%%%%%%%%
%                                 BIBLIOGRAPHY                                 %
%%%%%%%%%%%%%%%%%%%%%%%%%%%%%%%%%%%%%%%%%%%%%%%%%%%%%%%%%%%%%%%%%%%%%%%%%%%%%%%%

% arXiv bibliography macro
\def\arXiv#1{\href{http://arxiv.org/abs/#1}{arXiv:#1}}

\begin{thebibliography}{0}
 
 \bibitem[Gr]{grillet} P. A. Grillet,
   \emph{Abstract Algebra (Graduate Texts in Mathematics), Second Edition. \/}
   Springer New York, 2007. \\

 \bibitem[La]{lang} Serge Lang,
   \emph{Algebra (Graduate Texts in Mathematics), Third Edition. \/}
   Springer Science \& Business Media, 2005. \\

 \bibitem[Ro]{rotman} J. J. Rotman,
   \emph{Advanced Modern Algebra (Graduate Texts in Mathematics), Third Edition, Part I. \/}
   American Mathematical Soc., 2015. \\

 \bibitem[Ed]{edwards} H. M. Edwards,
   \emph{Galois Theory (Graduate Texts in Mathematics). \/}
   Springer New York, 1997. \\

 \bibitem[El]{elman} Richard Elman,
   \emph{Lectures on Abstract Algebra (Preliminary Version). \/}
   \url{http://www.math.ucla.edu/~rse/algebra_book.pdf} .\\

 \bibitem[Ch]{chen} Evan Chen,
   \emph{Math 55a Lecture Notes. \/}
   Harvard, Fall 2014 \url{http://www.mit.edu/~evanchen/notes/Harvard-55a.pdf} .\\

 \bibitem[Re]{reeder} Mark Reeder,
   \emph{Notes on Group Theory. \/}
   Boston College, 2015 \url{https://www2.bc.edu/mark-reeder/Groups.pdf} .\\

\end{thebibliography}

\end{document}
