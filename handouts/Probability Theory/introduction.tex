\documentclass[11pt]{scrartcl}
\usepackage{evan}
\usepackage{mdframed}
\makeatletter
\let\theorem\undefined
\let\c@theorem\undefined
\let\lemma\undefined
\let\c@lemma\undefined
\makeatother
\newmdtheoremenv[outerlinewidth=2,leftmargin=20,
  rightmargin=20,backgroundcolor=white,
  outerlinecolor=blue,innertopmargin=0.5\topskip,
  splittopskip=\topskip,skipbelow=\baselineskip,
skipabove=\baselineskip]
{theorem}
{Theorem}

\newmdtheoremenv[outerlinewidth=2,leftmargin=20,
  rightmargin=20,backgroundcolor=white,
  outerlinecolor=blue,innertopmargin=0.5\topskip,
  splittopskip=\topskip,skipbelow=\baselineskip,
skipabove=\baselineskip]
{lemma}
{Lemma}


\lhead{}
\rhead{Basics in Probabitlity Theory}

\begin{document}
\title{Basics in Probability Theory}
\author{Joel Antonio-V\'asquez\plusemail{hello@joelantonio.me}}
\date{\today}
\maketitle

%\setcounter{section}{-1}
%\section{Abstract Nonsense}
\begin{definition*}
  Let $\Omega$ be a set. A \emph{boolean algebra} is a collection $\SF$ of subsets from $\Omega$ such that
  \begin{itemize}
    \item{$\emptyset \in \Omega$},
    \item{If $E, F \in \SF$, then $E \cup F$ and $E \cap F$ lie in $\SF$ (closed under union and intersection)},
    \item{If $F \in \SF$, so does $\Omega \backslash F$ (closed under complement).}
  \end{itemize}
\end{definition*}
\begin{definition*}
  A $\sigma$-algebra is a boolean algebra $\SF \in \Omega$ and closed under countable unions and intersections (i.e. Let $E_{1}, \dots, E_{n} \in \SF$, then $\bigcup\limits_{i=1}^{n} E_{i}$ and $\bigcap\limits_{i=1}^{n} E_{i}$ also lie $\SF$).
\end{definition*}
We define the Borel $\sigma$-algebra on $\Omega$ to be the $\sigma$-algebra generated by the open sets of $\Omega$ (can be also generated by the compact subsets of $\Omega$). Example, let $\SF$ be the collection of all the open subsets from the interval $[0, 1]$, then the Borel $\sigma$-algebra is the smallest $\sigma$-algebra that contains $\SF$. \textbf{Proof:} Let $\SB$ be the set generated by $\SF$, \textbf{any} union and intersection from a $\sigma$-algebra is a $\sigma$-algebra. Finally let $\SA$ be a $\sigma$-algebra that contains $\SF$, then $\SB \subset \SA$ and claim follows.
\begin{definition*}
  A measurable space is a pair $(\Omega, \SF)$, where $\SF$ is a $\sigma$-algebra from the set $\Omega$.
\end{definition*}
\begin{definition*}[Probability theory]
  Let $(\Omega, \SF)$ be a measureable space and let $\mu$ a measure on that space such that $\mu:\SF \longrightarrow [0, +\infty]$. A probability space is a triple $(\Omega, \SF, \mu)$ obeying the \textbf{Kolmogorov axioms for probability}
  \begin{itemize}
    \item{${\bf P}(\emptyset)=0$},
    \item{${\bf P}(\overline{\emptyset})=1$},
    \item{If $E_ {1}, E_{2}, \dots$ are disjoint events, then ${\bf P}\left( \bigvee_{n=1}^{\infty}E_{n} \right) = \sum_{n=1}^{\infty} {\bf P}(E_{n})$.}
  \end{itemize}
\end{definition*}

\section{Expectation}
\begin{lemma}[Fatou's Lemma]
  Let $(\Omega, \SF, \mu)$ be a measure space, and let $f_{1}, f_{2}, \dots: \Omega \longrightarrow [0, +\infty]$ be a sequence of unsigned measurable function. Then
  $$
  \int_{\Omega} \liminf_{n \to \infty} f_{n} d\mu \leq \liminf_{n \to \infty} \int_{\Omega} f_{n} d\mu.
  $$ 
\end{lemma} 
\textbf{Proof:}
Let $g$ be an unsigned simple function (i.e. $g = \sum_{i=1}^{k} a_{i}1_{E_{i}}$ where $a_{i} > 0$ and $1_{E_{i}}$ be the indicator function for disjoint events $E_{i}$) such that $g \leq \liminf\limits_{n \to \infty} f_{n}$. It's enough to show that 
$$
(1 - \eps) \int_{\Omega} g \leq \liminf_{n \to \infty} \int_{\Omega} f_{n} d\mu
$$
by definition of the unsigned integral for any $0 < \eps < 1$. Writting as a sumation
$$
\int_{\Omega} f_{N} d\mu \geq \sum_{i=1}^{k} (1 - \eps)a_{i}\mu(E_{i, N})
$$
Where $N$ goes to $+\infty$ and $1_{E_{i, N}} = \mu(E_{i, N})$, taking the limit inferior
$$
\liminf_{N \to \infty} \int_{\Omega} f_{N} d\mu \geq \sum_{i=1}^{k} (1 - \eps)a_{i}\mu(E_{i})
$$
becuase $\mu(E_{i, N}) \to \mu(E_{i})$ as $N \to \infty$. Since the right-hand side is $(1 - \eps) \int_{\Omega} g d\mu$, and the claim follows. \\ \\
Rewriting the Fatou's lemma for random variables

\begin{lemma}[Fatou's lemma for random variables]
  Let $X_{1}, X_{2}, \dots$ be a sequence of unsigned random variables. Then
  $$
  {\bf E} \liminf_{n \to \infty} X_{n} \leq \liminf_{n \to \infty} {\bf E}X_{n}.
  $$
\end{lemma}
Now, let's establish Monotone convergence theorem
\begin{theorem}[Monotone convergence theorem]
  Let $(\Omega, \SF, \mu)$ be a measure space, and let $f_{1}, f_{2}, \dots: \Omega \longrightarrow [0, +\infty]$ be a sequence of unsigned measurables functions which is monotone increasing such that $f_{n}(\omega) \leq f_{n+1}(\omega)$for all $n$ and $\omega \in \Omega$. Then
  $$
  \int_{\Omega} \lim_{n \to \infty} f_{n} d\mu = \lim_{n \to \infty} \int_{\Omega} f_{n} d\mu.
  $$
\end{theorem} \\ \\ 
\textbf{Proof:}
Recalling the Fatou's lemma (only for the limit case), we have 
  $$
  \int_{\Omega} \lim_{n \to \infty} f_{n} d\mu \leq \lim_{n \to \infty} \int_{\Omega} f_{n} d\mu.
  $$ 

  By continuity below (i.e. if $E_{1} \supset E_{2} \supset \dots$ are measurable, then $\mu(\bigcup_{n=1}^{\infty} E_{n}) = \lim_{n \to \infty} \mu(E_{n})$), then there is a $m$ that goes to $+\infty$ such that
  $$
  \int_{\Omega} \lim_{n \to \infty} f_{n} d\mu \geq \int_{\Omega} f_{m} d\mu,
  $$
  since $f_{n}$  are indicator functions, then claim follows. \\ \\
  One more time, rewriting for random variables
  \begin{theorem}[Monotone convergence theorem for random variables]
    Let $0 \leq X_{1} \leq X_{2} \leq \dots$ be a monotone non-decreasing sequence of unsigned random variables. Then
    $$
    {\bf E} \lim_{n \to \infty} X_{n} = \lim_{n \to \infty} {\bf E} X_{n}.
    $$
  \end{theorem} \\ \\
  Suppose that $f_{i}: \Omega \longrightarrow \CC$ for $i = 1, 2, \dots$ and let $g: \Omega \longrightarrow [0, +\infty]$ which \emph{dominates} $f_{n}$ in the sense of $|f_{n}(\omega)| \leq |g(\omega)|$ for all $n$ and all $\omega $, What can one say about it? (see Problem 3).

\section{Products measures}
Let $(\Omega_{i}, \SF_{i}, \mu_{i})$ for $i= 1,2$ be probability spaces. Let $\SE_{1} \in \SF_{1}$ and $\SE_{2} \in \SF_{2}$, then $\mu_{1} \times \mu_{2}(\SE_{1} \times \SE_{2}) = \mu_{1}(\SE_{1})\mu_{2}(\SE_{2})$.

\begin{theorem}[Tonelli's theorem]
  Let $f: \Omega_{1} \times \Omega_{2} \longrightarrow [0, +\infty]$ be a measurable function, then
  \begin{enumerate}[1.]
    \item{$f_{\omega_{1}}$ is measurable for almost all $\omega_{1}$ and $f_{\omega_{2}}$ is measurable for almost all $\omega_{2}$.}
    \item{$\omega_{1} \mapsto \int_{\Omega_{2}} f(\omega_{1}, \omega_{2})d\mu_{2}(\omega_{2})$ is measurable as a function of $\omega_{1}$}.
    \item{$\omega_{2} \mapsto \int_{\Omega_{1}} f(\omega_{1}, \omega_{2})d\mu_{1}(\omega_{1})$ is measurable as a function of $\omega_{2}$}.
    \item{}
  $$
  \int_{\Omega_{1} \times \Omega_{2}} f d(\mu_{1} \times \mu_{2}) = \int_{\Omega_{1}} \left( \int_{\Omega_{2}} f(\omega_{1}, \omega_{2}) d\mu_{2}(\omega_{2})  \right) d\mu_{1}(\omega_{1})
  $$
\end{enumerate}
For each $\omega_{i} \in \Omega_{i}$ for $i= 1, 2$. The equality holds if we invert the integrals at (4).
\end{theorem} \\ \\
\textbf{Proof:}
Since $\mu_{1}, \mu_{2}$ are $\sigma$-algebra so is $(\mu_{1} \times \mu_{2})$. Let $S_{n}$ be a simple function, replacing with $S_{n}1_{\Omega_{1} \times \Omega_{2}}$ such that increases to $f$ for all $n$. For monotone convergence theorem claim follows. \\ \\

\section{Problems}
\begin{enumerate}
    \ii Show the linearity for the complex case when $f, g: \Omega \longrightarrow \CC$ 
    $$
    \int_{\Omega} f + g \ d\mu = \int_{\Omega} f d\mu + \int_{\Omega} g d\mu,
    $$
    and for a $c \in \CC$, then
    $$
    \int_{\Omega} cf \ d\mu = c\int_{\Omega} fd\mu.
    $$
    \ii Show that 
    $$
    \int_{\Omega} \min(f, n) d\mu \to \int_{\Omega} f d\mu.
    $$
    as $n \to \infty$ and $f: \Omega \to [0, +\infty]$ is unsigned measurable.
    \ii (Dominated convergence theorem) Let $(\Omega, \SF, \mu)$ be a measure space and let $f_{1}, f_{2}, \dots: \Omega \longrightarrow \CC$ be measurables functions which converges pointwise to some limit. Let $g: \Omega \longrightarrow [0, +\infty]$ which dominates $f_{n}$ in the sense that $|f_{n}(\omega)| \leq |g(\omega)|$ for all $n$ and all $\omega$. Show that 
    $$
    \int_{\Omega} \lim_{n \to \infty} f_{n} d\mu = \lim_{n \to \infty} \int_{\Omega} f_{n} d\mu.
    $$
    \ii (Fubini's theorem) If $f: \Omega_{1} \times \Omega_{2} \longrightarrow \CC$ is absoutely integrable, and
    \begin{enumerate}[I.]
    \ii{$f_{\omega_{1}}$ is measurable for almost all $\omega_{1}$ and $f_{\omega_{2}}$ is measurable for almost all $\omega_{2}$,}
    \ii{$\omega_{1} \mapsto \int_{\Omega_{2}} f(\omega_{1}, \omega_{2})d\mu_{2}(\omega_{2})$ is measurable as a function of $\omega_{1}$},
    \ii{$\omega_{2} \mapsto \int_{\Omega_{1}} f(\omega_{1}, \omega_{2})d\mu_{1}(\omega_{1})$ is measurable as a function of $\omega_{2}$}.
    \end{enumerate}
    Then, prove that
    $$
    \int_{\Omega_{1} \times \Omega_{2}} f d(\mu_{1} \times \mu_{2}) = \int_{\Omega_{1}} \left( \int_{\Omega_{2}} f(\omega_{1}, \omega_{2}) d\mu_{2}(\omega_{2})  \right) d\mu_{1}(\omega_{1}) 
    $$
    \ii Let $(\Omega_{i}, \SF_{i}, \mu_{i})_{i \in A}$ be a collection of probability spaces. Show that
    $$
    \prod_{i \in A} \mu_{i} = \left( \prod_{i \in A_{1}} u_{i}  \right) \times \left( \prod_{i \in A_{2}} u_{i}  \right)
    $$
    for any partition $A = A_{1} \uplus  A_{2}$.
\end{enumerate}

\section{Further Links}
\begin{itemize}
    \ii Terence Tao's notes on Probability Theory: \url{https://terrytao.wordpress.com/category/teaching/275a-probability-theory/}
    \ii Christopher King's notes on Probability Theory: \url{http://www.hamilton.ie/ollie/Downloads/ProbMain.pdf}
\end{itemize}


\end{document}
