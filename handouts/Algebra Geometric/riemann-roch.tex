\documentclass[11pt]{scrartcl}
\usepackage{evan}
\usepackage{mdframed}
\makeatletter
\let\theorem\undefined
\let\c@theorem\undefined
\let\lemma\undefined
\let\c@lemma\undefined
\makeatother
\newmdtheoremenv[outerlinewidth=2,leftmargin=20,
  rightmargin=20,backgroundcolor=white,
  outerlinecolor=blue,innertopmargin=0.5\topskip,
  splittopskip=\topskip,skipbelow=\baselineskip,
skipabove=\baselineskip]
{theorem}
{Theorem}

\newmdtheoremenv[outerlinewidth=2,leftmargin=20,
  rightmargin=20,backgroundcolor=white,
  outerlinecolor=blue,innertopmargin=0.5\topskip,
  splittopskip=\topskip,skipbelow=\baselineskip,
skipabove=\baselineskip]
{lemma}
{Lemma}


\lhead{}
\rhead{Riemann-Roch Theorem}

\newcommand{\CL}{\mathcal L}
\newcommand{\CR}{\mathcal R}
\newcommand{\card}{\text{card}}
\DeclareMathOperator{\Null}{null}
\DeclareMathOperator{\rk}{rk}

\begin{document}
\title{Riemann-Roch Theorem}
\author{Joel Antonio-V\'asquez\plusemail{hello@joelantonio.me}}
\date{\today}
\maketitle

%\setcounter{section}{-1}
%\section{Abstract Nonsense}
\begin{definition*}
  A domain is a commutative ring $\CR \neq 0$, with identity, such that for any $x, y \neq 0 \in \CR$ implies $xy \neq 0$.
\end{definition*}

\begin{definition*}
  Let $I$ be a subset of a ring $\CR$. Then $I$ is called an ideal if for all $y \in \CR$ and every $x \in I$ implies that $xy \in I$ and $yx \in I$.
\end{definition*}

\begin{definition*}
  A principal ideal domain (PID) is a domain in which every ideal is principal (i.e. an ideal generated by a single element).
\end{definition*}

\begin{theorem}
  \label{theo:1}
  Let $M$ over a PID $\CR$. There is a unique decresing sequence of proper ideals
  $$
  d_{1} \supseteq \dots \supseteq d_{n}
  $$
  such that $M$ is isomorphic to the sum of cyclic modules
  $$
  M \cong \bigoplus_{i} \CR/(d_{i}).
  $$
  The $d_{i}$s are called invariant factors of $M$.
\end{theorem}
\textbf{Proof: }Let $\varphi$ be a $\CR$-linear map such that can be determined by $\varphi(e_{1}) = f_{1}, \dots, \varphi(e_{n}) = f_{n}$
where $e_{1}, \dots, e_{n}$ is the basis of $n$-dimensional $\CR$. Then $\varphi(e_{j}) = \sum_{i=1}^{n} c_{ij}e_{i}$, such that $(c_{ij})$ is the
matrix presentation of $\varphi$ with respect to the basis. Then
$$
\varphi(\CR) = \CR\varphi(e_{1}) \oplus \dots \oplus \CR\varphi(e_{n}) = \CR f_{1} \oplus \dots \oplus \CR f_{n},
$$
by aligned bases of $\CR$ and its module $\varphi(\CR)$, we can say that
$$
\CR = \CR v_{1} \oplus \dots \oplus \CR v_{n}, \hspace{3em} \varphi(R) = \CR a_{1}v_{1} \oplus \dots \oplus \CR a_{n}v_{n},
$$
where $a_{i}$s are nonzero integers. Then
$$
\CR/\varphi(R) \cong \bigoplus_{i} \CR/a_{i}\CR.
$$
Obvioulsy, $\CR/\varphi(R)$ is our $M$ and claim follows. \qedsymbol \\ \\
As an useful comment, we can calculate the invariant factors with the Smith
Normal Form (SNF) (see problem 1).

\section{Submodules}
We need to remember that no every module has a basis, that's because
we use free module here.
\begin{definition*}
  A free module is a module with a basis.
\end{definition*}
\begin{lemma}
  \label{lem:1}
Let $\CR$ be a $n$-dimensional module over a PID, then every
$\CR$-submodule of $\CR$ is an ideal.
\end{lemma}
\textbf{Proof: }Let $x \neq 0 \in \CR$, then for $\CR x$ since all ideals in
$\CR$ are principal, it's clearly that $\CR x \cong \CR$ as $\CR$-modules.
\qedsymbol
\begin{lemma}
  \label{lem:2}
  Let $\CR$ be a commutative ring and $M$ be an $R$-module. Let $f$ be
  an $\CR$-linear and onto map such that $f: M \longrightarrow \CR$, then
  there is an $\CR$-module isomorphism $h: M \cong \CR^{n} \oplus \Ker f$
  where $h(m) = (f(m), *)$, making $f$ the first component of $h$.
\end{lemma}
\textbf{Proof: }Let $\CR^{n} = \CR e_{1} \oplus \dots \oplus \CR e_{n}$ where
$e_{1}, \dots, e_{n}$ is the basis of $\CR$, let $m_{i} \in M$ such that
$f(m_{i}) = e_{i}$ then there is a map $g: \CR^{n} \longrightarrow M$ such that
$$
g(c_{1}e_{1} + \dots + c_{n}e_{n}) = c_{1}m_{1} + \dots + c_{n}m_{n},
$$
Now, we define the function $h: M \longrightarrow \CR^{n} \oplus \Ker f$ such that
$h(m) = (f(m), m - g(f(m)))$. \qedsymbol
\begin{theorem}
  \label{theo:2}
  Let $M \subset \CR$ be a free $\CR$-module of rank $n$ where $\CR$ is a PID,
  then for any $S$ submodule of $M$ is free of rank $\leq n$.
\end{theorem}
\textbf{Proof: }The free $\CR$-module is $\CR^{n}$ by lemma \ref{lem:1}. By induction on $n$, let $S \subset \CR^{n+1}$ be a submodule.
We gonna show that $S$ is free of rak $\leq n+1$. The a projection of direct
sum $\phi: \CR \oplus \CR^{n} \longrightarrow \CR^{n}$ (i.e. $\CR^{n+1} = \CR \oplus \CR^{n}$), then $N = \phi(S) \subset \CR^{n}$ is free of rank $\leq n$.
Now, by lemma \ref{lem:2}
$$
S \cong N \oplus \Ker \phi \vert_{S},
$$
so $N \oplus \Ker \phi \vert_{S}$ is free of rank $\leq n+1$, so $S$ does. \qedsymbol

\section{Cardinality}
\begin{definition*}
  Let $\CR$ be a module and let $x \in \CR$, which is called a torsion
  element if there exists a nonzero $r \in \CR$ such that $rx = 0$. If $rx \neq 0$ for all $r \neq 0 \in R$, then the element $x$ is called a torsion-free.
\end{definition*}

\begin{definition*}
  Let $T$ be a module, we say that $T$ is called a torsion-free module, if every
  element of $T$ is a torsion-free module.
\end{definition*}

\begin{definition*}
  Let $T$ be a finitely torsion module over the PID $\CR$. By theorem \ref{theo:1},
  we write $T \cong R/(d_{1}) \oplus \dots \oplus R/(d_{m})$, then the $\CR$-cardinality of $T$ to be the ideal
  $$
  \card_{\CR}(T) = (d_{1}d_{2} \dots d_{m}).
  $$
\end{definition*}

\begin{theorem}
  \label{theo:3}
  Let $T_{1}$ and $T_{2}$ be two finitely generated torsion $\CR$-modules, then
  $$
  \card_{\CR}(T_{1} \oplus T_{2}) = \card_{\CR}(T_{1}) \card_{\CR}(T_{2}).
  $$
\end{theorem}
\textbf{Proof: }We combine cyclic decompositions of $T_{1}$ and $T_{2}$
and then get $T_{1} \oplus T_{2}$. \qedsymbol \\ \\
If we pick $x_{1}, \dots, x_{n}$ the generating set for a torsion-free module $T$ as an $\CR$-module,
then we have a linear map $f: \CR^{n} \longrightarrow T$ where $f(e_{i}) = x_{i}$
for the basis $e_{1}, \dots, e_{n}$ of $\CR^{n}$ such there exists a linearly
indepedent sequence $y_{1}, \dots, y_{n}$ of $T$ such that $y_{j} = \sum_{i=1}^{n} a_{ij}x_{i}$ with $a_{ij} \in \CR$. By zorn's lemma, there is a linearly
independent subset of $T$ with maximal size $t_{1}, \dots, t_{d}$ such that
$\sum_{j=1}^{d} At_{j} \cong T^{d}$. Then we can get an isomorphism map
$$
T \rightarrow aT \hookrightarrow \sum_{j=1}^{d} Tt_{j} \rightarrow A^{d},
$$
for a linearly dependent set $x, t_{1}, \dots, t_{d}$ and a nontrivial linear
realtion $ax + \sum_{i=1}^{d} a_{i}t_{i} = 0$ with $a \neq 0$. Now, we can
say the following
\begin{lemma}
  \label{lem:3}
  Let $T$ be a finitely generated torsion-free module over a PID $\CR $ such that
  $T \neq 0$, then there is an embedding $T \hookrightarrow \CR^{d}$ for some
  $d \geq 1$ such that the image of $T$ intersects each standard coordinate
  axis of $\CR^{d}$.
\end{lemma}
Now, we use the above lemma to formulate the next theorem
\begin{theorem}
  \label{theo:4}
  Let $\CR$ be a PID, then every finitely generated torsion-free $\CR$-module
  is a free $\CR$-module.
\end{theorem}
\textbf{Proof: }By lemma \ref{lem:3}, there is a module that embeds a finite
free $\CR$-module, then it's finite free too by theorem \ref{theo:2}. \qedsymbol \\ \\
As last, we have the following theorem
\begin{theorem}
  \label{lem:5}
  Let $\CR$ be a PID, every finitely $\CR$-module has the form $F \oplus T$ where
  $F$ is a finite free $\CR$-module and $T$ is a finitely generated torsion
  $\CR$-module. Moreover, $T \cong \bigoplus\limits_{j} \CR/(a_{j})$ with a nonzero $a_{j}$.
\end{theorem}
\textbf{Proof: }Let $T$ be a finitely generated $\CR$-module, with generators
$x_{1}, \dots, x_{n}$. We define $f:\CR^{n} \longrightarrow T$ by $f(e_{i}) = x_{i}$. We know that 
$$
\CR^{n}/N \cong \left(\bigoplus\limits_{j}^{m} \CR/(a_{j})\right) \oplus \CR^{n-m},
$$
for some $m \leq n$, a quotient $\CR^{n}/N$ and nonzero $a_{j}$s. The direct
sum of the $A/(a_{j})$'s is a torsion module and $\CR^{n-m}$ is a finite
free $\CR$-module. \qedsymbol
\section{Problems}
\begin{enumerate}[1.]
    \ii Describre, as a direct sum of cyclic groups, the cokernel $\varphi: \ZZ^{3} \longrightarrow \ZZ^{3}$ given by left multiplication by the matrix
    $$
    \left[ \begin{matrix}
      30 & 9 & 18 \\
      15 & 6 & 6 \\
      18 & 3 & 27
    \end{matrix} \right].
    $$
    (Hint: use \href{http://math.stackexchange.com/questions/1797112/describe-as-a-direct-sum-of-cyclic-groups-given-a-map-phi-mathbbz3-l}{SNF})
    \ii Let $M$ be a finitely generated $\CR$-module with submodules $S \subset N$
    such that $M/N$ is a torsion module. Show that
    $$
    [M:N]_{\CR} = [M:S]_{\CR}[S:N]_{\CR}.
    $$
    \ii Let $\CR$ be a PID. Show that a finitely generated $\CR$-module $M$
    is a torsion module iff there is some $r \neq 0$ in $\CR$ such that
    $r\CR = 0$.
    \ii Does theorem $\ref{theo:1}$ works for any kind of module? (Hint: First 
    think with free modules and after try with non-free modules).
    \ii What does SNF give us? (Hint: see \href{http://joelantonio.me/post/2016/06/02/Smith-Normal-Form-as-t    he-best-presentation-group/}{SNF})
\end{enumerate}
\section{Further Links}
\begin{itemize}
    \ii Keith Conrad's notes on Modules over a PID: \url{http://www.math.uconn.edu/~kconrad/blurbs/linmultialg/modulesoverPID.pdf} 
    \ii Prasad Senesi's notes on Modules over a Principal Ideal Domain: \url{http://math.ucr.edu/~prasad/PID%20mods.pdf} 
 \end{itemize}

\end{document}
